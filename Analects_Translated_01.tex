\documentclass{easternClassics}
\usepackage{kotex}

\setmainfont     {Noto Serif KR}[Scale=3.25,Script=Hangul]
\setmainhanjafont{Noto Serif JP}[Scale=3.25]
\setCharOption{main}{hangul}{xshift=0.02cm}
\setCharOption{comment}{}{scale=0.7,yscale=1.05}
\setCharOption{center} {}{scale=0.7}

\四周雙邊{0.05}{0.15}{0.2}
\半郭{23}{17.5}
\有界{0.05}
\半葉{19}{10}
\黑口{1/3}{4.5}{4.5}
\魚尾{trefoil}{trefoil}
\版心{1.33}
\版心字間{1}
\細註行間比{8/9} 
\版心題位置{0.8}
\張次位置{0.8}

\def\券#1#2{\文[parsing.assistIdeograph(3)]{[chapter|#1]#2}\文{[newColumn]}}
\def\篇#1  {\文[parsing.assistIdeograph(3)]{[newBlock]  #1}\文{[newColumn]}}
\def\原    {\文[parsing.assistIdeograph(3)]}
\def\諺#1  {\文{[indent|1]}\文[parsing.assistIdeograph(1)]{#1}\文{[indent|-1]}}

\begin{document}
\券{論語諺解卷之一}{論론語어〯諺언〯解ᄒᆡ〯卷권〯之지一일〮}
\篇{學ᄒᆞᆨ〮而ᅀᅵ第뎨〯一일〮}
\原{子ᄌᆞ〮ㅣ曰왈〮學ᄒᆞᆨ〮而ᅀᅵ時시習습〮之지면不블〮亦
  역〮說열〮乎호아}
\諺{子ᄌᆞ〮ㅣᄀᆞᆯᄋᆞ샤〮ᄃᆡ〮學ᄒᆞᆨ〮ᄒᆞ고〮時시[?|로]習습〮ᄒᆞ[?|면〮]
  ᄯᅩ〮ᄒᆞᆫ깃브디〮ᄋᆞ니〮ᄒᆞ랴〮}
\原{有유〯朋부ᇰ이自ᄌᆞ〮遠원〯方방來ᄅᆡ면不블〮亦역〮樂
  락〮乎호아}
\諺{버〯디〮遠원〯方바ᇰ으〮로브터〮오면〮ᄯᅩ〮ᄒᆞᆫ즐〮겁〮[?|디〮아]
  니〮ᄒᆞ랴〮}
\原{人ᅀᅵᆫ不블〮知디而ᅀᅵ不블〮溫온〯이면不블〮亦역〮君군〮子
  ᄌᆞ〮乎호아}
\諺{사〯ᄅᆞᆷ〮이〮아〯디〮몯〯ᄒᆞ〮야〮도〮慍온〯티〮아니〮ᄒᆞ면ᄯᅩ〮ᄒᆞᆫ
  君군〮子ᄌᆞ〮ㅣ아니〮가〮}
\原{○有유〯子ᄌᆞ〮ㅣ曰왈〮其기爲위人ᅀᅵᆫ也야〯ㅣ孝효〯
  弟뎨〯오而ᅀᅵ好호〯犯범〯上샤ᇰ〯者쟈〮ㅣ鮮션〯矣의〯니
  不블〮好호〯犯범〯上샤ᇰ〯이오而ᅀᅵ好호〯作작〮亂란〯者쟈〮
  ㅣ未미〯之지有유〯也야〯ㅣ니라}
\諺{有유〯子ᄌᆞ〮ㅣᄀᆞᆯ오〮ᄃᆡ〮그사〯ᄅᆞᆷ〮이론디〮孝효〯ᄒᆞ[?|며]
  弟뎨〯ᄒᆞ고〮上샤ᇰ〯을犯범〯홈〯을〮好호〯ᄒᆞᆯ者쟈〮ㅣ젹〯
  으니〮上샤ᇰ〯을〮犯범〯홈〯을〮好호〯티〮아니〮ᄒᆞ고〮亂란〯
  [?|을〮]作작〮홈〯을〮好호〯ᄒᆞᆯ者쟈〮ㅣ잇디〮아니〮ᄒᆞ니〮[?|라〮]}
\原{君군子ᄌᆞ〮ᄂᆞᆫ務무〯本본〮이니本본〮立립〮而ᅀᅵ道도〯生
  ᄉᆡᇰᄒᆞᄂᆞ니孝효〯弟뎨〯也야〯者쟈〮ᄂᆞᆫ其기爲위仁ᅀᅵᆫ之
  지本본〮與여ᅟᅵᆫ뎌}
\諺{君군子ᄌᆞ〮ᄂᆞᆫ本본〮을〮힘〮ᄡᅳᆯ〮디〮니本본〮이[?|셤〯애〮]道
  도〯ㅣ生ᄉᆡᇰᄒᆞ〮ᄂᆞ니〮孝효〯第뎨〯ᄂᆞᆫ〮그仁ᅀᅵᆫᄒᆞ[?|욜]本
  본〮인뎌〮}
\原{○子ᄌᆞ〮ㅣ曰왈〮巧교〯言언令려ᇰ〮色ᄉᆡᆨ〮이鮮션〯矣의〯
  仁ᅀᅵᆫ이니라}
\諺{子ᄌᆞ〮ㅣᄀᆞᆯᄋᆞ〮샤ᄃᆡ〮言언을〮巧교〯히〮ᄒᆞ며〮色ᄉᆡᆨ〮을〮
  令려ᇰ〮히〮ᄒᆞᆯ이〮仁ᅀᅵᆫᄒᆞᆯ이〮鮮션〯ᄒᆞ니〮라〮}
\原{○曾즈ᇰ子ᄌᆞ〮ㅣ曰왈〮吾오ㅣ日ᅀᅵᆯ〮三삼省[?|셔ᇰ〮]吾[?|오]}
\原{身신ᄒᆞ노니爲위〮人ᅀᅵᆫ謀모而ᅀᅵ不블〮忠튜ᇰ乎호아
  與여〯朋브ᇰ友우〯交교而ᅀᅵ不블〮信신〯乎호아傳뎐
  不블〮習습〮乎호애니라}
\諺{曾즈ᇰ子ᄌᆞ〮ㅣᄀᆞᆯᄋᆞ〮샤ᄃᆡ〮내〮날〮로세〯가지〮로내몸〮
  을〮ᄉᆞᆯ피〮노니〮사〯[?|ᄅᆞᆷ〮]을〮爲위〮ᄒᆞ〮야〮謀모홈〯애〮忠튜ᇰ
  티〮몯〯ᄒᆞᆫ가〮朋브ᇰ友우〯로〮더브〮러〮交교홈〯애〮信신〯
  티〮몯〯ᄒᆞᆫ가〮傳뎐코〮習습〮디〮몯〯ᄒᆞᆫ개〮니라〮}
\原{○子ᄌᆞ〮ㅣ曰왈〮道도〯千쳔乘스ᇰ〯之지國국〮호ᄃᆡ敬겨ᇰ〯
  事ᄉᆞ〯而ᅀᅵ信신〯ᄒᆞ며節졀〮用요ᇰ〯而ᅀᅵ愛ᄋᆡ〯人ᅀᅵᆫᄒᆞ며使
  ᄉᆞ〯民민以이〯時시니라}
\諺{子ᄌᆞ〮ㅣᄀᆞᆯᄋᆞ〮샤〮ᄃᆡ〮千쳔乘스ᇰ〯ㅅ나라〮흘〮道도〯호〮
  ᄃᆡ〮일〯을〮敬겨ᇰ〯ᄒᆞ고〮信신〯ᄒᆞ며〮ᄡᅳ〮기를〮節졀〮ᄒᆞ고〮
  사〯ᄅᆞᆷ〮을〮愛ᄋᆡ〯ᄒᆞ며〮民민을〮브〮료ᄃᆡ〮時시로〮[?|ᄡᅥ〮]홀〯
  띠〮니라〮}
\原{○子ᄌᆞ〮ㅣ曰왈〮弟뎨〯子ᄌᆞ〮ㅣ入ᅀᅵᆸ〮則즉〮孝효〮[?|ᄒᆞ고]出
  츌〮則즉〮弟뎨〯ᄒᆞ며謹근〯而ᅀᅵ信신〯ᄒᆞ며汎범〯愛ᄋᆡ〯衆쥬ᇰ〯
  호ᄃᆡ而ᅀᅵ親친仁ᅀᅵᆫ이니行ᄒᆡᇰ有유〯餘여力력〮이어든則
  즉〮以이〯學ᄒᆞᆨ〮文문이니라}
\諺{子ᄌᆞ〮ㅣᄀᆞᆯᄋᆞ〮샤〮ᄃᆡ〮弟뎨〯子ᄌᆞ〮ㅣ드〮러ᄂᆞᆫ〮孝효〯ᄒᆞ
  고〮나〮ᄂᆞᆫ〮弟뎨〯ᄒᆞ며〮謹근〯ᄒᆞ고〮信신〯ᄒᆞ며[?|너비〮]衆
  쥬ᇰ〯을〮愛ᄋᆡ〯호〯ᄃᆡ〮仁ᅀᅵᆫ을〮親친히〮홀〯띠〮니〮行ᄒᆡᇰ홈〯
  애〮남은〮힘〮이〮잇거든〮곧〮ᄡᅥ〮글〮을〮學ᄒᆞᆨ〮홀〯띠〮니라}
\原{○子ᄌᆞ〮夏하〯ㅣ曰왈〮賢현賢현호ᄃᆡ易역〮色ᄉᆡᆨ〮ᄒᆞ며事
  ᄉᆞ〯父부〮母모〯호ᄃᆡ能느ᇰ竭갈〮其기力력〮ᄒᆞ며事ᄉᆞ〯君군
  호ᄃᆡ〮能느ᇰ致티〯其기身신ᄒᆞ며與여〯朋브ᇰ友우〯交교호ᄃᆡ
  言언而ᅀᅵ有유〯信신〯이면雖슈曰왈〮未미〯學ᄒᆞᆨ〮이라도
  吾오必필〮謂위〮之지學ᄒᆞᆨ〮矣의〯라호리라}
\諺{子ᄌᆞ〮夏하〯ㅣᄀᆞᆯ오〮ᄃᆡ〮어〮딘〮이〮를〮어〮딜〮이〮너교〮ᄃᆡ〮
  色ᄉᆡᆨ〮을〮밧고〮며〮父부〮母모〯를〮셤교〮ᄃᆡ〮能느ᇰ히〮그
  힘〮을〮竭갈〮ᄒᆞ며〮님〯금〮을〮셤교〮ᄃᆡ〮能느ᇰ히〮그몸을
  致티〯ᄒᆞ며〮朋브ᇰ友우〯로〮더브〮러〮交교호〯ᄃᆡ〮言언
  홈〯애〮信신〯이〮이시면〮비록〮學ᄒᆞᆨ〮디〮몯〯ᄒᆞ〮얏〮다〮닐
  어〮도〮나〮ᄂᆞᆫ〮반〮ᄃᆞ〮시〮學ᄒᆞᆨ〮ᄒᆞ〮얏〮다〮닐오〮리라〮}
\原{○子ᄌᆞ〮ㅣ曰왈〮君군子ᄌᆞ〮ㅣ不블〮重듀ᇰ〯則즉〮不블〮
  威위니學ᄒᆞᆨ〮則즉〮不블〮固고〯ㅣ니라}
\諺{子ᄌᆞ〮ㅣᄀᆞᆯᄋᆞ샤〮ᄃᆡ〮君군子ᄌᆞ〮ㅣ重듀ᇰ〯티〮아니〮ᄒᆞ
  면〮威위티〮아니〮ᄒᆞ〮ᄂᆞ〮니〮學ᄒᆞᆨ〮ᄒᆞ〮면〮固고〯티〮몯〯ᄒᆞ〮
  ᄂᆞ〮니라〮}
\原{主쥬〮忠튜ᇰ信신〯ᄒᆞ며}
\諺{忠튜ᇰ信신〯으〮로〮主쥬〮ᄒᆞ며〮}
\原{無무友우〯不블〮如ᅀᅧ已긔〮者쟈〮ㅣ오}
\諺{已긔〮곤〮디〮몯〯ᄒᆞᆫ이〮를〮友우〯티〮말〮오〮}
\原{過과〮則즉〮勿믈〮憚탄〯改ᄀᆡ〯니라}
\諺{過과〮ㅣ어든〮改ᄀᆡ〯홈〯을〮憚탄〯티〮말을〮띠〮니라〮}
\原{○會즈ᇰ子ᄌᆞ〮ㅣ曰왈〮愼신〯終죠ᇰ追튜遠원〯이면民민
  德덕〮이歸귀厚후〯矣의〯리라}
\諺{會즈ᇰ子ᄌᆞ〮ㅣᄀᆞᆯᄋᆞ샤〮ᄃᆡ〮終죠ᇰ을〮愼신〯ᄒᆞ며〮遠원〯
  을〮追튜ᄒᆞ면〮民민의〮德덕〮이厚후〯에〮歸귀ᄒᆞ리〮
  라}
\原{○子ᄌᆞ〮禽금이問문〯於어子ᄌᆞ〮貢고ᇰ〯曰왈〮夫부子
  ᄌᆞ〮ㅣ至지〮於어是시〯邦바ᇰ也야〯ᄒᆞ샤必필〮聞문其기
  政져ᇰ〮ᄒᆞ시ᄂᆞ니求구之지與여아抑억〮與여〯之지與여
  아}
\諺{子ᄌᆞ〮禽금이子ᄌᆞ〮貢고ᇰ〯의게〯무러〮ᄀᆞᆯ오〮ᄃᆡ〮夫부
  子ᄌᆞ〮ㅣ이〮邦바ᇰ에〮니르〮샤반〮ᄃᆞ〮시그政져ᇰ〮을〮드
  르시〮ᄂᆞ니〮求구ᄒᆞ〮시〮ᄂᆞ냐〮與여〯ᄒᆞ〮ᄂᆞ냐〮}
\原{子ᄌᆞ〮貢고ᇰ〯이曰왈〮夫부子ᄌᆞ〮ᄂᆞᆫ溫온良랴ᇰ恭고ᇰ儉
  검〯讓ᅀᅣᇰ〯以이〯得득〮之지시니夫부子ᄌᆞ〮之지求구之
  지也야〯ᄂᆞᆫ其기諸져異이〯乎호人ᅀᅵᆫ之지求구之
  지與여ᅟᅵᆫ뎌}
\諺{子ᄌᆞ〮貢고ᇰ〯이ᄀᆞᆯ오〮ᄃᆡ〮夫부子ᄌᆞ〮ᄂᆞᆫ溫온ᄒᆞ〮시며
  良랴ᇰᄒᆞ〮시며恭고ᇰᄒᆞ〮시며儉검〯ᄒᆞ〮시며讓ᅀᅣᇰ〯ᄒᆞ〮
  시모로〮ᄡᅥ〮得득〮ᄒᆞ〮시ᄂᆞ니〮夫부子ᄌᆞ〮의〮求구ᄒᆞ
  시ᄆᆞᆫ〮그사〯ᄅᆞᆷ의〮求구홈〯애〮다ᄅᆞ신〮뎌}
\原{○子ᄌᆞ〮ㅣ曰왈〮父부〮在ᄌᆡ〯예觀관其기志지〮오父
  부〮沒몰〮에觀관其기行ᄒᆡᇰ〯이나三삼年년을無무改
  ᄀᆡ〯於어父부〮之지道도〯ㅣ라ᅀᅡ可가〯謂위〮孝효〯矣의〯
  니라}
\諺{子ᄌᆞ〮ㅣᄀᆞᆯᄋᆞ샤〮ᄃᆡ〮父부〮ㅣ在ᄌᆡ〯홈〯애〮그志지〮를〮
  보고〮父부〮ㅣ沒몰〮홈〯애〮그行ᄒᆡᇰ〯을〮볼띠〮나三삼
  年년을〮父부〮의道도〯애〮고티〮미업〯세〮ᅀᅡ〮可가〯히
  孝효〯ㅣ라닐을이〮니라〮}
\原{○有유〯子ᄌᆞ〮ㅣ曰왈〮禮례〮之지用요ᇰ〯이和화ㅣ爲
  위貴귀〯ᄒᆞ니先션王와ᇰ之지道도〯ㅣ斯ᄉᆞ爲위美ㅣ
  라小쇼〯大대〯由유之지니라}
\諺{有유〯子ᄌᆞ〮ㅣᄀᆞᆯ오〮ᄃᆡ〮禮례〮의〮用요ᇰ〯이〮和화ㅣ貴
  귀〯ᄒᆞ니〮先션王와ᇰ의〮道도〯ㅣ이〮아ᄅᆞᆷ〮다온〮디〮라
  小쇼〯와大대〯ㅣ말ᄆᆡ〮암으〮니라〮}
\原{有유〯所소〯不블〮行ᄒᆡᇰᄒᆞ니知디和화而ᅀᅵ和화ㅣ오不
  블〮以이〯禮례〮節졀〮之지면亦역〮不블〮可가〯行ᄒᆡᇰ也
  야〯ㅣ니라}
\諺{行ᄒᆡᇰ티〮몯〯ᄒᆞᆯ빼이시니〮和화ᄅᆞᆯ〮아라〮和화만〮ᄒᆞ
  고〮禮례〮로ᄡᅥ〮節졀티〮아니〮면〮ᄯᅩ〮ᄒᆞᆫ可가〯히〮行ᄒᆡᇰ
  티〮몯〯ᄒᆞ〮ᄂᆞ니라〮}
\原{○有유〯子ᄌᆞ〮ㅣ曰왈〮信신〯近근〯於어義의〯면言언
  可가〯復복〮也야〯ㅣ며恭고ᇰ近근〯於어禮례〮면遠원〯恥
  티〯辱ᅀᅭᆨ〮也야〯ㅣ며因인不블〮失실〮其기親친이면亦역〮
  可가〯宗조ᇰ也야〯ㅣ니라}
\諺{有유〯子ᄌᆞ〮ㅣᄀᆞᆯ오〮ᄃᆡ〮信신〯이義의〯예〮갓가오〮면〮
  言언을〮可가〯히〮復복〮ᄒᆞ며〮恭고ᇰ이〮禮례〮예〮갓가
  오〮면〮恥티〯와〮辱욕〮을〮遠원〯ᄒᆞ며〮因인홈〯애〮그親
  친ᄒᆞᆯ이를〮일티〮아니〮ᄒᆞ면〮ᄯᅩ〮ᄒᆞᆫ可가〯히〮宗조ᇰᄒᆞ〮
  얌〮즉〮ᄒᆞ니〮라〮}
\原{○子ᄌᆞ〮ㅣ曰왈〮君군子ᄌᆞ〮ㅣ食식〮無무求구飽포〯
  ᄒᆞ며居거無무求구安안ᄒᆞ며敏민〯於어事ᄉᆞ〯而ᅀᅡ愼
  신〯於어言언이오就ᄎᆔ〯有유〯道도〯而ᅀᅵ正져ᇰ〯焉언이면
  可가〯謂위〮好호〯學ᄒᆞᆨ〮也야〯巳이〯니라}
\諺{子ᄌᆞ〮ㅣᄀᆞᆯᄋᆞ샤〮ᄃᆡ〮君군子ᄌᆞ〮ㅣ食식〮홈〯애〮飽포〯
  홈〯을〮求구티〮아니〮ᄒᆞ며〮居거홈〯애〮安안홈〯을〮求
  구티〮아니〮ᄒᆞ며〮事ᄉᆞ〯애〮敏민〯ᄒᆞ며〮言언애〮愼신〯
  ᄒᆞ고〮道도〯인ᄂᆞᆫ〮ᄃᆡ〮나ᅀᅡ〮가〮正져ᇰ〯ᄒᆞ면〮可가〯히〮學
  ᄒᆞᆨ〮을〮됴〯히〮너긴〮다〮닐을이니라〮}
\原{○子ᄌᆞ〮貢고ᇰ〯이曰왈〮貧빈而ᅀᅵ無무諂텸〯ᄒᆞ며富부〯
  而ᅀᅵ無무驕교호ᄃᆡ何하如ᅀᅧᄒᆞ니잇고子ᄌᆞ〮ㅣ曰왈〮可
  가〯也야〯ㅣ나未미〯若ᅀᅣᆨ〮貧빈而ᅀᅵ樂락〮ᄒᆞ며富부〯而ᅀᅵ
  好호〯禮례〮者쟈〮也야〯ㅣ니라}
\諺{子ᄌᆞ〮貢고ᇰ〯이ᄀᆞᆯ오〮ᄃᆡ〮貧빈ᄒᆞ〮야〮도〮諂텸〯홈〯이〮업〯
  스며〮富부〯ᄒᆞ〮야〮도〮驕교홈〯이〮업〯소ᄃᆡ〮엇〯더ᄒᆞ니ᇰ〮
  잇〮고〮子ᄌᆞ〮ㅣᄀᆞᆯᄋᆞ샤〮ᄃᆡ〮可가〯ᄒᆞ나〮貧빈ᄒᆞ고〮樂
  락〮ᄒᆞ며〮富부〯ᄒᆞ고〮禮례〮를好호〯ᄒᆞᄂᆞᆫ〮者쟈〮만〮ᄀᆞᆮ〮
  디〮몯〯ᄒᆞ니라〮}
\原{子ᄌᆞ〮貢고ᇰ〯이曰왈〮詩시云운如ᅀᅧ切졀〮如ᅀᅧ磋차
  ᄒᆞ며如ᅀᅧ琢탁〮如ᅀᅧ磨마ㅣ리ᄒᆞ니其기斯ᄉᆞ之지謂위〮
  與여ᅟᅵᆫ뎌}
\諺{子ᄌᆞ〮貢고ᇰ〯이ᄀᆞᆯ오〮ᄃᆡ〮詩시예〮닐오〮ᄃᆡ〮切졀〮ᄐᆞᆺ〮ᄒᆞ
  고〮磋차ᄐᆞᆺ〮ᄒᆞ며〮琢탁〮ᄃᆞᆺ〮ᄒᆞ고〮磨마ᄐᆞᆺ〮ᄒᆞ다〮ᄒᆞ니〮
  그이〮ᄅᆞᆯ〮닐옴〮인뎌〮}
\原{子ᄌᆞ〮ㅣ曰왈〮賜ᄉᆞ〮也야〯ᄂᆞᆫ始시〯可가〯與여〯言언詩
  시已이〯矣의〯로다告고〮諸져往와ᇰ〯而ᅀᅵ知디來ᄅᆡ者
  쟈〮ㅣ온여}
\諺{子ᄌᆞ〮ㅣᄀᆞᆯᄋᆞ샤〮ᄃᆡ〮賜ᄉᆞ〮ᄂᆞᆫ비로〮소〮可가〯히〮더브〮
  러〮詩시를〮니ᄅᆞ리〮로다〮往와ᇰ〯을〮告고〮홈〯애〮來ᄅᆡ
  者쟈〮를〮알〯오녀〮}
\原{○子ᄌᆞ〮ㅣ曰왈〮不블〮患환〯人ᅀᅵᆫ之지不블〮己긔〮知
  디오患환〮不블〮知디人ᅀᅵᆫ也야〯ㅣ니라}
\諺{子ᄌᆞ〮ㅣᄀᆞᆯᄋᆞ샤〮ᄃᆡ〮人ᅀᅵᆫ의〮己긔〯ᄅᆞᆯ〮아〮디〮몯〯홈〯을〮
  患환〮티〮말〯고〮人ᅀᅵᆫ을〮아〯디〮몯〯홈〯을〮患환〮홀〯띠〮니
  라〮}

\篇{爲위政져ᇰ〮第뎨〯二ᅀᅵ〯}
\原{子ᄌᆞ〮ㅣ曰왈〮爲위政져ᇰ〮以이〯德덕〮이譬비〯如ᅀᅵ北
  븍〮辰신이居거其기所소〯ㅣ어든而ᅀᅵ衆쥬ᇰ〮星셔ᇰ이
  共고ᇰ〯之지니라}
\諺{子ᄌᆞ〮ㅣᄀᆞᆯᄋᆞ샤〮ᄃᆡ〮政져ᇰ〮을〮호〯ᄃᆡ〮德덕〮으〮로〮ᄡᅥ〮홈〯
  이譬비〯컨댄〮北븍〮辰신이그所소〯애〮居거ᄒᆞ〮얏〮
  거든〮모ᄃᆞᆫ〮별〯이共고ᇰ〯홈〯ᄀᆞ〮ᄐᆞ〮니라〮}
\原{○子ᄌᆞ〮ㅣ曰왈〮詩시三삼百ᄇᆡᆨ〮애一일〮言언以이〯
  蔽폐〮之지ᄒᆞ니曰왈〮思ᄉᆞ無무邪샤ㅣ니라}
\諺{子ᄌᆞ〮ㅣᄀᆞᆯᄋᆞ샤〮ᄃᆡ〮詩시ㅣ三삼百ᄇᆡᆨ〮애〮ᄒᆞᆫ말〯이〮
  ᄡᅥ〮蔽폐〮ᄒᆞ〮야〮시니〮ᄀᆞᆯ온〮思ᄉᆞㅣ邪샤업〯ᄉᆞᆷ〮이〮니
  라〮}
\原{○子ᄌᆞ〮ㅣ曰왈〮道도〯之지以이〯政져ᇰ〮ᄒᆞ고齊졔之지
  以이〯刑혀ᇰ이면民민免면〯而ᅀᅵ無무恥티〯니라}
\諺{子ᄌᆞ〮ㅣᄀᆞᆯᄋᆞ샤〮ᄃᆡ〮道도〯호〯ᄃᆡ〮政져ᇰ〮으〮로ᄡᅥ〮ᄒᆞ고〮
  齊졔호〯ᄃᆡ〮刑혀ᇰ으〮로ᄡᅥ〮ᄒᆞ면〮民민이免면〮ᄒᆞᆯ만
  ᄒᆞ고〮恥티〯홈〯은〮업〯ᄂᆞ니라〮}
\原{道도〯之지以이〯德덕〮ᄒᆞ고齊졔之지以이〯禮례〮면有
  유〯恥티〮且챠〯格격〮이니라}
\諺{道도〯호〯ᄃᆡ〮德덕으〮로ᄡᅥ〮ᄒᆞ고〮齊졔호〯ᄃᆡ〮禮례〮로
  ᄡᅥ〮ᄒᆞ면〮恥티〯홈〯이〮잇고〮ᄯᅩ〮格격〮ᄒᆞ〮ᄂᆞ니라〮}
\原{○子ᄌᆞ〮ㅣ曰왈〮吾오ㅣ十십〮有유〯五오〯而ᅀᅵ志지〮
  于우學ᄒᆞᆨ〮ᄒᆞ고}
\諺{子ᄌᆞ〮ㅣᄀᆞᆯᄋᆞ샤〮ᄃᆡ〮내〮열〮히〮오ᄯᅩ〮다ᄉᆞ〮새〮學ᄒᆞᆨ〮애〮
  志지〮ᄒᆞ고〮}
\原{三삼十십〮而ᅀᅵ立립〮ᄒᆞ고}
\諺{셜흔〮에〮立립〮ᄒᆞ고〮}
\原{四ᄉᆞ〯十십〮而ᅀᅵ不블〮惑혹〮ᄒᆞ고}
\諺{마ᄋᆞᆫ〮애〮惑혹〮디〮아니〮ᄒᆞ고〮}
\原{五오〯十십〮而ᅀᅵ知디天텬命며ᇰ〯ᄒᆞ고}
\諺{쉰에〮天텬命며ᇰ〯을〮알〯고〮}
\原{六륙〮十십〮而ᅀᅵ耳ᅀᅵ〯順슌〯ᄒᆞ고}
\諺{여슌〯에〮耳ᅀᅵ〯ㅣ順슌〯ᄒᆞ고〮}
\原{七칠〮十십〮而ᅀᅵ從죠ᇰ心심所소〯欲욕〮ᄒᆞ야不블〮踰유
  矩구〯호라}
\諺{닐흔〮에〮ᄆᆞᄋᆞᆷ의〮欲욕〮ᄒᆞ〮ᄂᆞᆫ바〮를〮조차〮矩구에〮넘〯
  디아니〮호〯라〮}
\原{○孟ᄆᆡᇰ〯懿의〯子ᄌᆞ〮ㅣ問문〯孝효〯ᄒᆞᆫ대子ᄌᆞ〮ㅣ曰왈〮無
  무違위니라}
\諺{孟ᄆᆡᇰ〯懿의〯子ᄌᆞ〮ㅣ孝효〯를〮묻〯ᄌᆞ온대〮子ᄌᆞ〮ㅣᄀᆞᆯ
  ᄋᆞ샤〮ᄃᆡ〮違위홈〯이〮업〯슴〮이니라〮}
\原{樊번遲디ㅣ御어〯ㅣ러니子ᄌᆞ〮ㅣ告고〯之지曰왈〮孟
  ᄆᆡᇰ〯孫손이問문〯孝효〯於어我아〯ㅣ어ᄂᆞᆯ我아〯ㅣ對ᄃᆡ〯
  曰왈〮無무違위라호라}
\諺{樊번遲디ㅣ御어〯ᄒᆞ〮야〮ᄯᅥ니子ᄌᆞ〮ㅣ告고〯ᄒᆞ〮야〮
  ᄀᆞᆯᄋᆞ샤〮ᄃᆡ〮孟ᄆᆡᇰ〯孫손이孝효〯를〮내게무러〮늘〮내〮
  對ᄃᆡ〯ᄒᆞ〮야〮ᄀᆞᆯ오〮ᄃᆡ〮違위〮홈〯이〮업〯슴〮이라〮호〯라〮}
\原{樊번遲디ㅣ曰왈〮何하謂위〮也야〯ㅣ잇고子ᄌᆞ〮ㅣ曰
  왈〮生ᄉᆡᇰ事ᄉᆞ〯之지以이〯禮례〮ᄒᆞ며死ᄉᆞ〯葬자ᇰ〯之지以
  이〯禮례〮ᄒᆞ며祭졔〯之지以이〯禮례〮니라}
\諺{樊번遲디ㅣᄀᆞᆯ오〮ᄃᆡ〮엇〯디〮닐옴〮이니ᇰ〮잇고〮子ᄌᆞ〮
  ㅣᄀᆞᆯᄋᆞ샤〮ᄃᆡ〮사라〮실쩨〮셤김을〮禮례〮로〮ᄡᅥ〮ᄒᆞ며〮
  죽음〮애〮葬자ᇰ〯홈〯을〮禮례〮로〮ᄡᅥ〮ᄒᆞ며〮祭졔〯홈〯을〮禮
  례〮로〮ᄡᅥ〮홈〯이니라〮}
\原{○孟ᄆᆡᇰ〯武무〯伯ᄇᆡᆨ〮이問문〯孝효〯ᄒᆞᆫ대子ᄌᆞ〮ㅣ曰왈〮父
  부〮母모〯ᄂᆞᆫ唯유其기疾질〮之지憂우ㅣ시니라}
\諺{孟ᄆᆡᇰ〯武무〯伯ᄇᆡᆨ〮이〮孝효〯를〮묻〯ᄌᆞ온대〮子ᄌᆞ〮ㅣᄀᆞᆯ
  ᄋᆞ샤〮ᄃᆡ〮父부〮母모〯ᄂᆞᆫ〮오직〮그疾질〮을〮근심ᄒᆞ〮시
  ᄂᆞ니〮라〮}
\原{○子ᄌᆞ〮游유ㅣ問문〯孝효〯ᄒᆞᆫ대子ᄌᆞ〮ㅣ曰왈〮今금之
  지孝효〯者쟈〮ᄂᆞᆫ是시〯謂위〮能느ᇰ養야ᇰ〯이니至지〮於어
  犬견〯馬마〯ᄒᆞ야도皆ᄀᆡ能느ᇰ有유〯養야ᇰ〯이니不블〮敬겨ᇰ〯
  이면何하以이〯別별〮乎호ㅣ리오}
\諺{子ᄌᆞ〮游유ㅣ孝효〯를〮묻〯ᄌᆞ온대子ᄌᆞ〮ㅣᄀᆞᆯᄋᆞ샤〮
  ᄃᆡ〮이〮젯孝효〯ᄂᆞᆫ〮이〮닐온〮能느ᇰ히〮養야ᇰ〯홈〯이니犬
  견〯과〮馬마〯애〮니르〮러도〮다〯能느ᇰ히〮養야ᇰ〯홈〯이〮인
  ᄂᆞ니〮敬겨ᇰ〯티〮아니〮ᄒᆞ면〮므스〮거스〮로ᄡᅥ〮別별〮ᄒᆞ
  리〮오〮}
\原{○子ᄌᆞ〮夏하〯ㅣ問문〯孝효〯ᄒᆞᆫ대子ᄌᆞ〮ㅣ曰왈〮色ᄉᆡᆨ〮難
  난이니有유〯事ᄉᆞ〯ㅣ어든弟뎨〯子ᄌᆞ〮ㅣ服복〮其기勞로
  ᄒᆞ고有유〯酒쥬〮食ᄉᆞ〯ㅣ어든先션生ᄉᆡᇰ饌찬〯이會즈ᇰ是
  시〯以이〯爲위孝효〯乎호아}
\諺{子ᄌᆞ〮夏하〯ㅣ孝효〯를〮묻〯ᄌᆞ온대〮子ᄌᆞ〮ㅣᄀᆞᆯᄋᆞ샤〮
  ᄃᆡ〮色ᄉᆡᆨ〮이어려〮오니일〯이잇거든〮弟뎨〯子ᄌᆞ〮ㅣ
  그勞로ᄅᆞᆯ〮服복〮ᄒᆞ고〮酒쥬〮와食ᄉᆞ〯ㅣ잇거든〮先
  션生ᄉᆡᇰ을〮饌찬〯홈〯이〮일즉〮이〮를〮ᄡᅥ〮孝효〯ㅣ라〮ᄒᆞ
  랴〮}
\原{○子ᄌᆞ〮ㅣ曰왈〮吾오與여〯回회로言언終죠ᇰ日ᅀᅵᆯ〮
  에不블〮違위如ᅀᅧ愚우ㅣ러니退퇴〯而ᅀᅵ省셔ᇰ〮其기
  私ᄉᆞ혼ᄃᆡ亦역〮足죡〮以이〯發발〮ᄒᆞᄂᆞ니回회也야〯ㅣ不
  블〮愚우ㅣ로다}
\諺{子ᄌᆞ〮ㅣᄀᆞᆯᄋᆞ샤〮ᄃᆡ〮내〮回회로〮더브〮러言언홈〯을〮
  日ᅀᅵᆯ〮을〮終죠ᇰ홈〯애〮어글〮웃디〮아니〮홈〯이〮어린〮ᄃᆞᆺ〮
  ᄒᆞ〮더니〮退퇴〯커든〮그私ᄉᆞ를〮省셔ᇰ〮혼〯ᄃᆡ〮ᄯᅩ〮ᄒᆞᆫ足
  죡〮히〮ᄡᅥ〮發발〮ᄒᆞ〮ᄂᆞ니回회ㅣ어리〮디아니〮ᄒᆞ〮도
  다〮}
\原{○子ᄌᆞ〮ㅣ曰왈〮視시〯其기所소〯以이〯ᄒᆞ며}
\諺{子ᄌᆞ〮ㅣᄀᆞᆯᄋᆞ샤〮ᄃᆡ〮그以이〯ᄒᆞ〮ᄂᆞᆫ바〮ᄅᆞᆯ〮視시〯[?|ᄒᆞ며]}
\原{觀관其기所소〯由유ᄒᆞ며}
\諺{그由유ᄒᆞᆫ바〮ᄅᆞᆯ〮觀관ᄒᆞ며〮}
\原{察찰〮其기所소〯安안이면}
\諺{그安안ᄒᆞ〮ᄂᆞᆫ바〮ᄅᆞᆯ〮察찰〮ᄒᆞ면〮}
\原{人ᅀᅵᆫ焉언廋수哉ᄌᆡ리오人ᅀᅵᆫ焉언廋수哉ᄌᆡ리오}
\諺{사〯ᄅᆞᆷ〮이〮엇〯디〮숨기〮리오〮사〯ᄅᆞᆷ〮이〮엇〯디〮숨기〮리오〮}
\原{○子ᄌᆞ〮ㅣ曰왈〮溫온故고〮而ᅀᅵ知디新신이면可가〯
  以이〯爲위師ᄉᆞ矣의〮니라}
\諺{子ᄌᆞ〮ㅣᄀᆞᆯᄋᆞ샤〮ᄃᆡ〮故고〮를〮溫온ᄒᆞ〮야〮新신을〮知
  디ᄒᆞ면〮可가〯히ᄡᅥ師ᄉᆞㅣ되염즉〮ᄒᆞ니〮라〮}
\原{○子ᄌᆞ〮ㅣ曰왈〮君군子ᄌᆞ〮ᄂᆞᆫ不블器긔〮니라}
\諺{子ᄌᆞ〮ㅣᄀᆞᆯᄋᆞ샤〮ᄃᆡ〮君군子ᄌᆞ〮ᄂᆞᆫ〮器긔〮ㅣ아니〮니
  라〮}
\原{○子ᄌᆞ〮貢고ᇰ〯이問문〯君군子ᄌᆞ〮ᄒᆞᆫ대子ᄌᆞ〮ㅣ曰왈〮先
  션行ᄒᆡᇰ其기言언이오而ᅀᅵ後후〯從죠ᇰ之지니라〮}
\諺{子ᄌᆞ〮貢고ᇰ〯이君군子ᄌᆞ〮를〮묻〯ᄌᆞ온대〮子ᄌᆞ〮ㅣᄀᆞᆯ
  ᄋᆞ샤〮ᄃᆡ〮몬져그言언을〮行ᄒᆡᇰᄒᆞ고〮後후〯에〮從죠ᇰ
  ᄒᆞ〮ᄂᆞ니라〮}
\原{○子ᄌᆞ〮ㅣ曰왈〮君군子ᄌᆞ〮ᄂᆞᆫ周쥬而ᅀᅵ不블〮比비〯
  ᄒᆞ고小쇼〯人ᅀᅵᆫᄋᆞᆫ比비〯而ᅀᅵ不블〮周쥬ㅣ니라}
\諺{子ᄌᆞ〮ㅣᄀᆞᆯᄋᆞ샤〮ᄃᆡ〮君군子ᄌᆞ〮ᄂᆞᆫ周쥬ᄒᆞ고〮比비〯
  티〮아니〮ᄒᆞ고〮小쇼〯人ᅀᅵᆫᄋᆞᆫ〮比비〯ᄒᆞ고周쥬티〮아
  니〮ᄒᆞ〮ᄂᆞ니라〮}
\原{○子ᄌᆞ〮ㅣ曰왈〮學ᄒᆞᆨ〮而ᅀᅵ不블〮思ᄉᆞ則즉〮罔마ᇰ〮ᄒᆞ고
  思ᄉᆞ而ᅀᅵ不블〮學ᄒᆞᆨ〮則즉〮殆ᄐᆡ〯니라}
\諺{子ᄌᆞ〮ㅣᄀᆞᆯᄋᆞ샤〮ᄃᆡ〮學ᄒᆞᆨ〮ᄒᆞ고〮思ᄉᆞ티〮아니〮ᄒᆞ면
  罔마ᇰ〮ᄒᆞ고〮思ᄉᆞᄒᆞ고〮學ᄒᆞᆨ〮디아니〮ᄒᆞ면〮殆ᄐᆡ〯ᄒᆞ〮
  ᄂᆞ니라〮}
\原{○子ᄌᆞ〮ㅣ曰왈〮攻고ᇰ乎호異이〯端단이면斯ᄉᆞ害해〯
  也야〯巳이〯니라}
\諺{子ᄌᆞ〮ㅣᄀᆞᆯᄋᆞ샤〮ᄃᆡ〮異이〯端단을〮攻고ᇰᄒᆞ면〮이〮害
  해〯니라〮}
\原{○子ᄌᆞ〮ㅣ曰왈〮由유아誨회〯女여〯知디之지乎호
  ᅟᅵᆫ뎌知디之지爲위知디之지오不블〮知디爲위不
  블〮知디ㅣ是시〯知디也야〯ㅣ니라}
\諺{子ᄌᆞ〮ㅣᄀᆞᆯᄋᆞ샤〮ᄃᆡ〮由유아〮너ᄅᆞᆯ〮알옴〯을〮ᄀᆞᄅᆞ칠〮
  띤뎌〮아〯ᄂᆞᆫ〮거슬〮아〯노〮라〮ᄒᆞ고〮아〯디몯〯ᄒᆞ〮ᄂᆞᆫ〮거슬〮
  아〯디몯〯ᄒᆞ〮노라〮홈〯이이〮알옴〯이니라〮}
\原{○子ᄌᆞ〮張댜ᇰ이學ᄒᆞᆨ〮干간祿록〮ᄒᆞᆫ대}
\諺{子ᄌᆞ〮張댜ᇰ이祿록〮을〮干간홈〯을〮學ᄒᆞᆨ〮호〯려〮ᄒᆞᆫ〮대〮}
\原{子ᄌᆞ〮ㅣ曰왈〮多다聞문闕궐〮疑의오愼신〯言언其
  기餘여則즉〮寡과〯尤우ㅣ며多다見견〯闕궐〮殆ᄐᆡ▩
  愼신〯行ᄒᆡᇰ其기餘여則즉〮寡과〯悔회〯니言언寡과〯
  尤우ᄒᆞ며行ᄒᆡᇰ〯寡과〯悔회〯면祿록〮在ᄌᆡ〯其기中듀ᇰ矣
  의〯니라}
\諺{子ᄌᆞ〮ㅣᄀᆞᆯᄋᆞ샤〮ᄃᆡ〮해〯드러〮疑의를〮闕궐〮ᄒᆞ고〮그
  남으〮니를〮삼가〮니르면〮허믈〮이젹〯으며해〯보〮와
  殆ᄐᆡ〯를〮闕궐〮ᄒᆞ고〮그남으〮니를〮삼가〮行ᄒᆡᇰᄒᆞ면〮
  뉘〯웃브미〮젹〯ᄂᆞ니〮言언이〮허믈〮이젹〯으며〮行ᄒᆡᇰ〯
  이뉘〯웃〮브미〮젹〯으면〮祿록〮이그가온대〮인ᄂᆞ니〮
  라〮}
\原{○哀ᄋᆡ公고ᇰ이問문〯曰왈〮何하爲위則즉〮民민服
  복〮이니ᇰ잇고孔고ᇰ〮子ᄌᆞ〮ㅣ對ᄃᆡ〯曰왈〮擧거〯直딕〮錯조〯諸
  져枉와ᇰ〯則즉〮民민服복〮ᄒᆞ고擧거〯枉와ᇰ〯錯조〯諸져直
  딕〮則즉〮民민不블〮服복〮이니ᇰ이다}
\諺{哀ᄋᆡ公고ᇰ이묻〯ᄌᆞ와〮ᄀᆞᆯ오〮ᄃᆡ〮엇〯디ᄒᆞ면〮民민이〮
  服복〮ᄒᆞ〮ᄂᆞ니ᇰ잇고〮孔고ᇰ〮子ᄌᆞ〮ㅣ對ᄃᆡ〯ᄒᆞ〮야ᄀᆞᆯᄋᆞ
  샤ᄃᆡ直딕〮을〮擧거〯ᄒᆞ고〮모ᄃᆞᆫ〮枉와ᇰ〯을〮錯조〯ᄒᆞ면〮
  民민이〮服복〮ᄒᆞ고〮枉와ᇰ〯을〮擧거〯ᄒᆞ고〮모ᄃᆞᆫ〮直딕〮
  을〮錯조〯ᄒᆞ면〮民민이〮服복〮디〮아니〮ᄒᆞ〮ᄂᆞ니ᇰ이〮다}
\原{○季계〯康가ᇰ子ᄌᆞ〮ㅣ問문〯使ᄉᆞ〯民민敬겨ᇰ〯忠튜ᇰ以
  이〯勸권〯호ᄃᆡ如ᅀᅧ之지何하ㅣ니ᇰ잇고子ᄌᆞ〮ㅣ曰왈〮臨림
  之지以이〯莊자ᇰ則즉〮敬겨ᇰ〯ᄒᆞ고孝효〯慈ᄌᆞ則즉〮忠튜ᇰ
  ᄒᆞ고擧거〯善션〯而ᅀᅵ敎교〯不블〮能느ᇰ則즉〮勸권〯이니라}
\諺{季계〯康가ᇰ子ᄌᆞ〮ㅣ묻〯ᄌᆞ오ᄃᆡ〮民민으로〮ᄒᆞ여〮곰
  敬겨ᇰ〯ᄒᆞ며〮忠튜ᇰᄒᆞ며〮ᄡᅥ〮勸권〯케〮호〯ᄃᆡ〮엇〯디ᄒᆞ리〮
  잇〮고子ᄌᆞ〮ㅣᄀᆞᆯᄋᆞ샤〮ᄃᆡ〮臨림호〯ᄃᆡ莊자ᇰ으〮로ᄡᅥ〮
  ᄒᆞ면〮敬겨ᇰ〯ᄒᆞ고〮孝효〯ᄒᆞ며〮慈ᄌᆞᄒᆞ면忠튜ᇰᄒᆞ고〮
  善션〯을擧거〯ᄒᆞ고〮能느ᇰ티〮몯〯ᄒᆞ〮ᄂᆞᆫ이를〮ᄀᆞᄅᆞ치〮
  면〮勸권〯ᄒᆞ〮ᄂᆞ니라〮}
\原{○或혹〮이謂위〮孔고ᇰ〮子ᄌᆞ〮曰왈〮子ᄌᆞ〮ᄂᆞᆫ奚ᄒᆡ不블〮
  爲위政져ᇰ〮이시니ᇰ잇고}
\諺{或혹〮이孔고ᇰ〮子ᄌᆞ〮ㅅᄭᅴ닐어ᄀᆞᆯ오〮ᄃᆡ子ᄌᆞ〮ᄂᆞᆫ엇〯
  디政져ᇰ〮을〮ᄒᆞ디〮아니〮ᄒᆞ〮시ᄂᆞ니ᇰ잇〮고}
\原{子ᄌᆞ〮ㅣ曰왈〮書셔云운孝효〯乎호ᅟᅵᆫ뎌惟유孝효〯ᄒᆞ며
  友우〯于우兄혀ᇰ弟뎨〯ᄒᆞ야施시於어有유〯政져ᇰ〮이라ᄒᆞ니
  是시〯亦역〮爲위政져ᇰ〮이니奚ᄒᆡ其기爲위爲위政져ᇰ〮
  이리오}
\諺{子ᄌᆞ〮ㅣᄀᆞᆯᄋᆞ샤〮ᄃᆡ書셔애〮孝효〯를〮닐런〮ᄂᆞᆫ뎌〮孝
  효〯ᄒᆞ며〮兄혀ᇰ弟뎨〯예〮友우〯ᄒᆞ〮야政져ᇰ〮에〮베〯프〮다
  ᄒᆞ니〮이〮ᄯᅩ〮ᄒᆞᆫ政져ᇰ〮을〮홈〯이니엇〯디ᄒᆞ야ᅀᅡ그政
  져ᇰ〮을〮ᄒᆞ다〮ᄒᆞ리〮오}
\原{○子ᄌᆞ〮ㅣ曰왈〮人ᅀᅵᆫ而ᅀᅵ無무信신〯이면不블〮知디
  其기可가〮也야〯케라大대〯車거ㅣ無무輗예ᄒᆞ며小쇼〯
  車거ㅣ無무軏월〮이면其기何하以이〯行ᄒᆡᇰ之지哉
  ᄌᆡ리오}
\諺{子ᄌᆞ〮ㅣᄀᆞᆯᄋᆞ샤〮ᄃᆡ사〯ᄅᆞᆷ〮이오〮信신〯이업〯스면〮그
  可가〯홈〯을〮아〯디몯〯게라大대〯ᄒᆞᆫ車거ㅣ輗예ㅣ
  업〯스〮며小쇼〯ᄒᆞᆫ車거ㅣ軏월〮이업〯스면〮그므서
  스〮로ᄡᅥ〮行ᄒᆡᇰᄒᆞ리〮오}
\原{○子ᄌᆞ〮張댜ᇰ이問문〯十십〮世셰〯를可가〯知디也야〯
  ㅣ잇가}
\諺{子ᄌᆞ〮張댜ᇰ이묻〯ᄌᆞ오ᄃᆡ〮十십〮世셰〯를〮可가〯히알〯
  꺼시〮니ᇰ잇〮가}
\原{子ᄌᆞ〮ㅣ曰왈〮殷은因인於어夏하〯禮례〮ᄒᆞ니所소〯損
  손〯益익〮을可가〯知디也야〯ㅣ며周쥬因인於어殷은
  禮례〮ᄒᆞ니所소〯損손〯益익〮을可가〯知디也야〯ㅣ니其기
  或혹〮繼계〯周쥬者쟈〮ㅣ면雖슈百ᄇᆡᆨ〮世셰〯라도可가〯知
  디也야〯ㅣ니라}
\諺{子ᄌᆞ〮ㅣᄀᆞᆯᄋᆞ샤〮ᄃᆡ殷은이〮夏하〯ㅅ禮례〮예因인
  ᄒᆞ니〮損손〯ᄒᆞ며〮益익〮ᄒᆞᆫ바〮ᄅᆞᆯ〮可가〯히〮알〯꺼시며〮
  周쥬ㅣ殷은ㅅ禮례〮예因인ᄒᆞ니〮損손〯ᄒᆞ며〮益
  익〮ᄒᆞᆫ바〮ᄅᆞᆯ〮可가〯히알〯꺼시니그或혹〮周쥬를〮니
  을〮者쟈〮ㅣ면〮비록〮百ᄇᆡᆨ〮世셰〯라도〮可가〯히알〯꺼
  시〮니라〮}
\原{○子ᄌᆞ〮ㅣ曰왈〮非비其기鬼귀〯而ᅀᅵ祭졔〯之지ㅣ
  諂텸〯也야〯ㅣ오}
\諺{子ᄌᆞ〮ㅣᄀᆞᆯᄋᆞ샤〮ᄃᆡ그鬼귀〯ㅣ아닌〮거슬〮祭졔〯홈〯
  이〮諂텸〯이오}
\原{見견〯義의〯不블〮爲위ㅣ無무勇요ᇰ〯也야〯ㅣ니라}
\諺{義의〯를〮보고〮ᄒᆞ디〮아니〮홈〯이勇요ᇰ〯이업〯슴〮이니
  라〮}

\篇{八팔〮佾일〮第뎨〯三삼}
\原{孔고ᇰ〮子ᄌᆞ〮ㅣ謂위〮季계〯氏시〮ᄒᆞ샤ᄃᆡ八팔〮佾일〮로舞
  무〯於어庭뎌ᇰᄒᆞ니是시〯可가〯忍ᅀᅵᆫ〯也야〯ㅣ온孰슉〮不블〮
  可가〯忍ᅀᅵᆫ〯也야〯ㅣ리오}
\諺{孔고ᇰ〮子ᄌᆞ〮ㅣ季계〯氏시〮를〮니ᄅᆞ샤〮ᄃᆡ八팔〮佾일〮
  로庭뎌ᇰ에〮舞무〯ᄒᆞ니〮이〮ᄅᆞᆯ〮可가〯히ᄎᆞᆷ〮아ᄒᆞ곤〮므
  스〮거ᄉᆞᆯ〮可가〯히ᄎᆞᆷ〮아몯〯ᄒᆞ리〮오}
\原{○三삼家가者쟈〮ㅣ以이〯雍오ᇰ徹텰〮이러니子ᄌᆞ〮ㅣ
  曰왈〮相샤ᇰ〮維유辟벽〮公고ᇰ이어늘天텬子ᄌᆞ〮穆목〮穆
  목〮을奚ᄒᆡ取ᄎᆔ〯於어三삼家가之지堂다ᇰ고}
\諺{三삼家가者쟈〮ㅣ雍오ᇰ으〮로ᄡᅥ〮徹텰〮ᄒᆞ〮더니子
  ᄌᆞ〮ㅣᄀᆞᆯᄋᆞ샤〮ᄃᆡ〮相샤ᇰ〮ᄒᆞ〮ᄂᆞᆫ이辟벽〮公고ᇰ이〮어ᄂᆞᆯ〮
  天텬子ᄌᆞ〮ㅣ穆목〮穆목〮ᄒᆞ〮욤을〮엇〯디三삼家가
  ㅅ堂다ᇰ에〮取ᄎᆔ〯ᄒᆞᆫ고〮}
\原{○子ᄌᆞ〮ㅣ曰왈〮人ᅀᅵᆫ而ᅀᅵ不블〮仁ᅀᅵᆫ이면如ᅀᅧ禮례〮
  예何하ㅣ며人ᅀᅵᆫ而ᅀᅵ不블〮仁ᅀᅵᆫ이면如ᅀᅧ樂악〮애何
  하오}
\諺{子ᄌᆞ〮ㅣᄀᆞᆯᄋᆞ샤〮ᄃᆡ〮사〯ᄅᆞᆷ이〮오仁ᅀᅵᆫ티〮아니〮ᄒᆞ면〮
  禮례〮예엇〯디ᄒᆞ며〮사〯ᄅᆞᆷ이〮오仁ᅀᅵᆫ티〮아니〮ᄒᆞ면〮
  樂악〮애엇〯디ᄒᆞ료〮}
\原{○林림放바ᇰ〯이問문〯禮례〮之지本본〮ᄒᆞᆫ대}
\諺{林림放바ᇰ〯이禮례〮의本본〮을〮묻〯ᄌᆞ온대〮}
\原{子ᄌᆞ〮ㅣ曰왈〮大대〯哉ᄌᆡ라問문〯이여}
\諺{子ᄌᆞ〮ㅣᄀᆞᆯᄋᆞ샤〮ᄃᆡ〮크〮다무롬〮이〮여}
\原{禮례〮ㅣ與여〯其기奢샤也야〯론寧녀ᇰ儉검〯이오喪사ᇰ
  이與여〯其기易이〯也야〯론寧녀ᇰ戚쳑〮이니라}
\諺{禮례〮ㅣ그奢샤홈〯으〮로더브〮러론〮ᄎᆞᆯ하리〮儉검〯
  홀〯띠〮오喪사ᇰ이그易이〯홈〯으〮로더브〮러론〮ᄎᆞᆯ하
  리〮戚쳑〮홀〯띠〮니라}
\原{○子ᄌᆞ〮ㅣ曰왈〮夷이狄뎍〮之지有유〯君군이不[?|블〮]
  如ᅀᅧ諸져夏하〯之지亡무也야〯ㅣ니라}
\諺{子ᄌᆞ〮ㅣᄀᆞᆯᄋᆞ샤〮ᄃᆡ夷이狄뎍〮의君군이〮이심〮이〮
  諸져夏하〯의〮업〯ᄉᆞ니ᄀᆞᆮ〮디〮아니〮ᄒᆞ니라}
\原{○季계〯氏시〮ㅣ旅려〯於어泰태〮山산이러니子ᄌᆞ〮ㅣ
  謂위〮冉ᅀᅧᆷ〯有유〯曰왈〮女예〯ㅣ弗블〮能느ᇰ救구〯與여
  아對ᄃᆡ〯曰왈〮不블〮能느ᇰ이로소이다子ᄌᆞ〮ㅣ曰왈〮嗚오
  呼호ㅣ라會즈ᇰ謂위〮泰태〮山산이不블〮如ᅀᅧ林림放
  바ᇰ〯乎호아}
\諺{季계〯氏시〮ㅣ泰태〮山산애旅려〯ᄒᆞ〮더〮니子ᄌᆞ〮ㅣ
  冉ᅀᅧᆷ〯有유〯ᄃᆞ려〮닐어〮ᄀᆞᆯᄋᆞ샤〮ᄃᆡ〮네〮能느ᇰ히〮救구〮
  티〮몯〯ᄒᆞ리로소냐〮對ᄃᆡ〯ᄒᆞ〮야〮ᄀᆞᆯ오〮ᄃᆡ〮能느ᇰ티〮몯〯
  ᄒᆞ리〮로소ᇰ이다〮子ᄌᆞ〮ㅣᄀᆞᆯᄋᆞ샤〮ᄃᆡ〮嗚오呼호ㅣ
  라일즉〮泰태〮山산이林림放바ᇰ〯만곤〮디〮몯〯ᄒᆞ다
  니ᄅᆞ랴〮}
\原{○子ᄌᆞ〮ㅣ曰왈〮君군子ᄌᆞ〮ㅣ無무所소〯爭ᄌᆡᇰ이나必
  필〮也야〯射샤〯乎호ᅟᅵᆫ뎌揖읍〮讓ᅀᅣᇰ〯而ᅀᅵ升스ᇰᄒᆞ야下하〯
  而ᅀᅵ飮음〯ᄒᆞᄂᆞ니其기爭ᄌᆡᇰ也야〯ㅣ君군子ᄌᆞ〮ㅣ니라}
\諺{子ᄌᆞ〮ㅣᄀᆞᆯᄋᆞ샤〮ᄃᆡ〮君군子ᄌᆞ〮ㅣᄃᆞ토〮ᄂᆞᆫ배〮업〯스
  나반〮ᄃᆞ시〮射샤〯ᅟᅵᆫ뎌〮揖읍〯讓ᅀᅣᇰ〯ᄒᆞ〮야〮을라〮ᄂᆞ려〮
  와〮머키〮ᄂᆞ니〮그ᄃᆞ토〮미〮君군子ᄌᆞ〮ㅣ니라}
\原{○子ᄌᆞ〮夏하〯ㅣ問문〯曰왈〮巧교〯笑쇼〯倩쳔〯兮혜며
  美미〯目목〮盼변〯兮혜여素소〯以이〯爲위絢현〯兮혜
  라ᄒᆞ니何하謂위〮也야〯ㅣ잇고}
\諺{子ᄌᆞ〮夏하〯ㅣ묻〯ᄌᆞ와〮ᄀᆞᆯ오〮ᄃᆡ巧교〯ᄒᆞᆫ笑쇼〯ㅣ倩
  쳔〯ᄒᆞ며〮美미〯ᄒᆞᆫ目목〮이盼변〯홈〯이여素소〯로ᄡᅥ〮
  絢현〯을〮ᄒᆞ다〮ᄒᆞ니〮엇〯디닐옴〮이〮니ᇰ잇〮고}
\原{子ᄌᆞ〮ㅣ曰왈〮繪회〯事ᄉᆞ〯ㅣ後후〯素소〯ㅣ니라}
\諺{子ᄌᆞ〮ㅣᄀᆞᆯᄋᆞ샤〮ᄃᆡ〮繪회〯ᄒᆞ〮ᄂᆞᆫ일〮이素소〯애〮後후〯
  ㅣ니라〮}
\原{曰왈〮禮례〮ㅣ後후〯乎호ᅟᅵᆫ뎌子ᄌᆞ〮ㅣ曰왈〮起긔〮予여
  者쟈〮ᄂᆞᆫ商샤ᇰ也야〯ㅣ로다始시〯可가〯與여〯言언詩시
  巳이〯矣의〯로다}
\諺{ᄀᆞᆯ오ᄃᆡ〮禮례〮ㅣ後후〯ᅟᅵᆫ뎌〮子ᄌᆞ〮ㅣᄀᆞᆯᄋᆞ샤〮ᄃᆡ〮나〯
  ᄅᆞᆯ〮起긔〮ᄒᆞᄂᆞᆫ者쟈〮ᄂᆞᆫ商샤ᇰ이〮로다〮비르〮소〮可가〯
  히〮더브〮러詩시를〮닐엄〮즉〮ᄒᆞ〮도다〮}
\原{○子ᄌᆞ〮ㅣ曰왈〮夏하〯禮례〮를吾오能느ᇰ言언之지
  나杞긔〮不블〮足죡〮徵디ᇰ也야〯ㅣ며殷은禮례〮를吾오
  能느ᇰ言언之지나宋소ᇰ〯不블〮足죡〮徵디ᇰ也야〯ᄂᆞᆫ文
  문獻헌〯이不블〮足죡〮故고〮也야〯ㅣ니足죡〮則즉〮吾오
  能느ᇰ徵디ᇰ之지矣의〯로리라}
\諺{子ᄌᆞ〮ㅣᄀᆞᆯᄋᆞ샤〮ᄃᆡ〮夏하〯ㅅ禮례〮를〮내〮能느ᇰ히〮니
  르나〮杞긔〮예足죡〮히徵디ᇰ티〮몯〯ᄒᆞ며〮殷은ㅅ禮
  례〮를〮내〮能느ᇰ히〮니르나〮宋소ᇰ〯에足죡〮히徵디ᇰ티〮
  몯〯홈〯은文문과〮獻헌〯이足죡〮디〮몯〯ᄒᆞᆫ故고〮ㅣ니
  足죡〮ᄒᆞ면〮내〮能느ᇰ히〮徵디ᇰ호〯리라〮}
\原{○子ᄌᆞ〮ㅣ曰왈〮禘톄〮ㅣ自ᄌᆞ〮旣긔〮灌관〮而ᅀᅵ往와ᇰ〯
  者쟈〮ᄂᆞᆫ吾오不블〮欲욕〮觀관之지矣의〯로라}
\諺{子ᄌᆞ〮ㅣᄀᆞᆯᄋᆞ샤〮ᄃᆡ〮禘톄〮ㅣ임의〮灌관〮홈〯으〮로브
  터〮往와ᇰ〯ᄒᆞᆫ者쟈〮ᄂᆞᆫ내〮보고〮져아니〮ᄒᆞ〮노라〮}
\原{○或혹〮이問문〯禘톄〮之지說셜〮ᄒᆞᆫ대子ᄌᆞ〮ㅣ曰왈〮不
  블〮知디也야〯ㅣ로라知디其기說셜〮者쟈〮之지於어
  天텬下하〯也야〯애其기如ᅀᅧ示시〯諸져斯ᄉᆞ乎호
  딘혀ᄉᆞᄀᆚ指지〮其기掌쟈ᇰ〯ᄒᆞ시다}
\諺{或혹〮이禘톄〮의說셜〮을〮묻〯ᄌᆞ온대〮子ᄌᆞ〮ㅣᄀᆞᆯᄋᆞ
  샤〮ᄃᆡ〮아〯디〮몯〯ᄒᆞ〮노라그說셜〮을〮아〯ᄂᆞᆫ者쟈〮ㅣ天
  텬下하〯애〮그이〮ᄅᆞᆯ〮봄〯ᄀᆞ〮ᄐᆞᆫ뎌〮ᄒᆞ〮시고〮그掌쟈ᇰ〯을〮
  ᄀᆞᄅᆞ치〮시다}
\原{○祭졔〯如ᅀᅧ在ᄌᆡ〯ᄒᆞ시며祭졔〯神신如ᅀᅧ神신在ᄌᆡ〮
  러시다}
\諺{祭졔〯ᄒᆞ〮샤〮ᄃᆡ〮인ᄂᆞᆫᄃᆞ〮시〮ᄒᆞ〮시며〮神신을〮祭졔〯ᄒᆞ〮
  샤〮ᄃᆡ〮神신이〮인ᄂᆞᆫᄃᆞ〮시ᄒᆞ〮더시다}
\原{子ᄌᆞ〮ㅣ曰왈〮吾오不블〮與여〯祭졔〯면如ᅀᅧ不블祭
  졔〯니라}
\諺{子ᄌᆞ〮ㅣᄀᆞᆯᄋᆞ샤〮ᄃᆡ〮내〮祭졔〯예〮與여〯티몯〯ᄒᆞ면〮祭
  졔〯아니〮홈〯ᄀᆞ〮ᄐᆞ니라〮}
\原{○王와ᇰ孫손賈가〯ㅣ問문〯曰왈〮與여〯其기媚미〯於
  어奧오〯론寧녀ᇰ媚미〯於어䆴조〯ㅣ라ᄒᆞ니何하謂위〮也
  야〯ㅣ잇고}
\諺{王와ᇰ孫손賈가〯ㅣ묻〯ᄌᆞ와〮ᄀᆞᆯ오〮ᄃᆡ그奧오〯애媚
  미〯홈〯으로〮더브〮러론〮ᄎᆞᆯ하리〮䆴조〯애媚미〯홀〮띠〮
  라ᄒᆞ니〮엇〯디닐오〮미니ᇰ잇고〮}
\原{子ᄌᆞ〮ㅣ曰왈〮不블〮然ᅀᅧᆫᄒᆞ다獲획〮罪죄〯於어天텬이면
  無무所소〯禱도〯也야〯ㅣ니라}
\諺{子ᄌᆞ〮ㅣᄀᆞᆯᄋᆞ샤〮ᄃᆡ〮그러티〮아니〮ᄒᆞ다〮罪죄〯를〮하
  ᄂᆞᆯ〮ᄭᅴ〮어〯드〮면〮禱도〯ᄒᆞᆯ〯빼〮업〯스니라〮}
\原{○子ᄌᆞ〮ㅣ曰왈〮周쥬監감於어二ᅀᅵ〯代ᄃᆡ〯ᄒᆞ니郁욱〮
  郁욱〮乎호文문哉ᄌᆡ라吾오從죠ᇰ周쥬호리라}
\諺{子ᄌᆞ〮ㅣᄀᆞᆯᄋᆞ샤〮ᄃᆡ〮周쥬ㅣ二ᅀᅵ〯代ᄃᆡ〯예監감ᄒᆞ
  니〮郁욱〮郁욱〮히〮文문ᄒᆞᆫ디〮라내〮周쥬를〮조초〮리
  라〮}
\原{○子ᄌᆞ〮ㅣ入ᅀᅵᆸ〮大태〮廟묘〯ᄒᆞ샤每ᄆᆡ〯事ᄉᆞ〯를問문〯ᄒᆞ신
  대或혹〮이曰왈〮孰슉〮謂위〮鄹추人ᅀᅵᆫ之지子ᄌᆞ〮를
  知디禮례〮乎호오入ᅀᅵᆸ〮大태〮廟묘〯ᄒᆞ야每ᄆᆡ〯事ᄉᆞ〯를
  問문〯이온여子ᄌᆞ〮ㅣ聞문之지ᄒᆞ시고曰왈〮是시〯ㅣ禮
  례〮也야〯ㅣ니라}
\諺{子ᄌᆞ〮ㅣ大태〮廟묘〯애드〮르샤每ᄆᆡ〯事ᄉᆞ〯를〮무르
  신대〮或혹〮이ᄀᆞᆯ오〮ᄃᆡ〮뉘〮닐오〮ᄃᆡ〮鄹추人ᅀᅵᆫ의〮子
  ᄌᆞ〮를〮禮례〮를〮안〯다〮ᄒᆞ〮더뇨〮大태〮廟묘〯애드〮러〮每
  ᄆᆡ〯事ᄉᆞ〯를〮묻〯고녀〮子ᄌᆞ〮ㅣ드ᄅᆞ시고ᄀᆞᆯᄋᆞ샤〮ᄃᆡ〮
  이〮禮례〮ㅣ니라}
\原{○子ᄌᆞ〮ㅣ曰왈〮射샤〯不블〮主쥬〮皮피ᄂᆞᆫ爲위〯力력〮
  不블〮同도ᇰ科과ㅣ니古고〯之지道도〯也야〯ㅣ니라}
\諺{子ᄌᆞ〮ㅣᄀᆞᆯᄋᆞ샤〮ᄃᆡ〮射샤〯홈〯애〮皮피를〮主쥬〮티아
  니〮홈〯은〮힘〮이科과ㅣ同도ᇰ티〮아님〮을〮爲위〯ᄒᆞ〮얘
  니〮녯〯道도〯ㅣ니라〮}
\原{○子ᄌᆞ〮貢고ᇰ〯이欲욕〮去거〯告곡〮朔삭〮之지餼희〯羊
  야ᇰᄒᆞᆫ대}
\諺{子ᄌᆞ〮貢고ᇰ〯이朔삭〮을〮告곡〮ᄒᆞ〮ᄂᆞᆫ餼희〯羊야ᇰ을〮去
  거〮코〮져ᄒᆞᆫ〯대}
\原{子ᄌᆞ〮ㅣ曰왈〮賜ᄉᆞ〯也야〯아爾ᅀᅵ〯愛ᄋᆡ〯其기羊야ᇰ가
  我아〯愛ᄋᆡ〯其기禮례〮ᄒᆞ노라}
\諺{子ᄌᆞ〮ㅣᄀᆞᆯᄋᆞ샤〮ᄃᆡ〮賜ᄉᆞ〯아너ᄂᆞᆫ〮그羊야ᇰ을〮愛ᄋᆡ〯
  ᄒᆞ〮ᄂᆞᆫ다〮나〮ᄂᆞᆫ그禮례〮를〮愛ᄋᆡ〯ᄒᆞ〮노라〮}
\原{○子ᄌᆞ〮ㅣ曰왈〮事ᄉᆞ〯君군盡진〯禮례〮를人ᅀᅵᆫ이以
  이〯爲위諂텸〯也야〯ㅣ라ᄒᆞᄂᆞ다}
\諺{子ᄌᆞ〮ㅣᄀᆞᆯᄋᆞ샤〮ᄃᆡ〮君군을셤굠〮애禮례〮를다〯홈〯
  을사〯ᄅᆞᆷ이〮ᄡᅥ〮諂텸〯ᄒᆞᆫ〮다ᄒᆞ〮ᄂᆞ다〮}
\原{○定뎌ᇰ〯公고ᇰ이問문〯君군使ᄉᆞ〯臣신ᄒᆞ며臣신事ᄉᆞ〯
  君군호ᄃᆡ如ᅀᅧ之지何하ㅣ잇고孔고ᇰ〮子ᄌᆞ〮ㅣ對ᄃᆡ〯曰
  왈〮君군使ᄉᆞ〯臣신以이〯禮례〮ᄒᆞ며臣신事ᄉᆞ〯君군以
  이〯忠튜ᇰ이니ᇰ이다}
\諺{定뎌ᇰ〯公고ᇰ이묻〯ᄌᆞ오ᄃᆡ〮君군이〮臣신을〮브〮리며
  臣신이〮君군을〮셤교〮ᄃᆡ〮엇〯디〮ᄒᆞ리ᇰ〮잇고〮孔고ᇰ〮子
  ᄌᆞ〮ㅣ對ᄃᆡ〯ᄒᆞ〮야〮ᄀᆞᆯᄋᆞ샤〮ᄃᆡ〮君군이〮臣신을〮브〮료〮
  ᄃᆡ〮禮례〮로ᄡᅥ〮ᄒᆞ며〮臣신이君군을〮셤교〮ᄃᆡ〮忠튜ᇰ
  으로ᄡᅥ〮홀〯띠〮니ᇰ이다〮}
\原{○子ᄌᆞ〮ㅣ曰왈〮關관睢져ᄂᆞᆫ樂락〮而ᅀᅵ不블〮淫음
  ᄒᆞ고哀ᄋᆡ而ᅀᅵ不블〮傷샤ᇰ이니라}
\諺{子ᄌᆞ〮ㅣᄀᆞᆯᄋᆞ샤〮ᄃᆡ〮關관睢져ᄂᆞᆫ〮樂락〮호〯ᄃᆡ〮淫음
  티〮아니〮ᄒᆞ고〮哀ᄋᆡ호〯ᄃᆡ〮傷샤ᇰ티〮아니〮ᄒᆞ니〮라〮}
\原{○哀ᄋᆡ公고ᇰ이問문〯社샤〯於어宰ᄌᆡ〯我아〯ᄒᆞ신대宰
  ᄌᆡ〯我아〯ㅣ對ᄃᆡ〯曰왈〮夏하〯后후〯氏시〮ᄂᆞᆫ以이〯松쇼ᇰ
  이오殷은人ᅀᅵᆫᄋᆞᆫ以이〯柏ᄇᆡᆨ〮이오周쥬人ᅀᅵᆫᄋᆞᆫ以이栗
  률〮이니曰왈〮使ᄉᆞ〯民민戰젼〯栗률〮이니ᇰ이다}
\諺{哀ᄋᆡ公고ᇰ이社샤〯를宰ᄌᆡ〯我아〯의게무ᄅᆞ〮신대〮
  宰ᄌᆡ〯我아〯ㅣ對ᄃᆡ〯ᄒᆞ〮야〮ᄀᆞᆯ오〮ᄃᆡ〮夏하〯后후〯氏시〮
  ᄂᆞᆫ松쇼ᇰ으〮로ᄡᅥ〮ᄒᆞ고〮殷은人ᅀᅵᆫᄋᆞᆫ〮柏ᄇᆡᆨ〮으〮로ᄡᅥ〮
  ᄒᆞ고〮周쥬人ᅀᅵᆫᄋᆞᆫ〮栗률〮로ᄡᅥ〮ᄒᆞ니〮ᄀᆞᆯ온〮民민으〮
  로ᄒᆞ여〮곰〮戰젼〯栗률〮케홈〮이〮니ᇰ이다〮}
\原{子ᄌᆞ〮ㅣ聞문之지ᄒᆞ시고曰왈〮成셔ᇰ事ᄉᆞ〯ㅣ라不블〮說
  셜〮ᄒᆞ며遂슈〮事ᄉᆞ〯ㅣ라不블〮諫간〯ᄒᆞ며旣긔〮往와ᇰ〯이라不블〮
  咎구〯ㅣ로다}
\諺{子ᄌᆞ〮ㅣ드ᄅᆞ시〮고〮ᄀᆞᆯᄋᆞ샤〮ᄃᆡ〮成셔ᇰᄒᆞᆫ일〯이라〮說
  셜〮티〮몯〯ᄒᆞ며〮遂슈〮ᄒᆞᆫ일〯이라諫간〯티〮몯〯ᄒᆞ며〮임
  의〮디〯난〮디〮라咎구〯티몯〯ᄒᆞ리〮로다}
\原{○子ᄌᆞ〮ㅣ曰왈〮管관〮仲듀ᇰ〯之지器긔〮ㅣ小쇼〯哉ᄌᆡ
  라}
\諺{子ᄌᆞ〮ㅣᄀᆞᆯᄋᆞ샤〮ᄃᆡ〮管관〮仲듀ᇰ〯의〮그르〮시〮小쇼〯ᄒᆞ
  다〮}
\原{或혹〮이曰왈〮管관〮仲듀ᇰ〯은儉검〯乎호ㅣ잇가曰왈〮管
  관氏시〮ㅣ有유〯三삼歸귀ᄒᆞ며官관事ᄉᆞ〯를不블〮攝
  셥〮ᄒᆞ니焉언得득〮儉검〯이리오}
\諺{或혹〮이ᄀᆞᆯ오〮ᄃᆡ〮管관〮仲듀ᇰ〯은〮儉검〯ᄒᆞ니ᇰ〮잇가〮ᄀᆞᆯ
  ᄋᆞ샤〮ᄃᆡ〮管관〮氏시〮ㅣ三삼歸귀를〮두며〮官관事
  ᄉᆞ〯를〮攝셥〮디〮아니〮ᄒᆞ니〮엇〯디시러〮곰〮儉검〯ᄒᆞ리〮
  오}
\原{然ᅀᅧᆫ則즉〮管관〮仲듀ᇰ〯은知디禮례〮乎호ㅣ잇가曰왈〮
  邦바ᇰ君군이ᅀᅡ樹슈〮塞ᄉᆡᆨ〮門문이어ᄂᆞᆯ管관〮氏시〮ㅣ亦
  역〮樹슈〮塞ᄉᆡᆨ〮門문ᄒᆞ며邦바ᇰ君군이ᅀᅡ爲위兩랴ᇰ〯君군
  之지好호〯애有유〯反반〯坫뎜〯이어ᄂᆞᆯ管관〮氏시〮ㅣ亦
  역〮有유〯反반〯坫덤〯ᄒᆞ니管관〮氏시〮而ᅀᅵ知디禮례〮면
  孰슉〮不블知디禮례〮리오}
\諺{그러〮면〮管관〮仲듀ᇰ〯은〮禮례〮를〮아〯니ᇰ〮잇가〮ᄀᆞᆯᄋᆞ샤〮
  ᄃᆡ邦바ᇰ君군이〮ᅀᅡ樹슈〮로〮門문을〮塞ᄉᆡᆨ〮ᄒᆞ거ᄂᆞᆯ〮
  管관〮氏시〮ᄯᅩ〮ᄒᆞᆫ樹슈〮로〮門문을〮塞ᄉᆡᆨ〮ᄒᆞ며〮邦바ᇰ
  君군이〮ᅀᅡ兩냐ᇰ〯君군의〮好호〯를〮홈〯애〮反반〯ᄒᆞ〮ᄂᆞᆫ〮
  坫덤〯을〮두〮거ᄂᆞᆯ〮管관〮氏시〮ᄯᅩ〮ᄒᆞᆫ反반〯ᄒᆞ〮ᄂᆞᆫ〮坫[?|덤〯]
  을〮두니〮管관〮氏시〯오禮례〮를〮알〯면〮뉘〮禮례〮를〮아〮
  디〮몯〯ᄒᆞ리〮오}
\原{○子ᄌᆞ〮ㅣ語어〯魯로〮大태〮師ᄉᆞ樂악〮曰왈〮樂악〮은
  其기可가〯知디也야〯ㅣ니始시〯作작〮애翕흡〮如ᅀᅧ也
  야〯ᄒᆞ야從죠ᇰ〯之지예純슌如ᅀᅧ也야〯ᄒᆞ며皦교〯如ᅀᅧ也
  야〯ᄒᆞ며繹역〮如ᅀᅧ也야〯ᄒᆞ야以이〯成셔ᇰ이니라}
\諺{子ᄌᆞ〮ㅣ魯로〮大태〮師ᄉᆞᄃᆞ려〮樂악〮을〮닐어〮ᄀᆞᆯᄋᆞ
  샤〮ᄃᆡ〮樂악〮은〮그可가〯히〮알〯띠〮니비르〮소作작〮홈〯
  애〮翕흡〮ᄃᆞᆺ〮ᄒᆞ〮야〮從죠ᇰ〯홈〯애〮純슌ᄐᆞᆺ〮ᄒᆞ며〮皦교〯ᄐᆞᆺ〮
  ᄒᆞ며〮繹역〮ᄃᆞᆺ〮ᄒᆞ〮야〮ᄡᅥ〮成셔ᇰᄒᆞ〮ᄂᆞ〮니라〮}
\原{○儀의封보ᇰ人ᅀᅵᆫ이請쳐ᇰ〮見현〯曰왈〮君군子ᄌᆞ〮之
  지至지〮於어斯ᄉᆞ也야〮애吾오未미〯嘗샤ᇰ不블〮得
  득〮見견〯也야〯ㅣ로라從죠ᇰ〯者쟈〮ㅣ見현〯之지ᄒᆞᆫ대出츌〮
  曰왈〮二ᅀᅵ〯三삼子ᄌᆞ〮ᄂᆞᆫ何하患환〯於어喪사ᇰ〯乎호
  ㅣ리오天텬下하〯之지無무道도〯也야〯ㅣ久구〯矣의〯
  리天텬將쟈ᇰ以이〯夫무子ᄌᆞ〮로爲위木목〮鐸탁〮이시
  리라}
\諺{儀의ㅅ封보ᇰ人ᅀᅵᆫ이뵈〯ᄋᆞ〮옴을〮請쳐ᇰ〮ᄒᆞ〮야〮ᄀᆞᆯ오〮
  ᄃᆡ〮君군子ᄌᆞ〮ㅣ이〮예니르〮롬애일즉〮시러〮곰〮見
  견〯티〮몯〯ᄒᆞ디〮아니〮ᄒᆞ〮얀〮노라〮從죠ᇰ〯者쟈〮ㅣ見현〯
  ᄒᆞ이〮온대〮나〮와〮ᄀᆞᆯ오〮ᄃᆡ〮二ᅀᅵ〯三삼子ᄌᆞ〮ᄂᆞᆫ〮엇〯디〮
  喪사ᇰ〯홈〯애〮患환〯ᄒᆞ리〮오〮天텬下하〯의〮道도〯ㅣ업〯
  슴이〮오란〮디〮라하ᄂᆞᆯ〮히〮쟈ᇰᄎᆞᆺ〮夫부子ᄌᆞ〮로〮ᄡᅥ〮木
  목〮鐸탁〮을〮삼으〮시리라〮}
\原{○子ᄌᆞ〮ㅣ謂위〮韶쇼ᄒᆞ샤ᄃᆡ盡진〯美미〯矣의〯오又우〯
  盡진〯善션〯也야〯ㅣ라ᄒᆞ시고謂위〮武무〯ᄒᆞ샤ᄃᆡ盡진〯美미〯
  矣의〯오未미〯盡진〯善션〯也야〯ㅣ라ᄒᆞ시다}
\諺{子ᄌᆞ〮ㅣ韶쇼를〮니ᄅᆞ샤〮ᄃᆡ〮극〮진히〮美미〯ᄒᆞ고ᄯᅩ
  극〮진〯히〮善션〯타〮ᄒᆞ〮시고〮武무〯를〮니ᄅᆞ샤〮ᄃᆡ〮극〮진〯
  히〮美미〯ᄒᆞ고〮극〮진〯히〮善션〯티〮몯〯ᄒᆞ다〮ᄒᆞ〮시다〮}
\原{○子ᄌᆞ〮ㅣ曰왈〮居거上샤ᇰ〯不블〮寬관ᄒᆞ며爲위禮례〮
  不블〮敬겨ᇰ〯ᄒᆞ며臨림喪사ᇰ不블〮哀ᄋᆡ면吾오何하以
  이〯觀관之지哉ᄌᆡ리오}
\諺{子ᄌᆞ〮ㅣᄀᆞᆯᄋᆞ샤〮ᄃᆡ〮上샤ᇰ〯애〮居거ᄒᆞ〮야〮寬관티〮아
  니〮ᄒᆞ며〮禮례〮를〮호〯ᄃᆡ〮敬겨ᇰ〯티〮아니〮ᄒᆞ며〮喪사ᇰ애〮
  臨림ᄒᆞ〮야〮哀ᄋᆡ티〮아니〮ᄒᆞ면〮내〮므스〮거스〮로〮ᄡᅥ〮
  보리〮오}

\篇{里리〯仁ᅀᅵᆫ第뎨〯四ᄉᆞ〯}
\原{子ᄌᆞ〮ㅣ曰왈〮里리〯仁ᅀᅵᆫ이爲위美미〯ᄒᆞ니擇ᄐᆡᆨ〮不블〮
  處쳐〯仁ᅀᅵᆫ이면焉언得득〮知디〮리오}
\諺{子ᄌᆞ〮ㅣᄀᆞᆯᄋᆞ샤〮ᄃᆡ〮ᄆᆞᄋᆞᆯ히〮仁ᅀᅵᆫ홈〯이아〮ᄅᆞᆷ〮다[?|오〮]
  니〮ᄀᆞᆯᄒᆡ〮오ᄃᆡ〮仁ᅀᅵᆫ에〮處쳐〯티〮아니〮ᄒᆞ면〮엇〯디〮[?|시]
  러〮곰〮知디〮타〮ᄒᆞ리〮오}
\原{○子ᄌᆞ〮ㅣ曰왈〮不블〮仁ᅀᅵᆫ者쟈〮ᄂᆞᆫ不블〮可가〯以이〯
  久구〯處쳐〯約약〮이며不블〮可가〯以이〯長댜ᇰ〮處쳐〯樂락〮
  이니仁ᅀᅵᆫ者쟈〮ᄂᆞᆫ安안仁ᅀᅵᆫᄒᆞ고知디〮者쟈〮ᄂᆞᆫ利리〯仁
  ᅀᅵᆫ이니라}
\諺{子ᄌᆞ〮ㅣᄀᆞᆯᄋᆞ샤〮ᄃᆡ〮仁ᅀᅵᆫ티〮아니〮ᄒᆞᆫ者쟈〮ᄂᆞᆫ〮可가〮
  히〮ᄡᅥ〮오래〮約약〮에〮處쳐〯티〮몯〯ᄒᆞ며〮可가히〮ᄡᅥ〮기
  리〮樂락〯에〮處쳐〮티〮몯〯ᄒᆞ〮ᄂᆞ니〮仁ᅀᅵᆫᄒᆞᆫ者쟈〮ᄂᆞᆫ〮仁
  ᅀᅵᆫ을〮安안ᄒᆞ고〮知디〮ᄒᆞᆫ者쟈〮ᄂᆞᆫ仁ᅀᅵᆫ에〮利리〯히〮
  너기〮ᄂᆞ니라〮}
\原{○子ᄌᆞ〮ㅣ曰왈〮惟유仁ᅀᅵᆫ者쟈〮ㅣᅀᅡ能느ᇰ好호〯人ᅀᅵᆫ
  ᄒᆞ며能느ᇰ惡오〯人ᅀᅵᆫ이니라}
\諺{子ᄌᆞ〮ㅣᄀᆞᆯᄋᆞ샤〮ᄃᆡ〮오직〮仁ᅀᅵᆫᄒᆞᆫ者쟈〮ㅣᅀᅡ能느ᇰ
  히〮사〯ᄅᆞᆷ〮을〮好호〯ᄒᆞ며〮能느ᇰ히〮사〯ᄅᆞᆷ〮을〮惡오〯ᄒᆞ〮ᄂᆞ
  니라〮}
\原{○子ᄌᆞ〮ㅣ曰왈〮苟구〯志지〮於어仁ᅀᅵᆫ矣의〯면無무
  惡악〮也야〯ㅣ니라}
\諺{子ᄌᆞ〮ㅣᄀᆞᆯᄋᆞ샤〮ᄃᆡ〮진실〯로仁ᅀᅵᆫ에〮志지〮ᄒᆞ면〮惡
  악〮이업〯ᄂᆞ니라〮}
\原{○子ᄌᆞ〮ㅣ曰왈〮富부〯與여〯貴귀〯ㅣ是시〯人ᅀᅵᆫ之지
  所소〯欲욕〮也야〯ㅣ나不블〮以이〯其기道도〯로得득〮之
  지어든不블〮處쳐〯也야〯ᄒᆞ며貧빈與어〯賤쳔〯이是시〯人
  ᅀᅵᆫ之지所소〯惡오〯也야〯ㅣ나不블〮以이〯其기道도〯로
  得득〮之지라도不블〮去거〮也야〯ㅣ니라}
\諺{子ᄌᆞ〮ㅣᄀᆞᆯᄋᆞ샤〮ᄃᆡ〮富부〯홈〯과〮다ᄆᆞᆺ〮貴귀〯홈〯이이〮
  사〯ᄅᆞᆷ〮의〮ᄒᆞ고〮져〮ᄒᆞ〮ᄂᆞᆫ배〮나〮그道도〯로ᄡᅥ〮아니〮ᄒᆞ
  야〮어〯더든〮處쳐〯티〮아니〮ᄒᆞ며〮貧빈홈〯과〮다ᄆᆞᆺ〮賤
  쳔〯홈〯이이〮사〯ᄅᆞᆷ〮의〮惡오〯ᄒᆞ〮ᄂᆞᆫ배〮나〮그道도〯로ᄡᅥ〮
  아니〮ᄒᆞ〮야〮어〯더도〮去거〮티〮아니홀띠〮니라}
\原{君군子ᄌᆞ〮ㅣ去거〮仁ᅀᅵᆫ이면惡오乎호成셔ᇰ名며ᇰ이리
  오}
\諺{君군子ᄌᆞ〮ㅣ仁ᅀᅵᆫ을〮去거〮ᄒᆞ면〮어듸〮일홈〯을〮일
  오〮리오〮}
\原{君군子ᄌᆞ〮ㅣ無무終죠ᇰ食식〮之지間간을違위仁
  ᅀᅵᆫ이니造조〮次ᄎᆞ〮애必필〮於어是시〯ᄒᆞ며顚뎐沛패예
  必필〮於어是시〯니라}
\諺{君군子ᄌᆞ〮ㅣ食식〮終죠ᇰᄒᆞᆯᄉᆞ이〮ᄅᆞᆯ〮仁ᅀᅵᆫ에〮違[?|위]
  홈〯이〮업〯ᄂᆞ니造조〮次ᄎᆞ〮애〮반〮ᄃᆞ시이〮예〮ᄒᆞ며顚
  뎐沛패〮예〮반〮ᄃᆞ시이〮예〮ᄒᆞ〮ᄂᆞ〮니라〮}
\原{○子ᄌᆞ〮ㅣ曰왈〮我아〯未미〯見견〯好호〯仁ᅀᅵᆫ者쟈〮와
  惡오〯不블〮仁ᅀᅵᆫ者쟈〮케라好호〯仁ᅀᅵᆫ者쟈〮ᄂᆞᆫ無무以
  이〯尙샤ᇰ〮之지오惡오〯不블〮仁ᅀᅵᆫ者쟈〮ᄂᆞᆫ其기爲위
  仁ᅀᅵᆫ矣의〯ㅣ不블〮使ᄉᆞ〯不블〮仁ᅀᅵᆫ者쟈〮로加가乎
  호其기身신이니라}
\諺{子ᄌᆞ〮ㅣᄀᆞᆯᄋᆞ샤〮ᄃᆡ〮내〮仁ᅀᅵᆫ을〮好호〯ᄒᆞ〮ᄂᆞᆫ〮者쟈〮와〮
  不블〮仁ᅀᅵᆫ을〮惡오〯ᄒᆞ〮ᄂᆞᆫ〮者쟈〮를보디〮몯〯게〮라仁
  ᅀᅵᆫ을〮好호〯ᄒᆞ〮ᄂᆞᆫ〮者쟈〮ᄂᆞᆫᄡᅥ〮더을꺼시〮업〯고〮不블〮
  仁ᅀᅵᆫ을〮惡오〮ᄒᆞ〮ᄂᆞᆫ〮者쟈〮ᄂᆞᆫ그仁ᅀᅵᆫ을〮ᄒᆞ〮욤이〮不
  블〮仁ᅀᅵᆫ으〮로ᄒᆞ여〯곰〮그몸애〮加가티〮아니〮ᄒᆞ〮ᄂᆞ〮
  니라〮}
\原{有유〯能느ᇰ一일〮日ᅀᅵᆯ〮에用요ᇰ〯其기力력〮於어仁ᅀᅵᆫ
  矣의〯乎호아我아〯未미〯見견〯力력〮不블〮足죡〮者쟈〮
  케라}
\諺{能느ᇰ히〮一일〮日ᅀᅵᆯ〮에〮그힘〮을〮仁ᅀᅵᆫ에〮ᄡᅳ〮리인ᄂᆞ
  냐〮내〮힘〮이足죡〮디몯〯ᄒᆞᆫ者쟈〮ᄅᆞᆯ〮보디〮몯〯게라〮}
\原{蓋개〯有유〯之지矣의〯어ᄂᆞᆯ我아〯未미〯之지見견〯也야〯
  ㅣ로다}
\諺{잇거ᄂᆞᆯ〮내〮보디〮돋〯ᄒᆞ〮엿〮도다〮}
\原{○子ᄌᆞ〮ㅣ曰왈〮人ᅀᅵᆫ之지過과〮也야〯ㅣ各각〮於어
  其기黨다ᇰ〮이니觀관過과에斯ᄉᆞ知디仁ᅀᅵᆫ矣의〯니라}
\諺{子ᄌᆞ〮ㅣᄀᆞᆯᄋᆞ샤〮ᄃᆡ〮사〯ᄅᆞᆷ〮의〮허믈〮이각〮각그류〯에〮
  니〮허믈〮을〮봄〯애〮이〮에〮仁ᅀᅵᆫ을〮알〯띠〮니라〮}
\原{○子ᄌᆞ〮ㅣ曰왈〮朝됴聞문道도〯ㅣ면夕셕〮死ᄉᆞ〯ㅣ라도
  可가〯矣의〯니라}
\諺{子ᄌᆞ〮ㅣᄀᆞᆯᄋᆞ샤〮ᄃᆡ〮아ᄎᆞᆷ의〮道도〯를〮드르면〮나죄
  죽어〮도〮可가〯ᄒᆞ니〮라}
\原{○子ᄌᆞ〮ㅣ曰왈〮士ᄉᆞ〯ㅣ志지〮於어道도〯而ᅀᅵ恥티〯
  惡악〮衣의惡악〮食식〮者쟈〮ᄂᆞᆫ未미〯足죡〮與여〯議의〯
  也야〯ㅣ니라}
\諺{子ᄌᆞ〮ㅣᄀᆞᆯᄋᆞ샤〮ᄃᆡ〮士ᄉᆞ〯ㅣ道도〯애志지〮호〯ᄃᆡ〮사
  오〮나온〮옷〮과사오〮나온〮음〯식〮을〮붓그〮리ᄂᆞᆫ者
  ᄂᆞᆫ足죡〮히더브〮러議의〯티몯〯ᄒᆞᆯ꺼시〮니라〮}
\原{○子ᄌᆞㅣ曰왈〮君군子ᄌᆞ〮之지於어天텬下하〯也
  야〯애無무適뎍〮也야〯ᄒᆞ며無무莫막〮也야〯ᄒᆞ야義의〯之
  지與여〯比비〯니라}
\諺{子ᄌᆞ〮ㅣᄀᆞᆯᄋᆞ샤〮ᄃᆡ〮君군子ᄌᆞ〮ㅣ天텬下하〯애適
  뎍〮홈〯도〮업〯스〮며莫막〮홈〯도〮업〯서義의〯로더브〮러
  比비〯ᄒᆞ〮ᄂᆞ니라}
\原{○子ᄌᆞ〮ㅣ曰왈〮君군子ᄌᆞᄂᆞᆫ懷회德덕〮ᄒᆞ고小쇼〯人
  ᅀᅵᆫ은懷회土토〮ᄒᆞ며君군子ᄌᆞ〮ᄂᆞᆫ懷회刑혀ᇰᄒᆞ고小쇼〯
  人ᅀᅵᆫ은懷회惠혜〯니라}
\諺{子ᄌᆞ〮ㅣᄀᆞᆯᄋᆞ샤〮ᄃᆡ〮君군子ᄌᆞ〮ᄂᆞᆫ〮德덕〮을〮懷회ᄒᆞ
  고〮小쇼〯人ᅀᅵᆫ은〮土토〮를〮懷회ᄒᆞ며〮君군子ᄌᆞ〮ᄂᆞᆫ〮
  刑혀ᇰ을〮懷회ᄒᆞ고〮小쇼〯人ᅀᅵᆫ은〮惠혜〯ᄅᆞᆯ懷회ᄒᆞ〮
  ᄂᆞ〮니라〮}
\原{○子ᄌᆞ〮ㅣ曰왈〮放바ᇰ〯於어利리〯而ᅀᅵ行ᄒᆡᇰ이면多다
  怨원〯이니라}
\諺{子ᄌᆞ〮ㅣᄀᆞᆯᄋᆞ샤〮ᄃᆡ〮利리〯예〮放바ᇰ〯ᄒᆞ〮야〮行ᄒᆡᇰᄒᆞ면〮
  怨원〯이하〮ᄂᆞ〮니라〮}
\原{○子ᄌᆞ〮ㅣ曰왈〮能느ᇰ以이〯禮례〮讓ᅀᅣᇰ〯이면爲위國국〮
  乎호애何하有유〯ㅣ며不블〮能느ᇰ以이〯禮례〮讓ᅀᅣᇰ〯으로
  爲위國국〮이면如ᅀᅧ禮례〮예何하ㅣ리오}
\諺{子ᄌᆞ〮ㅣᄀᆞᆯᄋᆞ샤〮ᄃᆡ〮能느ᇰ히〮禮례〮讓ᅀᅣᇰ〯으〮로ᄡᅥ〮ᄒᆞ
  면〮國국〮을〮홈〯애〮므서〮시이시며〮能느ᇰ히〮禮례〮讓
  ᅀᅣᇰ〯으〮로ᄡᅥ〮國국〮을〮ᄒᆞ디〮몯〯ᄒᆞ면〮禮례〮예〮엇〯디〮ᄒᆞ
  리〮오}
\原{○子ᄌᆞ〮ㅣ曰왈〮不블〮患환〯無무位위〮오患환〯所소〯
  以이〯立립〮ᄒᆞ며不블〮患환〯莫막〮己긔〮知디오求구爲
  위可가〯知디也야〯ㅣ니라}
\諺{子ᄌᆞ〮ㅣᄀᆞᆯᄋᆞ샤〮ᄃᆡ〮位위〮업〯ᄉᆞ믈〮患환〯티〮말〯오〮ᄡᅥ〮
  立립〮ᄒᆞᆯ빠〮ᄅᆞᆯ〮患환〯ᄒᆞ며〮己긔〮아〯디〮몯〯호〯ᄆᆞᆯ〮患환〯
  티〮말〯오〮可가〯히〮알〯게〮ᄒᆞ〮욤을〮求구홀〯띠〮니〮라}
\原{○子ᄌᆞ〮ㅣ曰왈〮參ᄉᆞᆷ乎호아吾오道도〯ᄂᆞᆫ一일〮以
  이〯貫관〮之지니라會즈ᇰ子ᄌᆞ〮ㅣ曰왈〮唯유〮ㅣ라}
\諺{子ᄌᆞ〮ㅣᄀᆞᆯᄋᆞ샤〮ᄃᆡ〮參ᄉᆞᆷ아〮吾오道도〯ᄂᆞᆫ〮一일〮이〮
  ᄡᅥ〮貫관〮ᄒᆞ〮얀〮ᄂᆞ니〮라會즈ᇰ子ᄌᆞ〮ㅣᄀᆞᆯᄋᆞ샤〮ᄃᆡ〮唯
  유〮ㅣ라}
\原{子ᄌᆞ〮ㅣ出츌〮커시ᄂᆞᆯ門문人ᅀᅵᆫ이問문〯曰왈〮何하謂
  위〮也야〯ㅣ잇고曾즈ᇰ子ᄌᆞ〮ㅣ曰왈〮夫부子ᄌᆞ〮之지道
  도〯ᄂᆞᆫ忠튜ᇰ恕셔〯而ᅀᅵ已이〯矣의〯니라}
\諺{子ᄌᆞ〮ㅣ出츌〮커시ᄂᆞᆯ〮門문人ᅀᅵᆫ이〮묻〯ᄌᆞ〮와ᄀᆞᆯ오〮
  ᄃᆡ〮엇〯디〮니ᄅᆞ심이니ᇰ〮잇고〮曾즈ᇰ子ᄌᆞ〮ㅣᄀᆞᆯᄋᆞ샤〮
  ᄃᆡ〮夫부子ᄌᆞ〮의道도〯ᄂᆞᆫ〮忠튜ᇰ과〮恕셔〯ᄯᆞᄅᆞᆷ이〮니
  라〮}
\原{○子ᄌᆞ〮ㅣ曰왈〮君군子ᄌᆞ〮ᄂᆞᆫ喩유〯於어義의〯ᄒᆞ고小
  쇼〯人ᅀᅵᆫ은喩유〯於어利리〯니라}
\諺{子ᄌᆞ〮ㅣᄀᆞᆯᄋᆞ샤〮ᄃᆡ〮君군子ᄌᆞ〮ᄂᆞᆫ〮義의〯예〮喩유〯ᄒᆞ
  고〮小쇼〯人ᅀᅵᆫ은〮利리〯예〮喩유〯ᄒᆞ〮ᄂᆞ니라〮}
\原{○子ᄌᆞ〮ㅣ曰왈〮見견〯賢현思ᄉᆞ齊졔焉언ᄒᆞ며見견〯
  不블〮賢현而ᅀᅵ內ᄂᆡ〯自ᄌᆞ〮省셔ᇰ〮也야〯ㅣ니라}
\諺{子ᄌᆞ〮ㅣᄀᆞᆯᄋᆞ샤〮ᄃᆡ〮賢현ᄒᆞᆫ이〮ᄅᆞᆯ〮보고〮齊졔홈〯을〮
  思ᄉᆞᄒᆞ며〮賢현티〮아〮니〮ᄒᆞᆫ이〮ᄅᆞᆯ〮보고〮안〮ᄒᆞ로〮스
  스〮로〮省셔ᇰ〮홀〯띠〮니라〮}
\原{○子ᄌᆞ〮ㅣ曰왈〮事ᄉᆞ〯父부〮母모〯호ᄃᆡ幾긔諫간〯이니見
  견〯志지〮不블〮從죠ᇰᄒᆞ고又우〯敬겨ᇰ〯不블〮違위ᄒᆞ며勞로
  而ᅀᅵ不블〮怨원〯이니라}
\諺{子ᄌᆞ〮ㅣᄀᆞᆯᄋᆞ샤〮ᄃᆡ〮父부〮母모〮를〮셤교〮ᄃᆡ〮幾긔히〮
  諫간〯홀〯띠〮니〮志지〮ㅣ좃디〮아니〮ᄒᆞ〮심을〮보고〮ᄯᅩ〮
  敬겨ᇰ〯ᄒᆞ〮야〮違위티〮아니〮ᄒᆞ〮며〮勞로ᄒᆞ〮야도〮怨원〯
  티〮아니〮홀〯띠〮니라〮}
\原{○子ᄌᆞ〮ㅣ曰왈〮父부〮母모〯ㅣ在ᄌᆡ〯어시는不블〮遠원〯
  遊유ᄒᆞ며遊유必필〮有유〯方바ᇰ이니라}
\諺{子ᄌᆞ〮ㅣᄀᆞᆯᄋᆞ샤〮ᄃᆡ〮父부〯母모〯ㅣ겨〯시〮거시든〮멀
  리〮遊유티〮아니〮ᄒᆞ며〮遊유호〯ᄃᆡ〮반〮ᄃᆞ시〮方바ᇰ을〮
  둘띠〮니라〮}
\原{○子ᄌᆞ〮ㅣ曰왈〮三삼年년을無무改ᄀᆡ〯於어父부〮
  之지道도〯ㅣ라ᅀᅡ可가〯謂위〮孝효〯矣의〯니라
  ○子ᄌᆞ〮ㅣ曰왈〮父부〮母모〯之지年년은不블〮可가〯
  不블〮知디也야〯ㅣ니一일〮則즉〮以이〯喜희〯오一일〮則
  즉〮以이〯懼구〯ㅣ니라}
\諺{子ᄌᆞ〮ㅣᄀᆞᆯᄋᆞ샤〮ᄃᆡ〮父부〮母모〯의〮나〮ᄒᆞᆫ可가〯히〮知
  디티〮아니〮티〮몯〮ᄒᆞᆯ꺼시〮니〮一일〮로ᄂᆞᆫᄡᅥ〮깃브고〮
  一일〮로ᄂᆞᆫᄡᅥ〮[?|저]프니〮라}
\原{○子ᄌᆞ〮ㅣ曰왈〮古고〯者쟈〮애言언之지不블〮出츌〮
  은恥티〯躬구ᇰ之지不블〮逮톄〮也야〯ㅣ니라}
\諺{子ᄌᆞ〮ㅣᄀᆞᆯᄋᆞ샤〮ᄃᆡ〮古고〯者쟈〮애〮말〯ᄉᆞᆷ을〮내〯디〮아
  니〮홈〯은〮몸의〮밋디〮몯〯홈〯을〮붓그〮림〮이〮니라〮}
\原{○子ᄌᆞ〮ㅣ曰왈〮以이〯約약〮失실〮之지者쟈〮ㅣ鮮션〯
  矣의〯니라}
\諺{子ᄌᆞ〮ㅣᄀᆞᆯᄋᆞ샤〮ᄃᆡ〮約약〮으〮로ᄡᅥ〮失실〮ᄒᆞᆯ者쟈〮ㅣ
  져〯그〮니라}
\原{○子ᄌᆞ〮ㅣ曰왈〮君군子ᄌᆞ〮ᄂᆞᆫ欲욕〮訥눌〮於어言언
  而ᅀᅵ敏민〮於어行ᄒᆡᇰ〯이니라}
\諺{子ᄌᆞ〮ㅣᄀᆞᆯᄋᆞ샤〮ᄃᆡ〮君군子ᄌᆞ〮ᄂᆞᆫ〮言언에〮訥눌〮ᄒᆞ
  고〮行ᄒᆡᇰ〯애〮敏민〮코〮져ᄒᆞ〮ᄂᆞ〮니라〮}
\原{○子ᄌᆞ〮ㅣ曰왈〮德덕〮不블〮孤고ㅣ라必필〮有유〯隣린
  이니리}
\諺{子ᄌᆞ〮ㅣᄀᆞᆯᄋᆞ샤〮ᄃᆡ〮德덕〮이孤고티〮아니〮ᄒᆞᆫ디〮라
  반〮ᄃᆞ시隣린이〮인ᄂᆞ니〮라}
\原{○子ᄌᆞ〮游유ㅣ曰왈〮事ᄉᆞ〯君군數삭〮이[?|면]斯ᄉᆞ辱ᅀᅭᆨ〮
  矣의〯오朋브ᇰ友우〯數삭〮이면斯ᄉᆞ疏소矣의〯니라}
\諺{子ᄌᆞ〮游유ㅣᄀᆞᆯ오〮ᄃᆡ〮君군을〮셤김〮애〮數삭〮ᄒᆞ면〮
  이〮에〮辱ᅀᅭᆨ〮ᄒᆞ고〮朋브ᇰ友우〯에〮數삭〮ᄒᆞ면〮이〮에〮疏
  소ᄒᆞ〮ᄂᆞ니라}

\篇{公고ᇰ冶야〯長댜ᇰ第뎨〯五오〯}
\原{子ᄌᆞ〮ㅣ謂위〮公고ᇰ冶야〯長댜ᇰᄒᆞ샤ᄃᆡ可가〯妻쳐〯也야〯
  ㅣ로다雖슈在ᄌᆡ〯縲류〯絏셜〮之지中듀ᇰ이나非비其기
  罪죄〯也야〯ㅣ라ᄒᆞ시고以이〯其기子ᄌᆞ〮로妻쳐〯之지ᄒᆞ시
  다}
\諺{子ᄌᆞ〮ㅣ公고ᇰ冶야〯長댜ᇰ을〮닐ᄋᆞ샤〮ᄃᆡ〮可가〯히〮妻
  쳐〯ᄒᆞ〮얌즉〮ᄒᆞ〮도다〮비록〮縲류〮絏셜〮ㅅ中듀ᇰ에〮이
  시나〮그罪죄〯ㅣ아니〮라ᄒᆞ〮시고〮그子ᄌᆞ〮로ᄡᅥ〮妻
  쳐〯ᄒᆞ〮시다〮}
\原{子ᄌᆞ〮ㅣ謂위〮南남容요ᇰᄒᆞ샤ᄃᆡ邦바ᇰ有유〯道도〯애不
  블〮廢폐〯ᄒᆞ며邦바ᇰ無무道도〯애免면〯於어刑혀ᇰ戮륙〮
  이라ᄒᆞ시고以이〯其기兄혀ᇰ之지子ᄌᆞ〮로妻쳐〯之지ᄒᆞ시
  다}
\諺{子ᄌᆞ〮ㅣ南남容요ᇰ을〮닐ᄋᆞ샤〮ᄃᆡ〮나라〮히〮道도〯ㅣ
  이숌〯애〮廢폐〯티〮아니〮ᄒᆞ며〮나라〮히〮道도〯ㅣ업〯슴〮
  애刑혀ᇰ戮륙〮에〮免면〯ᄒᆞ리〮라〮ᄒᆞ〮시고〮그兄혀ᇰ의〮
  子ᄌᆞ〮로ᄡᅥ〮妻쳐〯ᄒᆞ〮시다〮}
\原{○子ᄌᆞ〮ㅣ謂위〮子ᄌᆞ〮賤쳔〯ᄒᆞ샤ᄃᆡ君군子ᄌᆞ〮哉ᄌᆡ라
  若약〮人ᅀᅵᆫ이여魯로〮無무君군子ᄌᆞ〮者쟈〮ㅣ면斯ᄉᆞ焉
  언取ᄎᆔ〯斯ᄉᆞㅣ리오}
\諺{子ᄌᆞ〮ㅣ子ᄌᆞ〮賤쳔〯을〮닐ᄋᆞ샤〮ᄃᆡ〮君군子ᄌᆞ〮ᄃퟄᆫ
  라이〮러ᄐᆞᆺ〮ᄒᆞᆫ사〯ᄅᆞᆷ〮이〮여魯로〮애〮君군子ᄌᆞ〮ㅣ업〯
  스〮면〮이〮어듸〮가〮이〮ᄅᆞᆯ取ᄎᆔ〯ᄒᆞ리〮오}
\原{○子ᄌᆞ〮貢고ᇰ〯이問문〯曰왈〮賜ᄉᆞ〯也야〯ᄂᆞᆫ何하如ᅀᅧ
  ᄒᆞ니ᇰ잇고子ᄌᆞ〮ㅣ曰왈〮女여〯ᄂᆞᆫ器긔〮也야〯ㅣ니라曰왈〮何
  하器긔〮也야〯ㅣ잇고曰왈〮瑚호璉련〯也야〯ㅣ니라}
\諺{子ᄌᆞ〮貢고ᇰ〯이〮묻〯ᄌᆞ와ᄀᆞᆯ오〮ᄃᆡ〮賜ᄉᆞ〯ᄂᆞᆫ엇〯더ᄒᆞ니ᇰ〮
  잇고〮子ᄌᆞ〮ㅣᄀᆞᆯᄋᆞ샤〮ᄃᆡ〮너ᄂᆞᆫ〮器긔〮ㅣ니〮라ᄀᆞᆯ오〮
  ᄃᆡ〮엇〯던器긔〮ㅣ니ᇰ〮잇고〮ᄀᆞᆯᄋᆞ샤〮ᄃᆡ〮瑚호ㅣ며〮璉
  련〯이〮니〮라}
\原{○或혹〮이曰왈〮雍오ᇰ也야〯ᄂᆞᆫ仁ᅀᅵᆫ而ᅀᅵ不블〮佞녀ᇰ〮
  이로다}
\諺{或혹〮이〮ᄀᆞᆯ오〮ᄃᆡ雍오ᇰ은〮仁ᅀᅵᆫᄒᆞ고〮佞녀ᇰ〮티〮몯〯ᄒᆞ〮
  도다〮}
\原{子ᄌᆞ〮ㅣ曰왈〮焉언用요ᇰ〯佞녀ᇰ〮이리오禦어〯人ᅀᅵᆫ以이〯
  口구〯給급〮ᄒᆞ야屢루〯憎즈ᇰ於어人ᅀᅵᆫᄒᆞᄂᆞ니不블〮知디
  其기仁ᅀᅵᆫ이어니와焉언用요ᇰ〯佞녀ᇰ〮이리오}
\諺{子ᄌᆞ〮ㅣᄀᆞᆯᄋᆞ샤〮ᄃᆡ〮엇〯디〮佞녀ᇰ〮을〮ᄡᅳ〮리〮오人ᅀᅵᆫ을〮
  禦어〯호〯ᄃᆡ〮口구〯給급〮으〮로ᄡᅥᄒᆞ〮야〮ᄌᆞ조〮人ᅀᅵᆫ에〮
  憎즈ᇰᄒᆞ이〮ᄂᆞ니〮그仁ᅀᅵᆫ은〮아〯디〮몯〯ᄒᆞ〮거니〮와〮엇〯
  디〮佞녀ᇰ〮을〮ᄡᅳ〮리〮오}
\原{○子ᄌᆞ〮ㅣ使ᄉᆞ〯漆칠〮雕됴開ᄀᆡ로仕ᄉᆞ〯ᄒᆞ신대對ᄃᆡ〯
  曰왈〮吾오斯ᄉᆞ之지未미〯能느ᇰ信신〯이로이다子ᄌᆞ〮ㅣ
  說열〮ᄒᆞ시다}
\諺{子ᄌᆞ〮ㅣ漆칠〮雕됴開ᄀᆡ로〮ᄒᆞ여〮곰仕ᄉᆞ〯ᄒᆞ라〮ᄒᆞ〮
  신대〮對ᄃᆡ〯ᄒᆞ〮야〮ᄀᆞᆯ오〮ᄃᆡ〮내〮이〮ᄅᆞᆯ〮能느ᇰ히〮信신〯티
  몯ᄒᆞ얀〮노이다〮子ᄌᆞ〮ㅣ說열〮ᄒᆞ〮시다〮}
\原{○子ᄌᆞ〮ㅣ曰왈〮道도〯不블〮行ᄒᆡᇰ이라乘스ᇰ桴부ᄒᆞ야浮
  부于우海ᄒᆡ〯호리니從죠ᇰ〯我아〯者쟈〮ᄂᆞᆫ其기由유與
  여ᅟᅵᆫ뎌子ᄌᆞ路로〯ㅣ聞문之지ᄒᆞ고喜희〯ᄒᆞᆫ대子ᄌᆞ〮ㅣ曰
  왈〮由유也야〯ᄂᆞᆫ好호〯勇요ᇰ〯이過과〮我아〯ㅣ나無무所
  소〯取ᄎᆔ〯材ᄌᆡ로다}
\諺{子ᄌᆞ〮ㅣᄀᆞᆯᄋᆞ샤〮ᄃᆡ〮道도〯ㅣ行ᄒᆡᇰ티〮몯〯ᄒᆞ〮ᄂᆞᆫ디〮라
  桴부를〮乘스ᇰᄒᆞ〮야〮海ᄒᆡ〯에〮浮부호〯리〮니〮나〮ᄅᆞᆯ〮從
  죠ᇰ〯ᄒᆞᆯ者쟈〮ᄂᆞᆫ〮그〮由유ᅟᅵᆫ뎌〮子ᄌᆞ〮路로〯ㅣ듣고〮깃
  거〮ᄒᆞᆫ대〮子ᄌᆞ〮ㅣᄀᆞᆯᄋᆞ샤〮ᄃᆡ〮由유ᄂᆞᆫ〮勇요ᇰ〯을〮好호〯
  홈〯이〮내게過과〮ᄒᆞ나〮取ᄎᆔ〯ᄒᆞ〮야〮材ᄌᆡᄒᆞᆯ빼〮업〯도
  다〮}
\原{○孟ᄆᆡᇰ〯武무〯伯ᄇᆡᆨ〮이問문〯子ᄌᆞ〮路로〯ᄂᆞᆫ仁ᅀᅵᆫ乎호
  ㅣ잇가子ᄌᆞ〮ㅣ曰왈〮不블〮知디也야〯ㅣ로라}
\諺{孟ᄆᆡᇰ〯武무〯伯ᄇᆡᆨ〮이〮묻〯ᄌᆞ오ᄃᆡ〮子ᄌᆞ〮路로〯ᄂᆞᆫ仁ᅀᅵᆫ
  ᄒᆞ니ᇰ〮잇가〮子ᄌᆞ〮ㅣᄀᆞᆯᄋᆞ샤〮ᄃᆡ〮아〯디〮몯〯ᄒᆞ〮노라〮}
\原{又우〯問문〯ᄒᆞᆫ대子ᄌᆞ〮ㅣ曰왈〮由유也야ᄂᆞᆫ千쳔乘스ᇰ
  之지國국〮에可가〯使ᄉᆞ〯治티其기賦부〯也야〯ㅣ어니와
  不블〮知디其기仁ᅀᅵᆫ也야〯케라}
\諺{ᄯᅩ〮묻〯ᄌᆞ온대〮子ᄌᆞ〮ㅣᄀᆞᆯᄋᆞ샤〮ᄃᆡ〮由유ᄂᆞᆫ〮千쳔乘
  스ᇰ〮ㅅ나라〮ᄒᆡ可가〯히〮ᄒᆞ여〮곰그賦부〯ᄂᆞᆫ治티ᄒᆞ〮
  얌〮즉〮ᄒᆞ〮거니와그仁ᅀᅵᆫ은〮아〯디〮몯〯게라〮}
\原{求구也야〯ᄂᆞᆫ何하如ᅀᅧᄒᆞ니ᇰ잇고子ᄌᆞ〮ㅣ曰왈〮求구也
  야〯ᄂᆞᆫ千쳔室실〮之지邑읍〮과百ᄇᆡᆨ〮乘스ᇰ〮之지家가
  애可가〯使ᄉᆞ〯爲위之지宰ᄌᆡ〯也야〯ㅣ어니와不블〮知디
  其기仁ᅀᅵᆫ也야〯케라}
\諺{求구ᄂᆞᆫ〮엇〯더〮ᄒᆞ니ᇰ〮잇고〮子ᄌᆞ〮ㅣᄀᆞᆯᄋᆞ샤〮ᄃᆡ〮求구
  ᄂᆞᆫ〮千쳔室실〮ㅅ邑읍〮과百ᄇᆡᆨ〮乘스ᇰ〮ㅅ家가애〮可
  가〯히〮ᄒᆞ여〮곰宰ᄌᆡ〯되염〮즉〮ᄒᆞ〮거니와〮그〮仁ᅀᅵᆫ은〮
  아〯디〮몯〯게〮라}
\原{赤젹〮也야〯ᄂᆞᆫ何하如ᅀᅧᄒᆞ니ᇰ잇고子ᄌᆞ〮ㅣ曰왈〮赤젹〮也
  야〯ᄂᆞᆫ束속〮帶ᄃᆡ〯立립〮於어朝됴ᄒᆞ야可가〯使ᄉᆞ〯與어〯
  賓빈客ᄀᆡᆨ〮言언也야〯ㅣ어니외不블〮知디其기仁ᅀᅵᆫ也
  야〯케라}
\諺{赤젹〮은〮엇〯더ᄒᆞ니ᇰ〮잇고〮子ᄌᆞ〮ㅣᄀᆞᆯᄋᆞ샤〮ᄃᆡ〮赤젹〮
  은〮帶ᄃᆡ〯를〮束속〮ᄒᆞ〮야〮朝됴애〮立립〮ᄒᆞ〮야〮可가〯히〮
  ᄒᆞ여〮곰賓빈客ᄀᆡᆨ〮으〮로더브〮러言언ᄒᆞ얌즉〮ᄒᆞ〮
  거〮니와〮그仁ᅀᅵᆫ은〮아〯디〮몯〯게라〮}
\原{○子ᄌᆞ〮ㅣ謂위〮子ᄌᆞ〮貢고ᇰ〯曰왈〮女ᅀᅧ〯與여〯回회也
  야〯로孰슉〮愈유〯오}
\諺{子ᄌᆞ〮ㅣ子ᄌᆞ貢고ᇰ〯ᄃᆞ려〮닐어〮ᄀᆞᆯᄋᆞ샤〮ᄃᆡ〮네〯回회
  로〮더브〮러〮뉘〮愈유〯ᄒᆞ뇨〮}
\原{對ᄃᆡ〯曰왈〮賜ᄉᆞ〯也야〯ᄂᆞᆫ何하敢감〯望마ᇰ〮回회리ᇰ잇고
  回회也야〯ᄂᆞᆫ聞문一일〮以이〯知디十십〮ᄒᆞ고賜ᄉᆞ〯也
  야〯ᄂᆞᆫ聞문一일〮以이〯知디二ᅀᅵ〯ᄒᆞ노이다}
\諺{對ᄃᆡ〯ᄒᆞ〮야〮ᄀᆞᆯ오〮ᄃᆡ〮賜ᄉᆞ〯ᄂᆞᆫ〮엇〯디〮敢감〯히〮回회를〮
  ᄇᆞ〮라〮리ᇰ잇고〮回회ᄂᆞᆫ〮ᄒᆞ나흘〮들어〮ᄡᅥ〮열〮흘〮알〯고〮
  賜ᄉᆞ〯ᄂᆞᆫ〮ᄒᆞ나흘〮들어〮ᄡᅥ〮둘〯흘〮아〯노ᇰ이다〮}
\原{子ᄌᆞ〮ㅣ曰왈〮弗블〮如ᅀᅧ也야〯ㅣ니라吾오與여〯女ᅀᅧ〯
  의弗블〮如ᅀᅧ也야〯ᄒᆞ노라}
\諺{子ᄌᆞ〮ㅣᄀᆞᆯᄋᆞ샤〮ᄃᆡ〮ᄀᆞᆮ〮디〮몯〯ᄒᆞ니〮라내〮네의〮ᄀᆞᆮ〮디〮
  몯〯호〮라〮홈〯을〮與여〯ᄒᆞ〮노라}
\原{○宰ᄌᆡ〮予여ㅣ晝듀〮寢침〯이어ᄂᆞᆯ子ᄌᆞ〮ㅣ曰왈〮朽후〯
  木목〮은不블〮可가〯雕됴也야〯ㅣ며糞분〮土토〮之지牆
  쟈ᇰ은不블〮可가〯朽오也야〯ㅣ니於어予여與여에何
  하誅듀ㅣ리오}
\諺{宰ᄌᆡ〯予여ㅣ晝듀〮에寢침〯ᄒᆞ〮거ᄂᆞᆯ〮子ᄌᆞ〮ㅣᄀᆞᆯᄋᆞ
  샤〮ᄃᆡ〮朽후〯ᄒᆞᆫ〮木목〮은〮可가〯히〮雕됴티〮몯〯ᄒᆞᆯ〮꺼시〮
  며〮糞분〮土토〮ㅅ牆쟈ᇰ은〮可가〯히〮朽오티〮몯〯ᄒᆞᆯ〮꺼
  시〮니予여에〮엇〯디〮誅듀ᄒᆞ리〮오}
\原{子ᄌᆞ〮ㅣ曰왈〮始시〯吾오ㅣ於어人ᅀᅵᆫ也야〯애聽텨ᇰ〯
  其기言언而ᅀᅵ信신〯其기行ᄒᆡᇰ〯이라니今금吾오ㅣ
  於어人ᅀᅵᆫ也야〯애聽텨ᇰ〯其기言언而ᅀᅵ觀관其기
  行ᄒᆡᇰ〯ᄒᆞ노니於어予여〯與여에改ᄀᆡ〯是시〯와라}
\諺{子ᄌᆞ〮ㅣᄀᆞᆯᄋᆞ샤〮ᄃᆡ〮비르〮소내〮人ᅀᅵᆫ의〮게〮그言언
  을〮듣고〮그行ᄒᆡᇰ〯을〮信신〯ᄒᆞ〮다니이〮제〮내〮人ᅀᅵᆫ의〮
  게〮그言언을〮듣고〮그行ᄒᆡᇰ〯을〮觀관ᄒᆞ〮노니〮予여〯
  의게〮이〮ᄅᆞᆯ〮改ᄀᆡ〯ᄒᆞ과〮라}
\原{○子ᄌᆞ〮ㅣ曰왈〮吾오未미〯見견〯剛가ᇰ者쟈〮케라或혹〮
  이對ᄃᆡ〯曰왈〮申신棖뎌ᇰ이니ᇰ이다子ᄌᆞ〮ㅣ曰왈〮棖뎌ᇰ也
  야〯ᄂᆞᆫ慾욕〮이어니焉언得득〮剛가ᇰ이리오}
\諺{子ᄌᆞ〮ㅣᄀᆞᆯᄋᆞ샤〮ᄃᆡ〮내〮剛가ᇰᄒᆞᆫ〮者쟈〮를〮보디〮몯〯게〮
  라或혹〮이〮對ᄃᆡ〯ᄒᆞ〮야〮ᄀᆞᆯ오〮ᄃᆡ〮申신棖뎌ᇰ이〮니ᇰ이〮
  다〮子ᄌᆞ〮ㅣᄀᆞᆯᄋᆞ샤〮ᄃᆡ〮棖뎌ᇰ은〮慾욕〮ᄒᆞ거니〮엇〯디〮
  시러〮곰剛가ᇰᄒᆞ리〮오}
\原{○子ᄌᆞ〮貢고ᇰ〯이曰왈〮我아〯不블〮欲욕〮人ᅀᅵᆫ之지加
  가諸져我아〯也야〯를吾오亦역〮欲욕〮無무加가諸
  져人ᅀᅵᆫᄒᆞ노이다子ᄌᆞ〮ㅣ曰왈〮賜ᄉᆞ也야〯아非비爾ᅀᅵ〯
  所소〯及급〮也야〯ㅣ니라}
\諺{子ᄌᆞ〮貢고ᇰ〯이〮ᄀᆞᆯ오〮ᄃᆡ〮내〮人ᅀᅵᆫ이〮내게加가ᄒᆞ과〮
  댜〮아니〮ᄒᆞ〮ᄂᆞᆫ거슬〮내〮ᄯᅩ〮ᄒᆞᆫ人ᅀᅵᆫ의〮게〮加가홈〯이〮
  업〯고〮져ᄒᆞ〮노ᇰ이다〮子ᄌᆞ〮ㅣᄀᆞᆯᄋᆞ샤〮ᄃᆡ〮賜ᄉᆞ〯아〮네
  의〮及급〮홀〯빼〮아〮니〮니라〮}
\原{○子ᄌᆞ〮貢고ᇰ〯이曰왈〮夫부子ᄌᆞ〮之지文문章쟈ᇰ은
  可가〯得득〮而ᅀᅵ聞문也야〯ㅣ어니와夫부子ᄌᆞ〮之지言
  언性셔ᇰ〯與여〯天텬道도〯ᄂᆞᆫ不블〮可가〯得득〮而ᅀᅵ聞
  문也야〯ㅣ니라}
\諺{子ᄌᆞ〮貢고ᇰ〯이〮ᄀᆞᆯ오ᄃᆡ〮夫부子ᄌᆞ〮의〮文문章쟈ᇰ은〮
  可가〯히〮시러〮곰드르려〮니와〮夫부子ᄌᆞ〮의〮性셔ᇰ〯
  과〮다ᄆᆞᆺ〮天텬道도〯를〮닐ᄋᆞ샤〮믄〮可가〯히〮시러〮곰
  듣디〮몯ᄒᆞᆯ이〮니라〮}
\原{○子ᄌᆞ〮路로〯ᄂᆞᆫ有유〯聞문이오未미〯之지能느ᇰ行ᄒᆡᇰ
  ᄒᆞ야셔唯유恐고ᇰ〯有유〯聞문ᄒᆞ더라}
\諺{子ᄌᆞ〮路로〮ᄂᆞᆫ〮드롬〮이〮잇고〮能느ᇰ히〮行ᄒᆡᇰ티〮몯〯ᄒᆞ〮
  야〮셔드롬〮이〮이실가〮저허〮ᄒᆞ더라〮}
\原{○子ᄌᆞ〮貢고ᇰ〮이問문〯曰왈〮孔고ᇰ〮文문子ᄌᆞ〮를何하
  以이〯謂위〮之지文문也야〯ㅣ잇고子ᄌᆞ〮ㅣ曰왈〮敏민〮
  而ᅀᅵ好호〯學ᄒᆞᆨ〮ᄒᆞ며不블〮恥티〯下하〯問문〯이라是시〯以
  이〯謂위〮之지文문也야〯ㅣ니라}
\諺{子ᄌᆞ〮貢고ᇰ〯이〮묻〯ᄌᆞ〮와〮ᄀᆞᆯ오〮ᄃᆡ〮孔고ᇰ〮文문子ᄌᆞ〮를〮
  엇〯디〮ᄡᅥ〮文문이〮라니르니ᇰ〮잇고〮子ᄌᆞ〮ㅣᄀᆞᆯᄋᆞ샤〮
  ᄃᆡ〮敏민〯ᄒᆞ고〮學ᄒᆞᆨ〮을〮好호〯ᄒᆞ며下하〯問문〯을〮恥
  티〯티〮아니〮ᄒᆞᆫ디〮라일〮로ᄡᅥ〮文문이〮라니ᄅᆞ니〮라}
\原{○子ᄌᆞ〮ㅣ謂위〮子ᄌᆞ〮產산〯ᄒᆞ샤ᄃᆡ有유〯君군子ᄌᆞ〮之
  지道도〯ㅣ四ᄉᆞ〯焉언이니其기行ᄒᆡᇰ己긔〮也야〯ㅣ恭
  고ᇰᄒᆞ며其기事ᄉᆞ〯上샤ᇰ〯也야〯ㅣ敬겨ᇰ〯ᄒᆞ며其기養야ᇰ〯民
  민也야〯ㅣ惠혜〯ᄒᆞ며其기使ᄉᆞ〯民민也야〯ㅣ義의〯니라}
\諺{子ᄌᆞ〮ㅣ子ᄌᆞ〮產산〯을〮닐ᄋᆞ샤〮ᄃᆡ〮君군子ᄌᆞ〮의〮道
  도〯ㅣ네〯히〮인ᄂᆞ니〮그己긔〮를〮行ᄒᆡᇰ홈〯이〮恭고ᇰᄒᆞ
  며〮그上샤ᇰ〯을〮事ᄉᆞ〯홈〯이〮敬겨ᇰ〯ᄒᆞ며〮그民민을〮養
  야ᇰ〯홈〯이〮惠혜〯ᄒᆞ며〮그民민을〮使ᄉᆞ〯홈〯이〮義의〯ᄒᆞ
  니라}
\原{○子ᄌᆞ〮ㅣ曰왈〮晏안〯平펴ᇰ仲듀ᇰ〯은善션〯與여〯人ᅀᅵᆫ
  交교ㅣ로다久구〯而ᅀᅵ敬겨ᇰ〯之지온여}
\諺{子ᄌᆞ〮ㅣᄀᆞᆯᄋᆞ샤〮ᄃᆡ〮晏안〯平펴ᇰ仲듀ᇰ〯은〮人ᅀᅵᆫ으〮로
  더브〮러交교홈〯을〮善션〯히〮ᄒᆞ〮놋〮다〮오라〮되〮敬겨ᇰ〯
  ᄒᆞ곤〮여}
\原{○子ᄌᆞ〮ㅣ曰왈〮臧자ᇰ文문仲듀ᇰ〯이居거蔡채〯호ᄃᆡ山
  산節졀〮藻조〯梲졀〮ᄒᆞ니何하如ᅀᅧ其기知디〮也야〯ㅣ리
  오}
\諺{子ᄌᆞ〮ㅣᄀᆞᆯᄋᆞ샤〮ᄃᆡ〮臧자ᇰ文문仲듀ᇰ〯이〮蔡채〯를〮居
  거호〯ᄃᆡ〮節졀〮애〮山산을〮ᄒᆞ며〮梲졀〮애〮藻조〯를〮ᄒᆞ
  니〮엇〯디그知디〮라〮ᄒᆞ리〮오}
\原{○子ᄌᆞ〮張댜ᇰ이問문〯曰왈〮令려ᇰ〮尹윤〯子ᄌᆞ〮文문이
  三삼仕ᄉᆞ〯爲위令려ᇰ〮尹윤〯호ᄃᆡ無무喜희〯色ᄉᆡᆨ〮ᄒᆞ며三
  삼已이〯之지호ᄃᆡ無무慍온〯色ᄉᆡᆨ〮ᄒᆞ야舊구〯令려ᇰ〮尹윤〯
  之지政져ᇰ〮을必필〮以이〯告고〯新신令려ᇰ〮尹윤〯ᄒᆞ니何
  하如ᅀᅧᄒᆞ니ᇰ잇고子ᄌᆞ〮ㅣ曰왈〮忠튜ᇰ矣의〯니라曰왈〮仁ᅀᅵᆫ
  矣의〯乎호ㅣ잇가曰왈〮未미〯知디케라焉언得득〮仁ᅀᅵᆫ
  이리오}
\諺{子ᄌᆞ〮張댜ᇰ이〮묻〯ᄌᆞ〮와〮ᄀᆞᆯ오〮ᄃᆡ〮令려ᇰ〮尹윤〯子ᄌᆞ〮文
  문이〮세〯번仕ᄉᆞ〯ᄒᆞ〮야〮令려ᇰ〮尹윤〯이〮도요〯ᄃᆡ〮喜희〯
  ᄒᆞᆫ色ᄉᆡᆨ〮이업〯스〮며세〯번已이〯호〯ᄃᆡ〮慍온〯ᄒᆞᆫ色ᄉᆡᆨ〮
  이〮업〯서〮녯〯令려ᇰ〮尹윤〯의〮政져ᇰ〮을〮반〮ᄃᆞ시ᄡᅥ〮新신
  令려ᇰ〮尹윤〯에〮告고〯ᄒᆞ니〮엇〯더ᄒᆞ니ᇰ〮잇〮고〮子ᄌᆞ〮ㅣ
  ᄀᆞᆯᄋᆞ샤〮ᄃᆡ〮忠튜ᇰᄒᆞ니〮라ᄀᆞᆯ오〯ᄃᆡ〮仁ᅀᅵᆫᄒᆞ니ᇰ〮잇〮가〮
  ᄀᆞᆯᄋᆞ샤〮ᄃᆡ〮아〯디〮몯〯게〮라엇〯디〮시러〯곰仁ᅀᅵᆫᄒᆞ리〮
  오}
\原{崔최子ᄌᆞ〮ㅣ弑시〯齊졔君군이어ᄂᆞᆯ陳딘文문子ᄌᆞ〮
  ㅣ有유〯馬마〯十십〮乘스ᇰ〯이러니棄기〯而ᅀᅵ違위之지
  ᄒᆞ고至지〮於어他타邦바ᇰᄒᆞ야則즉〮曰왈〮猶유吾오大
  대〯夫부崔최子ᄌᆞ〮也야〯ㅣ라ᄒᆞ고違위之지ᄒᆞ며之지一
  일〮邦바ᇰᄒᆞ야則즉〮又우〯曰왈〮猶유吾오大대〯夫부崔
  죄子ᄌᆞ〮也야〯ㅣ라ᄒᆞ고違위之지ᄒᆞ니何하如ᅀᅧᄒᆞ니ᇰ잇고子
  ᄌᆞ〮ㅣ曰왈〮淸쳐ᇰ矣의〯ㅣ니라曰왈〮仁ᅀᅵᆫ矣의〯乎호ㅣ잇
  가曰왈〮未미〯知디케라焉언得득〮仁ᅀᅵᆫ이리오}
\諺{崔최子ᄌᆞ〮ㅣ齊졔君군을〮弑시〯ᄒᆞ〮야〮ᄂᆞᆯ〮陳딘文
  문子ᄌᆞ〮ㅣ馬마〯十십〮乘스ᇰ〯을〮둣〮더니〮棄기〯ᄒᆞ〮야〮
  違위ᄒᆞ고〮他타邦바ᇰ애〮至지〮ᄒᆞ〮야〮곧〮ᄀᆞᆯ오〮ᄃᆡ〮우〮
  리태〮우崔최子ᄌᆞ〮ᄀᆞᆮ〮다ᄒᆞ고〮違위ᄒᆞ며〮一일〮邦
  바ᇰ애〮之지ᄒᆞ〮야〮곧〮ᄯᅩ〮ᄀᆞᆯ오〮ᄃᆡ〮우〮리태〮우崔최子
  ᄌᆞ〮ᄀᆞᆮ〮다ᄒᆞ고〮違위ᄒᆞ니〮엇〯더ᄒᆞ니ᇰ〮잇고〮子ᄌᆞ〮ㅣ
  ᄀᆞᆯᄋᆞ샤〮ᄃᆡ〮淸쳐ᇰᄒᆞ니〮라ᄀᆞᆯ오〮ᄃᆡ〮仁ᅀᅵᆫᄒᆞ니ᇰ〮잇가〮
  ᄀᆞᆯᄋᆞ샤〮ᄃᆡ〮아〯디〮몯〯게〮라엇〯디〮시러〮곰仁ᅀᅵᆫᄒᆞ리〮
  오}
\原{○季계〯文문子ᄌᆞ〮ㅣ三삼〮思ᄉᆞ而ᅀᅵ後후〯에行ᄒᆡᇰ
  ᄒᆞ더니子ᄌᆞ〮ㅣ聞문之지ᄒᆞ시고曰왈〮再ᄌᆡ〯ㅣ斯ᄉᆞ可
  가〯矣의〯ㅣ니라}
\諺{季계〯文문子ᄌᆞ〮ㅣ세〯번思ᄉᆞᄒᆞᆫ後후〮에〮行ᄒᆡᇰᄒᆞ〮
  더니〮子ᄌᆞ〮ㅣ드ᄅᆞ시〮고ᄀᆞᆯᄋᆞ샤〮ᄃᆡ〮再ᄌᆡ〯ㅣ可가〯
  ᄒᆞ니〮라}
\原{○子ᄌᆞ〮ㅣ曰왈〮寗녀ᇰ〮武무〯子ᄌᆞ〮ㅣ邦바ᇰ有유〯道도〯
  則즉〮知디〮ᄒᆞ고邦바ᇰ無무道도〯則즉〮愚우ᄒᆞ니其기知
  디〮ᄂᆞᆫ可가〯及급〮也야〯ㅣ어니와其기愚우ᄂᆞᆫ不블〮可가〯
  及급〮也야〯ㅣ니라}
\諺{子ᄌᆞ〮ㅣᄀᆞᆯᄋᆞ샤〮ᄃᆡ〮寗녀ᇰ〮武무〯子ᄌᆞ〮ㅣ邦바ᇰ이〮道
  도〯ㅣ이시면〮知디〮ᄒᆞ고〮邦바ᇰ이〮道도〯ㅣ업〯스면〮
  愚우ᄒᆞ니〮그知디〮ᄂᆞᆫ〮可가〯히〮及급〮ᄒᆞ려〮니와〮[?|그]
  愚우ᄂᆞᆫ〮可가〯히〮及급〮디몯〯ᄒᆞᆯ이〮니라〮}
\原{○子ᄌᆞ〮ㅣ在ᄌᆡ〯陳딘ᄒᆞ샤曰왈〮歸귀與여歸귀與여
  ᅟᅵᆫ뎌吾오黨다ᇰ〮之지小쇼〯子ᄌᆞ〮ㅣ狂과ᇰ簡간〯ᄒᆞ야斐비〯
  然ᅀᅧᆫ成셔ᇰ章쟈ᇰ이오不블〮知디所소〯以이〯裁ᄌᆡ之지
  로다}
\諺{子ᄌᆞ〮ㅣ陳딘에〮겨〯샤〮ᄀᆞᆯᄋᆞ샤〮ᄃᆡ〮歸귀홀〯띤〮뎌〮歸
  귀홀〯띤〮뎌〮우〮리黨다ᇰ〮앳〮小쇼〯子ᄌᆞ〮ㅣ狂과ᇰ簡간〯
  ᄒᆞ〮야〮斐비〯然ᅀᅧᆫ히〮章쟈ᇰ을〮成셔ᇰᄒᆞ고〮ᄡᅥ裁ᄌᆡ홀〯
  빠〮를〮아〯디〮몯〯ᄒᆞ〮놋다〮}
\原{○子ᄌᆞ〮ㅣ曰왈〮伯ᄇᆡᆨ〮夷이叔슉〮齊졔ᄂᆞᆫ不블〮念념〯
  舊구〯惡악〮이라怨원〯是시〯用요ᇰ〯希희니라}
\諺{子ᄌᆞ〮ㅣᄀᆞᆯᄋᆞ샤〮ᄃᆡ〮伯ᄇᆡᆨ〮夷이와〮叔슉〮齊졔ᄂᆞᆫ〮舊
  구〯惡악〮을〮念념〯티〮아니〮ᄒᆞ〮논디〮라怨원〯이〮일〮로
  ᄡᅥ〮드〮므〮니라〮}
\原{○子ᄌᆞ〮ㅣ曰왈〮孰슉〮謂위〮微미生ᄉᆡᇰ高고直딕〮고
  或혹〮이乞걸〮醯혜〮焉언이어ᄂᆞᆯ乞걸〮諸져其기隣린
  而ᅀᅵ與여〯之지온여}
\諺{子ᄌᆞ〮ㅣᄀᆞᆯᄋᆞ샤〮ᄃᆡ〮뉘〮微미生ᄉᆡᇰ高고를〮닐오〮ᄃᆡ〮
  直딕〮다ᄒᆞ〮ᄂᆞ뇨〮或혹〮이醯혜를〮乞걸〮ᄒᆞ〮여ᄂᆞᆯ〮그
  隣린에〮乞걸〮ᄒᆞ〮야〮與여〯ᄒᆞ곤〮여}
\原{○子ᄌᆞ〮ㅣ曰왈〮巧교〯言언令려ᇰ〮色ᄉᆡᆨ〮足주〯恭고ᇰ을
  左자〯丘구明며ᇰ이恥티〯之지러니丘구〮亦역〮恥티〯之
  지ᄒᆞ노라匿닉〮怨원〯而ᅀᅵ友우〯其기人ᅀᅵᆫ을左자〯丘
  구明며ᇰ이恥티〯之지러니丘구亦역〮恥티〯之지ᄒᆞ노라}
\諺{子ᄌᆞ〮ㅣᄀᆞᆯᄋᆞ샤〮ᄃᆡ〮言언을〮巧교〯히〮ᄒᆞ며〮色ᄉᆡᆨ〮을〮
  令려ᇰ〮히〮ᄒᆞ며〮恭고ᇰ을〮足주〯히〮홈〯을〮左자〯丘구明
  며ᇰ이〮恥티〯ᄒᆞ〮더니〮丘구ㅣᄯᅩ〮ᄒᆞᆫ恥티〯ᄒᆞ〮노라怨
  원〯을〮匿닉〮ᄒᆞ고〮그사〯ᄅᆞᆷ을〮友우〯홈〯을〮左자〯丘구
  明며ᇰ이〮恥티〯ᄒᆞ〮더니〮丘구ㅣᄯᅩ〮ᄒᆞᆫ恥티〯ᄒᆞ노라}
\原{○顏안淵연季계〯路로〯ㅣ侍시〯러니子ᄌᆞ〮ㅣ曰왈〮盍
  합〮各각〮言언爾ᅀᅵ〯志지〮리오}
\諺{顏안淵연과〮季계〯路로〯ㅣ侍시〯ᄒᆞ〮얏더니〮子ᄌᆞ〮
  ㅣᄀᆞᆯᄋᆞ샤〮ᄃᆡ〮엇〯디〮각〮각〮네의〮ᄠᅳᆮ〮을〮니ᄅᆞ디〮아니〮
  ᄒᆞ리〮오}
\原{子ᄌᆞ〮路로〯ㅣ曰왈〮願원〯車거馬마〯와衣의〯輕겨ᇰ裘
  구를與여〯朋브ᇰ友우〯共고ᇰ〯ᄒᆞ야弊폐〯之지而ᅀᅵ無무
  憾감〯ᄒᆞ노ᇰ이다}
\諺{子ᄌᆞ〮路로〯ㅣᄀᆞᆯ오〮ᄃᆡ〮願원〯컨댄〮車거馬마〯와〮輕
  겨ᇰ裘구를〮衣의〯홈〯을〮朋브ᇰ友우〯로더브〮러ᄒᆞᆫ가
  지〮로〮ᄒᆞ〮야〮弊폐〯ᄒᆞ〮야〮도憾감〯홈〯이〮업〯고져ᄒᆞ〮노ᇰ
  이다〮}
\原{顏안淵연이曰왈〮願원〯無무伐벌〮善션〯ᄒᆞ며無무施
  시勞로ᄒᆞ노ᇰ이다}
\諺{顏안淵연이〮ᄀᆞᆯ오〮ᄃᆡ〮願원〯컨댄〮善션〯을〮伐벌〮홈〯
  이〮업〯ᄉᆞ며〮勞로를〮施시[?|홈〯]이〮업〯고져ᄒᆞ〮노ᇰ이다〮}
\原{子ᄌᆞ〮路로〯ㅣ曰왈〮願원〯聞문子ᄌᆞ〮之지志지〮ᄒᆞ노ᇰ이다
  子ᄌᆞ〮ㅣ曰왈〮老로〯者쟈〮를安안之지ᄒᆞ며朋브ᇰ友우〯
  를〮信신〯之지ᄒᆞ며少쇼〯者쟈〮를懷회之지니라}
\諺{子ᄌᆞ〮路로〯ㅣᄀᆞᆯ오〮ᄃᆡ〮願원〯컨댄〮子ᄌᆞ〮의〮志지〮를〮
  듣ᄌᆞᆸ〮고져〮ᄒᆞ〮노ᇰ이다〮子ᄌᆞ〮ㅣᄀᆞᆯᄋᆞ샤〮ᄃᆡ〮老로〯者
  쟈〮를〮安안ᄒᆞ며〮朋브ᇰ友우〯를〮信신〯으로ᄒᆞ며〮少
  쇼〯者쟈〮를懷회홈〯이〮니라〮}
\end{document}

%%% Local Variables:
%%% mode: latex 
%%% TeX-engine: luatex
%%% TeX-master: t 
%%% End:
