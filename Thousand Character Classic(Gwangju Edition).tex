\documentclass{easternClassics}
\usepackage{kotex}

\setmainfont        {Noto Serif KR Bold}
\setmainhangulfont  {Noto Serif JP Bold}
\setmainhanjafont   {Noto Serif SC Bold}
\setmainfallbackfont{Noto Serif TC Bold}
\newfontfamily{\mediumKR}{Noto Serif KR Medium}[Script=Hangul]
\setCharOption{center}{}{font=\mediumKR,scale=2.7}

\四周單邊{0.25}
\半郭{22}{16}
\有界{0.05}
\半葉{4}{3}
\黑口{1/2}{5.25}{5.25}
\魚尾{simpleTail}{simpleTail}
\版心{1.75}
\版心字間{1.15}
\張次位置{0.85}

\newcommand{\character}[2]{
  \begin{tikzpicture}
    \clip (-2.5,-1) rectangle (2.5,5);
    \node[scale=11.5] at (0,8/3){#1};
    \node[scale= 2.7] at (0,0  ){\mediumKR{\rightToLeft{#2}}};
  \end{tikzpicture}}
\def\字#1#2{\文{[put|\string\character{#1}{#2}|yshift=-0.1cm]}}

\begin{document}
\張次{1}
\字{千}{}
\字{字}{}
\字{文}{}
\改行
\字{天}{하ᄂᆞᆯ 텬}
\字{地}{ᄯᅡ 디}
\字{玄}{가ᄆᆞᆯ 현}
\字{黃}{누를 황}
\字{宇}{집 우}
\字{宙}{집 듀}
\字{洪}{너블 홍}
\字{荒}{거츨 황}
\字{日}{날 ᅀᅵᆯ}
\字{月}{ᄃᆞᆯ 월}
\字{盈}{ᄎᆞᆯ 영}
\字{仄}{기울 ᄎᆡᆨ}
\字{辰}{미르 진}
\字{宿}{잘 슉}
\字{列}{벌 렬}
\字{張}{베플 댱}
\字{寒}{ᄎᆞᆯ 한}
\字{來}{올 ᄅᆡ}
\字{暑}{더울 셔}
\字{往}{갈 왕}
\字{秋}{ᄀᆞᅀᆞᆯ 츄}
\字{收}{가ᄃᆞᆯ 슈}
\字{冬}{겨ᅀᅳ 동}
\字{藏}{갈ᄆᆞᆯ 장}
\字{閏}{부를 윤}
\字{餘}{나ᄆᆞᆯ 여}
\字{成}{일 셩}
\字{歲}{ᄒᆡ 셰}
\字{律}{법ᄧᆞᆯ 률}
\字{呂}{법ᄧᆞᆯ 려}
\字{調}{고ᄅᆞᆯ 됴}
\字{陽}{나ᄆᆡ 양}
\字{雲}{구룸 운}
\字{騰}{ᄂᆞᆯ 등}
\字{致}{니를 티}
\字{雨}{비 우}
\字{露}{이슬 로}
\字{結}{ᄆᆡᆯ 결}
\字{爲}{ᄒᆞᆯ 위}
\字{霜}{셔리 상}
\字{金}{쇠 금}
\字{生}{날 ᄉᆡᆼ}
\字{麗}{나오머글 려}
\字{水}{믈 슈}
\字{玉}{구슬 옥}
\字{出}{날 츌}
\字{崑}{묏브리 곤}
\字{岡}{묏브리 강}
\字{劒}{칼 검}
\字{號}{일훔 호}
\字{巨}{클 거}
\字{闕}{집 궐}
\字{珠}{구슬 쥬}
\字{稱}{잇ᄀᆞᄅᆞ 친}
\字{夜}{밤 야}
\字{光}{빗 광}
\字{果}{여름 과}
\字{珍}{그르 딘}
\字{李}{외앋 니}
\字{奈}{멋 내}
\字{菜}{ᄂᆞᄆᆞᆯ ᄎᆡ}
\字{重}{므거울 듕}
\字{芥}{계ᄌᆞᆺ 개}
\字{薑}{ᄉᆡᆼ양 강}
\字{海}{바다 ᄒᆡ}
\字{鹹}{ᄧᆞᆯ 함}
\字{河}{ᄀᆞᄅᆞᆷ 하}
\字{淡}{ᄆᆞᆯᄀᆞᆯ 담}
\字{鱗}{비ᄂᆞᆯ 린}
\字{潛}{ᄌᆞᆷᄀᆞᆯ ᄌᆞᆷ}
\字{羽}{지 우}
\字{翔}{ᄂᆞᆯ개 샹}
\字{龍}{미ᄅᆞ 룡}
\字{師}{스승 ᄉᆞ}
\字{火}{블 화}
\字{帝}{님금 뎨}
\字{鳥}{새 됴}
\字{官}{귀 관}
\字{人}{사ᄅᆞᆷ 인}
\字{皇}{님금 황}
\字{始}{비르슬 시}
\字{制}{ᄆᆞᄅᆞᆯ 졔}
\字{文}{글월 문}
\字{字}{글월 ᄌᆞ}
\字{乃}{사 내}
\字{服}{옷 복}
\字{衣}{옷 의}
\字{裳}{고외 샹}
\字{推}{밀 츄}
\字{位}{벼슬 위}
\字{讓}{ᄉᆡ양 양}
\字{國}{나라 국}
\字{有}{이실 유}
\字{虞}{나라 우}
\字{陶}{딜아비 도}
\字{唐}{대랑 당}
\字{弔}{됴문 됴}
\字{民}{ᄇᆡᆨ셩 민}
\字{伐}{버힐 벌}
\字{罪}{허믈 죄}
\字{周}{두ᄅᆞ 쥬}
\字{發}{베플 발}
\字{殷}{은국 은}
\字{湯}{더을 탕}
\字{坐}{안ᄌᆞᆯ 자}
\字{朝}{아ᄎᆞᆷ 됴}
\字{問}{무를 문}
\字{道}{도릿 도}
\字{垂}{드를 슈}
\字{拱}{고ᄌᆞᆯ 공}
\字{平}{평ᄒᆞᆯ 평}
\字{章}{글월 쟝}
\字{愛}{ᄃᆞᆺᄋᆞᆯ ᄋᆡ}
\字{育}{칠 휵}
\字{黎}{가ᄆᆞᆯ 려}
\字{首}{마리 슈}
\字{臣}{신하 신}
\字{伏}{굿블 복}
\字{戎}{되 융}
\字{羌}{되 강}
\字{遐}{멀 하}
\字{邇}{갓가올 이}
\字{壹}{ᄒᆞᆫ 일}
\字{體}{몸 톄}
\字{率}{ᄃᆞ닐 졸}
\字{賓}{손 빙}
\字{歸}{갈 귀}
\字{王}{긔ᄌᆞ 왕}
\字{鳴}{울 명}
\字{鳳}{새 봉}
\字{在}{이실 ᄌᆡ}
\字{樹}{나모 슈}
\字{白}{ᄒᆡᆫ ᄇᆡᆨ}
\字{駒}{ᄆᆡ야지 구}
\字{食}{밥 식}
\字{場}{맏 당}
\字{化}{될 화}
\字{被}{니블 피}
\字{草}{플 초}
\字{木}{나모 목}
\字{賴}{힘니블 뢰}
\字{及}{밋 급}
\字{萬}{일만 만}
\字{方}{못 방}
\字{蓋}{두웨 개}
\字{此}{이 ᄎᆞ}
\字{身}{몸 신}
\字{髮}{터럭 발}
\字{四}{넉 ᄉᆞ}
\字{大}{큰 대}
\字{五}{다ᄉᆞᆺ 오}
\字{常}{샹녜 샹}
\字{恭}{온공 공}
\字{惟}{오직 유}
\字{鞠}{칠 국}
\字{養}{칠 양}
\字{豈}{엇뎌 긔}
\字{敢}{구ᄐᆡᆯ 감}
\字{毁}{헐 ᄒᆀ}
\字{傷}{헐 샹}
\字{女}{겨집 녀}
\字{慕}{ᄉᆞ못 모}
\字{貞}{고ᄃᆞᆫ 뎡}
\字{潔}{ᄆᆞᆯᄀᆞᆯ 결}
\字{男}{아ᄃᆞ 남}
\字{效}{ᄌᆞ욀 효}
\字{才}{죄조 ᄌᆡ}
\字{良}{알 량}
\字{知}{알 디}
\字{過}{디날 과}
\字{必}{반득 필}
\字{改}{가ᄉᆡᆯ ᄀᆡ}
\字{得}{시를 득}
\字{能}{능ᄒᆞᆯ 능}
\字{莫}{말 막}
\字{忘}{니즐 망}
\字{罔}{거츨 망}
\字{談}{말ᄉᆞᆷ 담}
\字{彼}{뎌 피}
\字{短}{뎌를 단}
\字{靡}{안등 미}
\字{恃}{미들 시}
\字{己}{몸 긔}
\字{長}{긴 댱}
\字{信}{미들 신}
\字{使}{브릴 ᄉᆞ}
\字{可}{직 가}
\字{覆}{두플 복}
\字{器}{긔용 긔}
\字{欲}{바개 욕}
\字{難}{얼려울 란}
\字{量}{혜아리 량}
\字{墨}{믁 믁}
\字{悲}{슬ᄒᆞᆯ 비}
\字{絲}{실 ᄉᆞ}
\字{染}{므들 염}
\字{詩}{글월 시}
\字{讚}{기릴 찬}
\字{羔}{염 고}
\字{羊}{염 양}
\字{景}{볃 경}
\字{行}{녈 ᄒᆡᆼ}
\字{維}{얼글 유}
\字{賢}{어딜 현}
\字{剋}{이긜 극}
\字{念}{념ᄒᆞᆯ 념}
\字{作}{지을 작}
\字{聖}{님금 셩}
\字{德}{큰 덕}
\字{建}{셜 건}
\字{名}{ᅀᅵᆯ홈 명}
\字{立}{셜 닙}
\字{形}{즛 형}
\字{端}{귿 단}
\字{表}{밧 표}
\字{正}{못 졍}
\字{空}{뷜 공}
\字{谷}{골 곡}
\字{傳}{옴길 뎐}
\字{聲}{소ᄅᆡ 셩}
\字{虛}{뷜 허}
\字{堂}{집 당}
\字{習}{ᄇᆡ홀 습}
\字{聽}{드를 텽}
\字{禍}{ᄌᆡ홧 화}
\字{因}{지즐 인}
\字{惡}{모딜 악}
\字{積}{울 젹}
\字{福}{복 복}
\字{緣}{말믜 연}
\字{善}{어딜 션}
\字{慶}{길결 경}
\字{尺}{자 쳑}
\字{璧}{구슬 벽}
\字{非}{안득 비}
\字{寶}{보ᄇᆡᆺ 보}
\字{寸}{ᄆᆞᄃᆡ 촌}
\字{陰}{ᄀᆞᄂᆞᆯ 음}
\字{是}{이 시}
\字{競}{ᄃᆞ톨 경}
\字{資}{부ᄂᆞᆯ ᄌᆞ}
\字{父}{아비 부}
\字{事}{셤길 ᄉᆞ}
\字{君}{님굼 군}
\字{曰}{ᄀᆞᆯ 왈}
\字{嚴}{클 엄}
\字{與}{다ᄆᆞᆺ 여}
\字{敬}{공경 경}
\字{孝}{효도 효}
\字{當}{반ᄃᆞᆨ 당}
\字{竭}{다ᄋᆞᆯ 갈}
\字{力}{힘 녁}
\字{忠}{튱셩 튱}
\字{則}{법즉 즉}
\字{盡}{다ᄋᆞᆯ 진}
\字{命}{목숨 명}
\字{臨}{디늘 림}
\字{深}{기플 심}
\字{履}{ᄇᆞᆯ올 리}
\字{薄}{열울 박}
\字{夙}{녜 슈}
\字{興}{닐 흥}
\字{溫}{ᄃᆞᄉᆞᆯ 온}
\字{淸}{ᄎᆞᆯ 쳥}
\字{似}{ᄀᆞᄐᆞᆯ ᄉᆞ}
\字{蘭}{난초 난}
\字{斯}{이 ᄉᆞ}
\字{馨}{곳다을 항}
\字{如}{ᄀᆞᄐᆞᆯ 여}
\字{松}{솔 숑}
\字{之}{갈 지}
\字{盛}{셩ᄒᆞᆯ 셩}
\字{川}{내 쳔}
\字{流}{흐를 류}
\字{不}{안득 블}
\字{息}{쉴 식}
\字{淵}{못 연}
\字{澄}{ᄆᆞᆯᄀᆞᆯ 딩}
\字{取}{아ᄋᆞᆯ ᄌᆔ}
\字{映}{ᄇᆞᄋᆡᆯ 영}
\字{容}{즛 옹}
\字{止}{그츨 지}
\字{若}{ᄀᆞᄐᆞᆯ 약}
\字{思}{ᄉᆞ량 ᄉᆞ}
\字{言}{말ᄉᆞᆷ 언}
\字{辭}{말 ᄉᆞ}
\字{安}{편안 안}
\字{定}{뎡ᄒᆞᆯ 뎡}
\字{篤}{도타올 독}
\字{初}{처엄 초}
\字{誠}{졍셩 셩}
\字{美}{아ᄅᆞᆷ다올 미}
\字{愼}{삼갈 신}
\字{終}{ᄆᆞᄎᆞᆷ 죵}
\字{宜}{맛당 의}
\字{令}{ᄒᆡ 령}
\字{榮}{영화 영}
\字{業}{업 업}
\字{所}{바 소}
\字{基}{터 긔}
\字{籍}{글월 젹}
\字{甚}{심ᄒᆞᆯ 심}
\字{無}{업슬 무}
\字{竟}{ᄆᆞᄎᆞᆷ 경}
\字{學}{ᄇᆡ흘 ᄒᆡᆨ}
\字{優}{어글어울 우}
\字{登}{ᄐᆞᆯ 등}
\字{仕}{벼ᄉᆞᆯ ᄉᆞ}
\字{攝}{자ᄇᆞᆯ 셥}
\字{職}{벼슬 직}
\字{從}{조ᄎᆞᆯ 죵}
\字{政}{졍ᄉᆞ 졍}
\字{存}{이실 존}
\字{以}{ᄡᅥ 이}
\字{甘}{ᄃᆞᆯ 감}
\字{棠}{아가외 당}
\字{去}{갈 거}
\字{而}{마리 이}
\字{益}{더을 익}
\字{詠}{이플 영}
\字{樂}{낙ᄒᆞᆯ 락}
\字{殊}{다ᄅᆞᆯ 슈}
\字{貴}{귀ᄒᆞᆯ 귀}
\字{賤}{쳔ᄒᆞᆯ 쳔}
\字{禮}{절 례}
\字{別}{다ᄅᆞᆯ 별}
\字{尊}{존ᄒᆞᆯ 존}
\字{卑}{ᄂᆞᄌᆞᆯ 비}
\字{上}{마ᄃᆡ 샹}
\字{和}{고ᄅᆞᆯ 화}
\字{下}{아래 하}
\字{睦}{고ᄅᆞᆯ 목}
\字{夫}{샤옹 부}
\字{唱}{브를 챵}
\字{婦}{며ᄂᆞ리 부}
\字{隨}{조ᄎᆞᆯ 슈}
\字{外}{밧 외}
\字{受}{ᄐᆞᆯ 슈}
\字{傅}{스승 부}
\字{訓}{ᄀᆞᄅᆞ칠 훈}
\字{入}{들 입}
\字{奉}{바ᄃᆞᆯ 봉}
\字{母}{어미 모}
\字{儀}{다ᄉᆞᆷ 의}
\字{諸}{모ᄃᆞᆯ 졔}
\字{姑}{할미 고}
\字{伯}{ᄆᆞᆮ ᄇᆡᆨ}
\字{叔}{아자비 슉}
\字{猶}{오힐 유}
\字{子}{아ᄃᆞ ᄌᆞ}
\字{比}{ᄀᆞᄌᆞᆯ 비}
\字{兒}{아ᄒᆡ ᄋᆞ}
\字{孔}{구무 공}
\字{懷}{훋ᄎᆞᆯ 회}
\字{兄}{ᄆᆞᆮ 형}
\字{弟}{아ᅀᆞ 뎨}
\字{同}{오힌 동}
\字{氣}{긔운 ᄭᅴ}
\字{連}{니을 련}
\字{枝}{가지 지}
\字{交}{사괼 교}
\字{友}{벋 우}
\字{投}{머드리 투}
\字{分}{ᄂᆞᆫ홀 분}
\字{切}{ᄀᆞᆫ졀 졀}
\字{磨}{ᄀᆞᆯ 마}
\字{箴}{빈혀 ᄌᆞᆷ}
\字{規}{여을 규}
\字{仁}{클 인}
\字{慈}{ᄌᆞ비 ᄌᆞ}
\字{隱}{그을 은}
\字{惻}{슬흘 측}
\字{造}{지을 조}
\字{次}{ᄀᆞᅀᆞᆷ ᄎᆞ}
\字{弗}{덜 블}
\字{離}{여흴 리}
\字{節}{ᄆᆞᄃᆡ 졀}
\字{義}{클 의}
\字{廉}{발 렴}
\字{退}{므늘 퇴}
\字{顚}{업더딜 뎐}
\字{沛}{젓바딜 패}
\字{匪}{이즐 비}
\字{虧}{이즐 휴}
\字{性}{셩 셩}
\字{靜}{괴오 졍}
\字{情}{ᄠᅳᆮ 졍}
\字{逸}{안일 일}
\字{心}{ᄆᆞᅀᆞᆷ 심}
\字{動}{뮐 동}
\字{神}{실령 신}
\字{疲}{시드러올 피}
\字{守}{딕힐 슈}
\字{眞}{ᄎᆞᆷ 진}
\字{志}{ᄠᅳᆮ 지}
\字{滿}{ᄎᆞᆯ 만}
\字{逐}{조ᄎᆞᆯ 튝}
\字{物}{갓 믈}
\字{意}{ᄠᅳᆮ 의}
\字{移}{옴길 이}
\字{堅}{구들 견}
\字{持}{디닏 ᄯᅵ}
\字{雅}{ᄆᆞᆯᄀᆞᆯ 아}
\字{操}{자ᄇᆞᆯ 조}
\字{好}{됴ᄒᆞᆯ 호}
\字{爵}{벼ᄉᆞ 쟉}
\字{自}{스스리 ᄌᆞ}
\字{縻}{얼글 미}
\字{都}{모ᄃᆞᆯ 도}
\字{邑}{고ᄋᆞᆯ 읍}
\字{華}{빈날 화}
\字{夏}{녀ᄅᆞᆷ 하}
\字{東}{동녁 동}
\字{西}{션녁 셔}
\字{二}{두 이}
\字{京}{셔울 경}
\字{背}{질 ᄇᆡ}
\字{邙}{터 망}
\字{面}{ᄂᆞᆫ 면}
\字{洛}{믓ᄀᆞᆺ 낙}
\字{浮}{ᄠᅳᆯ 부}
\字{渭}{믓ᄀᆞᆺ 위}
\字{據}{누를 거}
\字{涇}{믓ᄀᆞᆺ 경}
\字{宮}{집 궁}
\字{殿}{집 뎐}
\字{盤}{서릴 반}
\字{鬱}{덤ᄭᅥ울 울}
\字{樓}{릇 루}
\字{觀}{볼 관}
\字{飛}{ᄂᆞᆯ 비}
\字{驚}{놀날 경}
\字{圖}{그림 도}
\字{寫}{슬 샤}
\字{禽}{새 금}
\字{獸}{즘승 슈}
\字{畵}{그림 화}
\字{綵}{빗날 ᄎᆡ}
\字{仙}{션간 션}
\字{靈}{녕ᄒᆞᆯ 령}
\字{丙}{ᄆᆞᆮ 병}
\字{舍}{집 샤}
\字{傍}{겯 방}
\字{啓}{여틀 계}
\字{甲}{갑 갑}
\字{帳}{댱 댱}
\字{對}{샹ᄃᆡᆺ ᄃᆡ}
\字{楹}{딕누리 영}
\字{肆}{베플 ᄉᆞ}
\字{筵}{돗 연}
\字{設}{베플 셜}
\字{席}{돗 셕}
\字{鼓}{붑 고}
\字{瑟}{비화 슬}
\字{吹}{불 ᄎᆔ}
\字{笙}{뎌 ᄉᆡᆼ}
\字{陞}{되 승}
\字{階}{버텅 계}
\字{納}{드릴 납}
\字{陛}{버텅 폐}
\字{弁}{곡도 변}
\字{轉}{술위 뎐}
\字{疑}{의심 의}
\字{星}{별 셩}
\字{右}{올ᄒᆞᆯ 우}
\字{通}{ᄉᆞᄆᆞᄎᆞᆯ 통}
\字{廣}{너블 광}
\字{內}{안 ᄂᆡ}
\字{左}{욀 좌}
\字{達}{ᄉᆞᄆᆞᄎᆞᆯ 달}
\字{丞}{니을 승}
\字{明}{ᄇᆞᆯᄀᆞᆯ 명}
\字{旣}{이믜 긔}
\字{集}{모ᄃᆞᆯ 집}
\字{墳}{무덤 분}
\字{典}{법 뎐}
\字{亦}{ᄯᅩ 역}
\字{聚}{모ᄃᆞᆯ ᄎᆔ}
\字{群}{물 군}
\字{英}{곳부리 영}
\字{杜}{진ᄃᆞᆯ위 두}
\字{藁}{딥 고}
\字{鍾}{붑 죵}
\字{隷}{마치 예}
\字{漆}{옷칠 칠}
\字{書}{글월 셔}
\字{壁}{ᄇᆞᄅᆞᆷ 벽}
\字{經}{디날 경}
\字{府}{마을 부}
\字{羅}{쇠릉 라}
\字{將}{쟝슈 쟝}
\字{相}{서르 샹}
\字{路}{길 로}
\字{俠}{길 협}
\字{槐}{누튀 괴}
\字{卿}{벼슬 경}
\字{戶}{입 호}
\字{封}{봉ᄒᆞᆯ 봉}
\字{八}{여듧 팔}
\字{縣}{고을 현}
\字{家}{집 가}
\字{給}{줄 급}
\字{千}{즈믄 쳔}
\字{兵}{병맛 병}
\字{高}{노ᄑᆞᆯ 고}
\字{冠}{곳갈 관}
\字{陪}{뫼실 ᄇᆡ}
\字{輦}{술위 년}
\字{驅}{몰 구}
\字{轂}{술위 곡}
\字{振}{너틸 진}
\字{纓}{갇긴 영}
\字{世}{누릴 셰}
\字{祿}{녹 녹}
\字{侈}{샤치 치}
\字{富}{가ᅀᆞ멸 부}
\字{車}{술위 챠}
\字{駕}{멍에 가}
\字{肥}{ᄉᆞᆯ질 비}
\字{輕}{가ᄇᆡ야올 경}
\字{策}{무을 ᄎᆡᆨ}
\字{功}{공봇 공}
\字{茂}{덤거울 무}
\字{實}{염믈 실}
\字{勒}{굴에 륵}
\字{碑}{빗 비}
\字{刻}{사길 ᄀᆞᆨ}
\字{銘}{조올 명}
\字{磻}{돌 반}
\字{溪}{시내 계}
\字{伊}{소얌 이}
\字{尹}{ᄆᆞᆮ 윤}
\字{佐}{도올 좌}
\字{時}{ᄞᅵ니 시}
\字{阿}{ᄭᅳᆷ 아}
\字{衡}{저울 형}
\字{奄}{클 엄}
\字{宅}{집 ᄐᆡᆨ}
\字{曲}{고ᄇᆞᆯ 곡}
\字{阜}{두던 부}
\字{微}{마ᄎᆞᆯ 미}
\字{旦}{아ᄎᆞᆷ 단}
\字{孰}{누국 슉}
\字{營}{집 영}
\字{桓}{나모 환}
\字{公}{공정 공}
\字{匡}{광짓 광}
\字{合}{모ᄃᆞᆯ 합}
\字{濟}{거닐 졔}
\字{弱}{바ᄃᆞ라올 약}
\字{扶}{더위자블 부}
\字{傾}{기울 경}
\字{綺}{깁 긔}
\字{廻}{도로 회}
\字{漢}{하ᄂᆞᆯ 한}
\字{惠}{은혜 혜}
\字{說}{니를 셜}
\字{感}{깃글 감}
\字{武}{ᄆᆡ올 무}
\字{丁}{ᄉᆞᆫ 뎡}
\字{俊}{어딜 쥰}
\字{乂}{어딜 애}
\字{密}{볼 밀}
\字{勿}{말 믈}
\字{多}{할 다}
\字{士}{계ᄎᆞᆷ ᄉᆞ}
\字{寔}{클 식}
\字{寧}{안령 령}
\字{晋}{진국 진}
\字{楚}{초국 초}
\字{更}{가ᄉᆡᆯ ᄀᆡᆼ}
\字{覇}{사홈 패}
\字{趙}{됴국 됴}
\字{魏}{위국 위}
\字{困}{잇ᄲᅳᆯ 곤}
\字{橫}{비길 횡}
\字{假}{빌 가}
\字{途}{길 도}
\字{滅}{ᄢᅳᆯ 멸}
\字{虢}{나라 괵}
\字{踐}{ᄇᆞᆯ올 쳔}
\字{土}{ᄒᆞᆰ 토}
\字{會}{모ᄃᆞᆯ 회}
\字{盟}{ᄆᆡᆼ솃 ᄆᆡᆼ}
\字{何}{엇디 하}
\字{遵}{준ᄒᆞᆯ 준}
\字{約}{긔약 약}
\字{法}{법ᄒᆞᆯ 법}
\字{韓}{나라 한}
\字{弊}{폐ᄒᆞᆯ 폐}
\字{煩}{어즈러올 번}
\字{刑}{형벌 형}
\字{起}{닐 긔}
\字{翦}{ᄇᆞ릴 젼}
\字{頗}{ᄌᆞᄆᆞ 파}
\字{牧}{칠 믁}
\字{用}{ᄡᅥ 용}
\字{軍}{군 군}
\字{最}{안직 최}
\字{精}{솝 졍}
\字{宣}{님굼 션}
\字{威}{위엄 위}
\字{沙}{몰애 사}
\字{漠}{아ᄃᆞᆨᄒᆞᆯ 막}
\字{馳}{ᄃᆞᆯ일 티}
\字{譽}{소ᄅᆡ 예}
\字{丹}{블글 단}
\字{靑}{ᄑᆞᄅᆞᆯ 쳥}
\字{九}{아홉 구}
\字{州}{고을 ᄌᆔ}
\字{禹}{님금 우}
\字{跡}{자최 젹}
\字{百}{온 ᄇᆡᆨ}
\字{郡}{고을 군}
\字{秦}{나라 진}
\字{幷}{아올 병}
\字{嶽}{묏부리 악}
\字{宗}{ᄆᆞᄅᆞ 종}
\字{恒}{ᄒᆞᆼ샹 ᄒᆞᆼ}
\字{岱}{뫼 ᄃᆡ}
\字{禪}{션뎡 션}
\字{主}{님 쥬}
\字{云}{ᄀᆞᄅᆞᆯ 운}
\字{亭}{뎡ᄌᆞ 뎡}
\字{鴈}{그러기 안}
\字{門}{오래 문}
\字{紫}{블글 ᄌᆞ}
\字{塞}{ᄀᆞᆺ ᄉᆡᆨ}
\字{鷄}{ᄃᆞᆰ 계}
\字{田}{받 뎐}
\字{赤}{블글 젹}
\字{城}{잣 셩}
\字{昆}{ᄆᆞᆮ 곤}
\字{池}{못 디}
\字{碣}{돌 갈}
\字{石}{돌 셕}
\字{鉅}{톱 거}
\字{野}{뫼 야}
\字{洞}{골 동}
\字{庭}{ᄠᅳᆯ 뎡}
\字{曠}{ᄒᆡᆺ긔 광}
\字{遠}{멀 원}
\字{綿}{소옴 면}
\字{邈}{아닥ᄒᆞᆯ 막}
\字{巖}{바회 암}
\字{岫}{묏부리 슈}
\字{杳}{아ᄃᆞᆨᄒᆞᆯ 묘}
\字{冥}{아ᄃᆞᆨᄒᆞᆯ 명}
\字{治}{다ᄉᆞ릴 티}
\字{本}{믿 본}
\字{於}{늘 어}
\字{農}{녀늠지을 롱}
\字{務}{힘ᄡᅳᆯ 무}
\字{玆}{읻 ᄌᆞ}
\字{稼}{시믈 가}
\字{穡}{벼뷜 ᄉᆡᆨ}
\字{俶}{비르ᄉᆞᆯ 슉}
\字{載}{시르 ᄌᆡ}
\字{南}{앏 남}
\字{畝}{이랑 묘}
\字{我}{나 아}
\字{藝}{ᄌᆡ조 예}
\字{黍}{기장 셔}
\字{稷}{피 직}
\字{稅}{이삭 셰}
\字{熟}{니글 슉}
\字{貢}{바틸 공}
\字{新}{새 신}
\字{勸}{권ᄒᆞᆯ 권}
\字{賞}{샹ᄒᆞᆯ 샹}
\字{黜}{내조ᄎᆞᆯ 튤}
\字{陟}{올릴 텩}
\字{孟}{ᄆᆡ올 ᄆᆡᆼ}
\字{軻}{술위 가}
\字{敦}{도타올 돈}
\字{素}{ᄒᆡᆯ 소}
\字{史}{ᄉᆞ긔 ᄉᆞ}
\字{魚}{고기 어}
\字{秉}{자ᄇᆞᆯ 병}
\字{直}{고든 딕}
\字{庶}{물 셔}
\字{幾}{몃마 긔}
\字{中}{가온ᄃᆡ 듕}
\字{庸}{듕용 용}
\字{勞}{잇블 로}
\字{謙}{말ᄉᆞᆷ 겸}
\字{謹}{말ᄉᆞᆷ 근}
\字{勅}{저릴 틱}
\字{聆}{드릉 령}
\字{音}{소ᄅᆡ 음}
\字{察}{ᄉᆞᆯ필 찰}
\字{理}{고틸 리}
\字{鑑}{거우로 감}
\字{貌}{즛 모}
\字{辨}{ᄀᆞᆯᄒᆡᆯ 변}
\字{色}{빗 ᄉᆡᆨ}
\字{貽}{기틸 이}
\字{厥}{적 궐}
\字{嘉}{아ᄅᆞᆷ다올 가}
\字{猷}{ᄭᅬ 유}
\字{勉}{힘ᄡᅳᆯ 면}
\字{其}{적 기}
\字{祗}{오직 지}
\字{植}{시믈 식}
\字{省}{ᄉᆞᆯ필 ᄉᆡᆼ}
\字{躬}{몸 궁}
\字{譏}{우ᅀᅳᆯ 긔}
\字{誡}{브즈런ᄒᆞᆫ 계}
\字{寵}{괴일 툥}
\字{增}{더을 증}
\字{抗}{ᄀᆞ재 항}
\字{極}{ᄀᆞ재 극}
\字{殆}{바ᄃᆞ라올 ᄐᆡ}
\字{辱}{욕ᄒᆞᆯ 욕}
\字{近}{갓가올 근}
\字{恥}{븟그릴 티}
\字{林}{수플 림}
\字{睾}{두던 고}
\字{幸}{ᄒᆡᆼᄒᆞᆯ ᄒᆡᆼ}
\字{卽}{고 즉}
\字{兩}{두 냥}
\字{疎}{섯글 소}
\字{見}{볼 견}
\字{機}{틀 긔}
\字{解}{그르 ᄒᆡ}
\字{組}{인ᄭᅵᆫ 조}
\字{誰}{누굿 슈}
\字{逼}{버길 핍}
\字{索}{노 삭}
\字{居}{살 거}
\字{閑}{겨늘 한}
\字{處}{바라 쳐}
\字{沈}{ᄃᆞᄆᆞᆯ 팀}
\字{黙}{괴외 믁}
\字{寂}{괴외 젹}
\字{寥}{괴외 료}
\字{求}{구ᄒᆞᆯ 구}
\字{古}{녜 고}
\字{尋}{ᄎᆞᄌᆞᆯ 심}
\字{論}{말ᄉᆞᆷ 논}
\字{散}{흐틀 산}
\字{慮}{ᄉᆞ년 려}
\字{逍}{아ᅀᆞ라올 쇼}
\字{遙}{아ᅀᆞ라올 요}
\字{欣}{깃글 흔}
\字{奏}{ᄉᆞ올 주}
\字{累}{ᄠᆡ 류}
\字{遣}{보낼 견}
\字{戚}{슬흘 쳑}
\字{謝}{샤녯 샤}
\字{歡}{깃글 환}
\字{招}{브를 툐}
\字{渠}{걸 거}
\字{荷}{년 하}
\字{的}{마ᄌᆞᆯ 뎍}
\字{歷}{디날 력}
\字{園}{위원 원}
\字{莽}{ᄲᅡ일 망}
\字{抽}{ᄲᅡ일 듀}
\字{條}{올 됴}
\字{枇}{나모 피}
\字{杷}{나모 파}
\字{晩}{느즐 만}
\字{翠}{프늘 ᄎᆔ}
\字{梧}{머귀 오}
\字{桐}{머귀 동}
\字{早}{이를 조}
\字{彫}{ᄠᅳᆮ드를 됴}
\字{陳}{무글 딘}
\字{根}{불희 근}
\字{委}{ᄇᆞ릴 위}
\字{翳}{ᄀᆞ릴 예}
\字{落}{딜 락}
\字{葉}{닙 엽}
\字{飄}{나봇필 표}
\字{颻}{나봇필 요}
\字{遊}{노릴 유}
\字{鵾}{믓ᄃᆞᆰ 곤}
\字{獨}{홀을 독}
\字{運}{옴길 운}
\字{凌}{업쇼올 릉}
\字{摩}{ᄆᆞ릴 마}
\字{絳}{블글 강}
\字{霄}{하ᄂᆞᆯ 쇼}
\字{耽}{귀울 탐}
\字{讀}{닐글 독}
\字{翫}{샹원 완}
\字{市}{져제 시}
\字{寓}{브를 우}
\字{目}{눈 목}
\字{囊}{ᄂᆞᄆᆞᆺ 랑}
\字{箱}{샹ᄌᆞᆺ 샹}
\字{易}{밧ᄭᅩᆯ ᅀᅧᆨ}
\字{輶}{술위 유}
\字{攸}{배 유}
\字{畏}{저흘 외}
\字{屬}{브틀 쇽}
\字{耳}{귀 이}
\字{垣}{담 원}
\字{墻}{담 쟝}
\字{具}{ᄀᆞᄌᆞᆫ 구}
\字{膳}{션믈 션}
\字{飡}{반찬 찬}
\字{飯}{법 반}
\字{適}{마ᄌᆞᆯ 뎍}
\字{口}{ᅀᅵᆸ 구}
\字{充}{ᄎᆞᆯ 츙}
\字{腸}{ᅀᅢ 댱}
\字{飽}{ᄇᆡᄎᆞᆯ 포}
\字{飫}{ᄇᆡ촐 어}
\字{烹}{ᄉᆞᆯᄆᆞᆯ ᄑᆡᆼ}
\字{宰}{사ᄒᆞᆯ ᄌᆡ}
\字{飢}{주릴 긔}
\字{厭}{아쳘 염}
\字{糟}{ᄉᆞ라기 조}
\字{糠}{겨 걍}
\字{親}{어버이 친}
\字{戚}{아ᅀᆞᆷ 쳑}
\字{故}{주글 고}
\字{舊}{녜 구}
\字{老}{늘글 로}
\字{少}{아ᄒᆡ 쇼}
\字{異}{다를 이}
\字{糧}{약식 랑}
\字{妾}{곳갓 쳡}
\字{御}{님금 어}
\字{紡}{ᄭᅮ리 방}
\字{績}{ᄭᅮ리 젹}
\字{侍}{뫼실 시}
\字{巾}{뵈 건}
\字{帷}{댱 유}
\字{房}{구돌 방}
\字{紈}{깁 환}
\字{扇}{부체 션}
\字{圓}{두리 원}
\字{潔}{ᄎᆞᆯ 결}
\字{銀}{은 은}
\字{燭}{쵸 쵹}
\字{煒}{홰 휘}
\字{煌}{홰 황}
\字{晝}{낫 듀}
\字{眠}{조ᅀᆞᄅᆞᆷ 면}
\字{夕}{나죄 셕}
\字{寐}{잘 ᄆᆡ}
\字{藍}{족 남}
\字{筍}{대 슌}
\字{象}{고키리 샹}
\字{床}{나모 상}
\字{絃}{시울 현}
\字{歌}{놀애 가}
\字{酒}{술 쥬}
\字{讌}{잔ᄎᆡ 연}
\字{接}{브틀 졉}
\字{杯}{잔 ᄇᆡ}
\字{擧}{들 거}
\字{觴}{잔 샹}
\字{矯}{납ᄯᅡ올 교}
\字{手}{손 슈}
\字{頓}{조ᄋᆞᆯ 돈}
\字{足}{발 죡}
\字{悅}{깃글 열}
\字{豫}{미리 예}
\字{且}{ᄯᅩ 챠}
\字{康}{안강 강}
\字{嫡}{뎍실 뎍}
\字{後}{뒤 후}
\字{嗣}{니을 ᄉᆞ}
\字{續}{니을 쇽}
\字{祭}{이바ᄃᆞᆯ 졔}
\字{祀}{이바ᄃᆞᆯ ᄉᆞ}
\字{蒸}{ᄠᅵᆯ 증}
\字{嘗}{맛볼 샹}
\字{稽}{니마 계}
\字{顙}{니마 상}
\字{再}{노올 ᄌᆡ}
\字{拜}{절 ᄇᆡ}
\字{悚}{저흘 속}
\字{懼}{저흘 구}
\字{恐}{저흘 공}
\字{惶}{저흘 황}
\字{牋}{글월 젼}
\字{牒}{글월 텹}
\字{簡}{글월 간}
\字{要}{요강 요}
\字{顧}{도라볼 고}
\字{答}{ᄃᆡ답 답}
\字{審}{ᄉᆞᆯ필 심}
\字{詳}{ᄉᆞᆯ필 샹}
\字{骸}{ᄲᅧ ᄒᆡ}
\字{垢}{ᄠᆡ 구}
\字{想}{슷칠 샹}
\字{浴}{모욕 욕}
\字{執}{자ᄇᆞᆯ 집}
\字{熱}{더울 열}
\字{願}{원ᄒᆞᆯ 원}
\字{凉}{간다올 량}
\字{驢}{나괴 려}
\字{騾}{노새 로}
\字{犢}{쇼야지 독}
\字{特}{쇼 특}
\字{駭}{롤랄 ᄒᆡ}
\字{躍}{봉노을 약}
\字{超}{건널 툐}
\字{驤}{ᄀᆞᆯ욀 양}
\字{誅}{버힐 듀}
\字{斬}{버힐 참}
\字{賊}{도적 적}
\字{盜}{도적 도}
\字{捕}{자ᄇᆞᆯ 보}
\字{獲}{시를 획}
\字{叛}{ᄇᆡ반 반}
\字{亡}{주글 망}
\字{布}{뵈 포}
\字{射}{ᄡᅩᆯ 샤}
\字{遼}{료동 료}
\字{丸}{모작 환}
\字{嵆}{ᄒᆡ강 ᄒᆡ}
\字{琴}{거믄고 금}
\字{阮}{완젹 완}
\字{嘯}{ᄑᆞ람 쇼}
\字{恬}{알렴 렴}
\字{筆}{붇 필}
\字{倫}{물 륜}
\字{紙}{죠ᄒᆡ 지}
\字{鈞}{도관 균}
\字{巧}{공굣 교}
\字{任}{ᄀᆞᄋᆞᆷ 임}
\字{釣}{낙ᄭᅳᆯ 됴}
\字{釋}{그를 셕}
\字{紛}{어즈러울 분}
\字{利}{ᄂᆞᆯ카올 리}
\字{俗}{풍쇽 쇽}
\字{並}{다ᄆᆞᆺ 병}
\字{皆}{다 ᄀᆡ}
\字{佳}{됴ᄒᆞᆯ 가}
\字{妙}{미묫 묘}
\字{毛}{터럭 모}
\字{施}{베플 시}
\字{淑}{ᄆᆞᆯᄀᆞᆯ 슉}
\字{姿}{고올 ᄌᆞ}
\字{工}{바지 공}
\字{嚬}{ᄡᅵᆼ일 빙}
\字{姸}{나머글 연}
\字{笑}{우음 쇼}
\字{年}{ᄒᆡ 년}
\字{矢}{살 시}
\字{每}{니으 ᄆᆡ}
\字{催}{뵈알 최}
\字{羲}{ᄒᆡᆺ귀 희}
\字{暉}{ᄒᆡᆺ귀 휘}
\字{朗}{ᄆᆞᆯᄀᆞᆯ 랑}
\字{曜}{빗날 요}
\字{璇}{구슬 션}
\字{璣}{구슬 긔}
\字{懸}{ᄃᆞᆯ 현}
\字{幹}{읏듬 간}
\字{晦}{그믐 회}
\字{魄}{넉 ᄇᆡᆨ}
\字{環}{골ᄒᆡ 환}
\字{照}{ᄇᆞ일 죠}
\字{指}{손가락 지}
\字{薪}{섭 신}
\字{脩}{길 슈}
\字{祐}{도을 우}
\字{永}{긴 영}
\字{綏}{편ᄒᆞᆯ 유}
\字{吉}{멀 길}
\字{邵}{힘ᄡᅳᆯ 쇼}
\字{矩}{고ᄇᆞᆫ자 구}
\字{步}{거름 보}
\字{引}{혈 인}
\字{領}{목 령}
\字{俯}{구블 부}
\字{仰}{울월 앙}
\字{廊}{ᄒᆡᆼ낭 낭}
\字{廟}{종묘 묘}
\字{束}{뭇 속}
\字{帶}{ᄯᅴ ᄃᆡ}
\字{矜}{쟈랑 ᄭᅳᆼ}
\字{莊}{ᄭᅮ밀 쟝}
\字{徘}{머믈 ᄇᆡ}
\字{佪}{머믈 회}
\字{瞻}{볼 쳠}
\字{眺}{볼 도}
\字{孤}{외ᄅᆞ올 고}
\字{陋}{더러울 루}
\字{寡}{홀어비 과}
\字{聞}{드늘 문}
\字{愚}{어릴 우}
\字{蒙}{니블 몽}
\字{等}{ᄀᆞᆯ을 등}
\字{誚}{ᄭᅮ숑 쵸}
\字{謂}{니ᄂᆞᆯ 위}
\字{語}{말ᄉᆞᆷ 어}
\字{助}{도올 조}
\字{者}{놈 쟈}
\字{焉}{입겻 언}
\字{哉}{입겻 ᄌᆡ}
\字{乎}{온 호}
\字{也}{입겻 야}
\end{document}

%%% Local Variables:
%%% mode: latex 
%%% TeX-engine: luatex
%%% TeX-master: t 
%%% End:
