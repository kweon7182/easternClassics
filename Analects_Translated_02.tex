\documentclass{easternClassics}
\usepackage{kotex}

\setmainfont     {Noto Serif KR}[Scale=3.25,Script=Hangul]
\setmainhanjafont{Noto Serif JP}[Scale=3.25]
\setCharOption{main}{hangul}{xshift=0.02cm}
\setCharOption{comment}{}{scale=0.7,yscale=1.05}
\setCharOption{center} {}{scale=0.7}

\四周雙邊{0.05}{0.15}{0.2}
\半郭{23}{17.5}
\有界{0.05}
\半葉{19}{10}
\黑口{1/3}{4.5}{4.5}
\魚尾{trefoil}{trefoil}
\版心{1.33}
\版心字間{1}
\細註行間比{8/9} 
\版心題位置{0.8}
\張次位置{0.8}

\def\券#1#2{\文[parsing.assistIdeograph(3)]{[chapter|#1]#2}\文{[newColumn]}}
\def\篇#1  {\文[parsing.assistIdeograph(3)]{[newBlock]  #1}\文{[newColumn]}}
\def\原    {\文[parsing.assistIdeograph(3)]}
\def\諺#1  {\文{[indent|1]}\文[parsing.assistIdeograph(1)]{#1}\文{[indent|-1]}}

\begin{document}
\券{論語諺解卷之二}{론語어〯諺언〯解ᄒᆡ〯卷권〯之지二ᅀᅵ〯}
\篇{雍오ᇰ也야〯第뎨〯六륙〮}
\原{子ᄌᆞ〮ㅣ曰왈〮雍오ᇰ也야〯ᄂᆞᆫ可가〯使ᄉᆞ〯南남面면〯이로
  다}
\諺{子ᄌᆞ〮ㅣᄀᆞᆯᄋᆞ샤〮ᄃᆡ〮雍오ᇰ은〮可가〯히〮ᄒᆡ여〯곰〮南남
  面면〯ᄒᆞ〮얌즉〮ᄒᆞ〮두다〮}
\原{仲듀ᇰ〯弓구ᇰ이問문〯子ᄌᆞ〮桑사ᇰ伯ᄇᆡᆨ〮子ᄌᆞ〮ᄒᆞᆫ대子ᄌᆞ〮
  曰왈〮可가〯也야〯ㅣ簡간〯이니라}
\諺{仲듀ᇰ〯弓구ᇰ이子ᄌᆞ〮桑사ᇰ伯ᄇᆡᆨ〮子ᄌᆞ〮를〮묻〯ᄌᆞ〮온대〮
  子ᄌᆞ〮ㅣᄀᆞᆯᄋᆞ샤ᄃᆡ〮可가〯홈〯이簡간■■■}
\原{仲듀ᇰ〯弓구ᇰ이曰왈〮居거敬겨ᇰ〯而ᅀᅵ行ᄒᆡᇰ簡간〯ᄒᆞ야以
  이〯臨림其기民민이면不블〮亦역〮可가〯乎호ㅣ잇가居
  거簡간〯而ᅀᅵ行ᄒᆡᇰ簡간〯이면無무乃내〯大태〮簡간〯乎
  호ㅣ잇가}
\諺{仲듀ᇰ〯弓구ᇰ이〮ᄀᆞᆯ오〮ᄃᆡ〮敬겨ᇰ〯애〮居거ᄒᆞ고〮簡간〯을〮
  行ᅙᆡᇰᄒᆞ〮야〮ᄡᅥ〮그ᄇᆡᆨ〮셔ᇰ을〮臨림ᄒᆞ면〮ᄯᅩ〮ᄒᆞᆫ可가〯티〮
  아니〮ᄒᆞ니ᇰ〮잇〮가簡간〯애〮居거ᄒᆞ고〮簡간〯을〮行ᄒᆡᇰ
  ᄒᆞ면〮아니〮너무簡간〯ᄒᆞ니ᇰ〮잇가〮}
\原{子ᄌᆞ〮ㅣ曰왈〮雍오ᇰ之지言언이然ᅀᅧᆫᄒᆞ다}
\諺{子ᄌᆞ〮ㅣᄀᆞᆯᄋᆞ샤ᄃᆡ〮雍오ᇰ의〮말〯이〮그러〮ᄒᆞ다〮}
\原{○哀ᄋᆡ公고ᇰ이問문〯弟뎨〯子ᄌᆞ〮ㅣ孰슉〮爲위好호〯
  學ᄒᆞᆨ〮이니ᇰ잇고孔고ᇰ〮子ᄌᆞ〮ㅣ對ᄃᆡ〯曰왈〮有유〯顏안回회
  者쟈〮ㅣ好호〯學ᄒᆞᆨ〮ᄒᆞ야不블〮遷쳔怒노〯ᄒᆞ며不블〮貳ᅀᅵ〯
  過과〮ᄒᆞ더니不블〮幸ᄒᆡᇰ〯短단〯命며ᇰ〯死ᄉᆞ〯矣의〯라今금
  也야〯則즉〮亡무ᄒᆞ니未미〯聞문好호〯學ᄒᆞᆨ〮者쟈〮也야〯
  케ᇰ이다}
\諺{哀ᄋᆡ公고ᇰ이〮묻〯ᄌᆞ오ᄃᆡ〮弟뎨〯子ᄌᆞ〮ㅣ뉘〮學ᄒᆞᆨ〮을〮
  됴〯히〮너기〮ᄂᆞ니ᇰ〮잇고〮孔고ᇰ〮子ᄌᆞ〮ㅣ對ᄃᆡ〯ᄒᆞ〮야〮ᄀᆞᆯ
  ᄋᆞ샤〮ᄃᆡ〮顏안回회라〮ᄒᆞ리〮學ᄒᆞᆨ〮을됴〮히〮너겨〮怒
  노〯를〮遷쳔티〮아니〮ᄒᆞ며〮過과〮를〮貳ᅀᅵ〯티〮아니〮ᄒᆞ
  더니〮辛ᄒᆡᇰ〯티〮몯〯ᄒᆞ야〮命며ᇰ〯이〮短단〯ᄒᆞ〮야〮죽은〮디〮
  라〮이〮제ᄂᆞᆫ〮업〯스〮니〮學ᄒᆞᆨ〮을〮됴〯히〮너기〮ᄂᆞᆫ이〮를〮듣
  디〮몯〯게ᇰ이다〮}
\原{○子ᄌᆞ〮華화ㅣ使시〯於어齊제러니冉ᅀᅧᆷ〯子ᄌᆞ〮ㅣ爲
  위〮其기ᅀᅵᆫ모〯請쳐ᇰ〮粟속〮ᄒᆞᆫ대子ᄌᆞ〮ㅣ曰왈〮與여〯之지
  釜부〯ᄒᆞ라請쳐ᇰ〮益익〮ᄒᆞᆫ대曰왈〮與여〯之지庚유〯ᄒᆞ라ᄒᆞ야시ᄂᆞᆯ
  冉ᅀᅧᆷ〯子ᄌᆞ〮ㅣ與여〯之지粟속〮五오〯秉벼ᇰ〯ᄒᆞᆫ대}
\諺{子ᄌᆞ〮華화ㅣ齊졔예〮브리이〮더니〮冉ᅀᅧᆷ〯子ᄌᆞ〮ㅣ
  그어〮미를〮爲위〮ᄒᆞ〮야〮粟속〮을請쳐ᇰ〮ᄒᆞᆫ대〮子ᄌᆞ〮ㅣ
  ᄀᆞᆯᄋᆞ샤〮ᄃᆡ〮釜부〯를〮주라〮더홈〯을〮請쳐ᇰ〮ᄒᆞᆫ대〮ᄀᆞᆯᄋᆞ
  샤〮ᄃᆡ〮庚유〯를〮주라〮ᄒᆞ〮야〮시ᄂᆞᆯ冉ᅀᅧᆷ子ᄌᆞ〮ㅣ粟속〮
  다ᄉᆞᆺ〮秉벼ᇰ〯을〮준대〮}
\原{子ᄌᆞ〮ㅣ曰왈〮赤젹〮之지適뎍〮齊졔也야〯애乘스ᇰ肥
  비馬마〯ᄒᆞ며衣의〯輕겨ᇰ裘구ᄒᆞ니吾오ᄂᆞᆫ聞문之지也
  야〯호니君군子ᄌᆞ〮ᄂᆞᆫ周쥬急급〮이오不블〮繼계〯富부〯ㅣ라
  호라}
\諺{子ᄌᆞ〮ㅣᄀᆞᆯᄋᆞ샤〮ᄃᆡ〮赤젹〮의〮齊졔예〮갈제〮肥비■
  마〯ᄅᆞᆯ〮ᄐᆞ〮며輕겨ᇰ裘구ᄅᆞᆯ〮닙으〮니나〮ᄂᆞᆫ〮들오〮니君
  군子ᄌᆞ〮ᄂᆞᆫ〮急급〮ᄒᆞᆫ이〮를〮周쥬ᄒᆞ고〮富부〯ᄒᆞᆫ이〮ᄅᆞᆯ〮
  繼계〯티〮아니〮ᄒᆞᆫ〮다호〮라〮}
\原{原원思ᄉᆞㅣ爲위之지宰ᄌᆡ〯러니與여〯之지粟속〮九
  구〮百ᄇᆡᆨ〮이어시ᄂᆞᆯ辭ᄉᆞᄒᆞᆫ대}
\諺{原원思ᄉᆞㅣ宰ᄌᆡ〯되엿〮더니〮粟속〮九구〮百ᄇᆡᆨ〮을〮
  주〮어시ᄂᆞᆯ〮ᄉᆞ야ᇰ〯ᄒᆞᆫ대〮}
\原{子ᄌᆞ〮ㅣ曰왈〮毋무ᄒᆞ야以ᅀᅵ〯與여〯爾ᅀᅵ〯隣린里리〯鄕
  햐ᇰ黨다ᇰ〮乎호ᅟᅵᆫ뎌}
\諺{子ᄌᆞ〮ㅣᄀᆞᆯᄋᆞ샤〮ᄃᆡ〮말아〯ᄡᅥ〮네의〮隣린이〮며〮里리〯
  며〮鄕햐ᇰ이〮며〮黨다ᇰ〮을〮줄띤〮뎌}
\原{○子ᄌᆞ〮ㅣ謂위〮仲듀ᇰ〯弓구ᇰ曰왈〮犁리牛우之지子
  ᄌᆞ〮ㅣ騂셔ᇰ且챠〯角각〮이면雖슈欲욕〮勿믈〮用요ᇰ〯이나山
  산川쳔ᄋᆞᆫ其기舍샤〯諸져ㅣ아}
\諺{子ᄌᆞ〮ㅣ仲듀ᇰ〯弓구ᇰ을〮닐어〮ᄀᆞᆯᄋᆞ샤〮ᄃᆡ〮犁리牛우
  의〮子ᄌᆞ〮ㅣ騂셔ᇰᄒᆞ고〮ᄯᅩ〮角각〮ᄒᆞ면〮비록〮ᄡᅳ〮디말〯
  고쟈〮ᄒᆞ나〮山산川쳔은〮그ᄇᆞ리〮랴〮}
\原{○子ᄌᆞ〮ㅣ曰왈〮回회也야〯ᄂᆞᆫ其기心심이三삼月
  ᄋᅠ〮ᅟힵᆯ不블〮違위仁ᅀᅵᆫ이오其기餘여則즉〮日ᅀᅵᆯ〮月월〮至
  지〮焉언而ᅀᅵ已이〯矣의〯니라}
\諺{子ᄌᆞ〮ㅣᄀᆞᆯᄋᆞ샤〮ᄃᆡ〮回회ᄂᆞᆫ〮그ᄆᆞᄋᆞᆷ이〮석〯ᄃᆞᆯ〮을〮仁
  ᅀᅵᆫ에〮어글〮웃디〮아니〮ᄒᆞ고〮그나ᄆᆞᆫ〮이〮ᄂᆞᆫ〮날〮이며〮
  ᄃᆞᆯ〮로니를〮ᄯᆞᄅᆞᆷ이〮니라〮}
\原{○季계〯康가ᇰ子ᄌᆞ〮ㅣ問문〯仲듀ᇰ〯由유ᄂᆞᆫ可가〯使ᄉᆞ〯
  從죠ᇰ政져ᇰ〮也야〯與여잇■子ᄌᆞ〮ㅣ曰왈〮由유也야〯ᄂᆞᆫ
  果과〯ᄒᆞ니於어從죠ᇰ政져ᇰ〮乎호〮애何하有유〯ㅣ리오曰
  왈〮賜ᄉᆞ〯也야〯ᄂᆞᆫ可가〯使ᄉᆞ〯從죠ᇰ政져ᇰ〮也야〯與여잇가
  曰왈〮賜ᄉᆞ〯也야〯ᄂᆞᆫ達달〮ᄒᆞ니於어從죠ᇰ政져ᇰ〮乎호애
  何하有유〯ㅣ리오曰왈〮求구也야〯ᄂᆞᆫ可가〯使ᄉᆞ〯從죠ᇰ
  政져ᇰ〮也야〯與여잇가曰왈〮求구也야〯ᄂᆞᆫ藝예〯ᄒᆞ니於어
  從죠ᇰ政져ᇰ乎호애何하有유〯ㅣ리오}
\諺{季계〯康가ᇰ子ᄌᆞ〮ㅣ묻〯ᄌᆞ〮오ᄃᆡ〮仲듀ᇰ〯由유ᄂᆞᆫ〮可가〯
  히〮ᄒᆡ여〯곰〮政져ᇰ〮을〮從죠ᇰᄒᆞ〮얌즉〮ᄒᆞ니ᇰ〮잇가〮子ᄌᆞ〮
  ㅣᄀᆞᆯᄋᆞ샤〮ᄃᆡ〮由유ᄂᆞᆫ〮果과〯ᄒᆞ니〮政져ᇰ을〮從죠ᇰ홈〯
  애〮므스〮거시〮이시리〮오〮ᄀᆞᆯ오〮ᄃᆡ〮賜ᄉᆞ〯ᄂᆞᆫ〮可가〯히〮
  ᄒᆡ여〯곰〮政져ᇰ〮을〮從죠ᇰᄒᆞ〮얌〮즉〮ᄒᆞ니ᇰ〮잇〮가〮ᄀᆞᆯᄋᆞ샤〮
  ᄃᆡ〮賜ᄉᆞ〯ᄂᆞᆫ〮達달〮ᄒᆞ니〮政져ᇰ〮을〮從죠ᇰ홈〯애〮므스〮거
  시〮이시리〮오〮ᄀᆞᆯ오〮ᄃᆡ〮求구ᄂᆞᆫ〮可가〯히〮ᄒᆡ여〯곰〮政
  져ᇰ〮을〮從죠ᇰᄒᆞ〮얌즉〮ᄒᆞ니ᇰ〮잇〮가〮ᄀᆞᆯᄋᆞ샤〮ᄃᆡ〮求구ᄂᆞᆫ〮
  藝예〯ᄒᆞ니〮政져ᇰ〮을〮從죠ᇰ홈〯애〮므스〮거시〮이시리〮
  오〮}
\原{○季계〯氏시〮ㅣ使ᄉᆞ〯閔민〮子ᄌᆞ〮騫건으로爲위費비〯
  宰ᄌᆡ〯ᄒᆞᆫ대閔민〮子ᄌᆞ〮騫건이曰왈〮善션〯爲위〮我아〯辭
  ᄉᆞ焉언ᄒᆞ라如ᅀᅧ有유〯復부〯我아〯者자〯딘ᅟᅢᆫ則즉〮吾오
  ㅣ必필〮在ᄌᆡ〯汶문〯上샤ᇰ〯矣의〯로리라}
\諺{季계〯氏시〮ㅣ閔민〮子ᄌᆞ〮騫건으〮로ᄒᆡ여〮곰〮費비〯
  ㅅ宰ᄌᆡ〯를〮ᄒᆞ인〮대〮閔민〮子ᄌᆞ〮騫건이〮ᄀᆞᆯ오〮ᄃᆡ〮善
  션〯히〮나〯ᄅᆞᆯ爲위〮ᄒᆞ〮야〮辭ᄉᆞᄒᆞ라〮만〯일〮에〮내게다
  시〮홈〯이〮이실띤〮댄〮곧〮내〮반〮ᄃᆞ〮시汶문〯人上샤ᇰ〯애〮
  이쇼〮리라〮}
\原{○伯ᄇᆡᆨ〮牛우ㅣ有유〯疾질〮이어ᄂᆞᆯ子ᄌᆞ〮ㅣ問문〯之지
  ᄒᆞ실ᄉᆡ自ᄌᆞ〮牗유〯로執집〮其기手슈〯曰왈〮亡무之지
  러니命며ᇰ〯矣의〯夫부ㅣ라斯ᄉᆞ人ᅀᅵᆫ也야〯ㅣ而시有유〯
  斯ᄉᆞ疾질〮也야〯ᄒᆞᆯ셔斯ᄉᆞ人ᅀᅵᆫ也야〯ㅣ而ᅀᅵ有유〯斯
  ᄉᆞ疾질〮也야〯ᄒᆞᆯ셔}
\諺{伯ᄇᆡᆨ〮牛우ㅣ疾질〮이잇거ᄂᆞᆯ〮子ᄌᆞ〮ㅣ무ᄅᆞ〮실ᄉᅠ〮ᅟퟅ
  牗유〯로〮브터〮그손〮을〮잡아〮ᄀᆞᆯᄋᆞ샤〮ᄃᆡ〮업〯스〮리〮러
  니〮命며ᇰ〯이〮라〮이〮사〯ᄅᆞᆷ〮이이〮疾질〮을〮둘셔〮이〮사〯ᄅᆞᆷ〮
  이이〮疾질〮을〮둘셔〮}
\原{○子ᄌᆞ〮ㅣ曰왈〮賢현哉ᄌᆡ라回회也야〯ㅣ여一일〮簞
  단食ᄉᆞ〯와一일〮瓢표飮음〯으로在ᄌᆡ〯陋루〯卷하ᇰ〯을人
  ᅀᅵᆫ不블〮堪감其기憂우ㅣ어늘回회也야〯ㅣ不불〮改
  ᄀᆡ〯其기樂락〮ᄒᆞ니賢현哉ᄌᆡ라回회也야〯ㅣ여}
\諺{子ᄌᆞ〮ㅣᄀᆞᆯᄋᆞ샤〮ᄃᆡ〮賢현ᄒᆞ다〮回회ㅣ여〮ᄒᆞᆫ簞단
  앳〮食ᄉᆞ〯와〮ᄒᆞᆫ瓢표앳〮飮음〯으〮로陋루〯卷하ᇰ〯애〮이
  심을〮사〯ᄅᆞᆷ〮이〮그시름〯을〮이긔〮디몯〯ᄒᆞ〮거늘〮回회
  ㅣ그樂락〮을〮改ᄀᆡ〯티〮아니〮ᄒᆞ니〮賢현ᄒᆞ다〮回회
  ㅣ여〮}
\原{○冉ᅀᅧᆷ〯求구ㅣ曰왈〮非비不블〮說열〮子ᄌᆞ〮之지道
  도〯ㅣ언마ᄂᆞᆫ力력〮不블〮足죡〮也야〯ㅣ로이다子ᄌᆞ〮ㅣ曰왈〮力
  력〮不블〮足죡〮者쟈〮ᄂᆞᆫ中듀ᇰ道도〯而ᅀᅵ廢폐〯ᄒᆞᄂᆞ니今
  금女ᅀᅧ〯ᄂᆞᆫ畵획〮이로다}
\諺{冉ᅀᅧᆷ〯求구ㅣᄀᆞᆯ오〮ᄃᆡ〮子ᄌᆞ〮의〮道도를〮說열〮티아
  니〮홈〯이〮아니〮언마ᄂᆞᆫ〮힘〮이〮足죡〮디〮몯〯호〯이〮다〮子
  ᄌᆞ〮ㅣᄀᆞᆯᄋᆞ샤〮ᄃᆡ〮힘〮이〮足죡〮디〮몯〯ᄒᆞᆫ者쟈〮ᄂᆞᆫ〮道도〯
  애〮中듀ᇰᄒᆞ〮야〮廢폐〯ᄒᆞ〮ᄂᆞ〮니이〮제너ᄂᆞᆫ〮畵획〮홈〯이〮
  로다}
\原{○子ᄌᆞ〮ㅣ謂위〮子ᄌᆞ〮夏하〯曰왈女ᅀᅧ〯爲위君군子
  ᄌᆞ〮儒유ㅣ오無무爲위小쇼〯人ᅀᅵᆫ儒유ᄒᆞ라}
\諺{子ᄌᆞ〮ㅣ子ᄌᆞ〮夏하〯ᄃᆞ려〮닐어〮ᄀᆞᆯᄋᆞ샤〮ᄃᆡ〮네〯君군
  子ᄌᆞ〮人儒유ㅣ되고〮小쇼〯人ᅀᅵᆫ人儒유ㅣ되디〮
  말〯라〮}
\原{○子ᄌᆞ〮游유ㅣ爲위武무〯城셔ᇰ宰ᄌᆡ〯러니子ᄌᆞ〮ㅣ曰
  왈〮女ᅀᅧ〯ㅣ得득〮人ᅀᅵᆫ焉언爾ᅀᅵ〯乎호아曰왈〮有유〯
  澹담臺ᄃᆡ滅멸〮明며ᇰ者쟈〮ᄒᆞ니行ᄒᆡᇰ不블〮由유徑겨ᇰ〮
  ᄒᆞ며非비公고ᇰ事ᄉᆞ〯ㅣ어든未미〯嘗샤ᇰ至지〮於어偃언〯
  之지室실也야〯ᄒᆞᄂᆞ니ᇰ이다}
\諺{子ᄌᆞ〮游유ㅣ武무〯城셔ᇰ人宰ᄌᆡ〯되엿〯더니〮子ᄌᆞ〮
  ㅣᄀᆞᆯᄋᆞ샤〮ᄃᆡ〮네〯샤〯ᄅᆞᆷ〮을〮어〯던〮ᄂᆞᆫ다〮ᄀᆞᆯ오〮ᄃᆡ〮澹담
  臺ᄃᆡ滅멸〮明며ᇰ이〮라〮ᄒᆞᆯ이〮이시니〮行ᄒᆡᇰ홈〯애〮徑
  겨ᇰ〮을〮말ᄆᆡ〮암〯디〮아니〮ᄒᆞ며〮公고ᇰ事ᄉᆞ〯ㅣ아니〮어
  든〮일쯕〮偃언〯의〮室실〮에〮니르〮디〮아니〮ᄒᆞ〮ᄂᆞ〮니ᇰ이
  다〮}
\原{○子ᄌᆞ〮ㅣ曰왈〮孟ᄆᆡᇰ〯之지反반〯ᄋᆞᆫ不블〮伐벌〮이로다
  奔분而ᅀᅵ殿뎐〯ᄒᆞ야將쟈ᇰ入입〮門문ᄒᆞᆯᄉᆡ策ᄎᆡᆨ〮其기馬
  마〯曰왈〮非비敢감〯後후〯也야〯ㅣ라馬마〯不블〮進진〯也
  야〯ㅣ라ᄒᆞ니라}
\諺{子ᄌᆞ〮ㅣᄀᆞᆯᄋᆞ샤〮ᄃᆡ〮孟ᄆᆡᇰ〯之지反반〯ᄋᆞᆫ〮伐벌〮티〮아
  니〮ᄒᆞ〮놋다〮奔분홈〯애〮殿뎐〯ᄒᆞ〮야〮將쟈ᇰᄎᆞᆺ門문의〮
  들〮ᄉᆡ그ᄆᆞᆯ을〮策ᄎᆡᆨᄒᆞ〮야〯ᄀᆞᆯ오〮ᄃᆡ〮敢감〯히〮後후〯ᄒᆞ〮
  ᄂᆞᆫ〮줄〮이〮아니〮라ᄆᆞᆯ이〮나ᅀᅡ〮가디〮안이홈〯이〮라ᄒᆞ
  니〮라}
\原{○子ᄌᆞ〮ㅣ曰왈〮不블〮有유〯祝츅〮鮀타之지佞녀ᇰ〮이며
  而ᅀᅵ有유〯宋소ᇰ〯朝됴之지美미〯면難난乎호免면〯
  於어今금之지世셰〯矣의〯니라}
\諺{子ᄌᆞ〮ㅣᄀᆞᆯᄋᆞ샤〮ᄃᆡ〮祝츅〮鮀타의〮佞녀ᇰ〮을〮두며〮宋
  소ᇰ〯朝됴의〮美미〯를〮두디〮아니〮면〮이〮젯〮世셰〯예〮免
  면〯홈〯이〮어려〮우니〮라}
\原{○子ᄌᆞ〮ㅣ曰왈〮誰슈能느ᇰ出츌〮不블〮由유戶호〯ㅣ리
  오마ᄂᆞᆫ何하莫막〮由유斯ᄉᆞ道도〯也야〯오}
\諺{子ᄌᆞ〮ㅣᄀᆞᆯᄋᆞ샤〮ᄃᆡ〮뉘〮能느ᇰ히〮出츌〮홈〯애〮戶호〯를〮
  由유티〮아니〮리오〮마ᄂᆞᆫ〮엇〯디〮이〮道도〯를〮由유티〮
  아니〮ᄒᆞ〮ᄂᆞᆫ〮고〮}
\原{○子ᄌᆞ〮ㅣ曰왈〮質질〮勝스ᇰ〯文문則즉〮野야〯ㅣ오文문
  勝스ᇰ〯質질〮則즉〮史ᄉᆞ〯ㅣ니文문質질〮이彬빈彬빈然
  ᅀᅧᆫ後후〯에君군子ᄌᆞ〮ㅣ니라}
\諺{子ᄌᆞ〮ㅣᄀᆞᆯᄋᆞ샤〮ᄃᆡ〮質질〮이文문을〮勝스ᇰ〯ᄒᆞ면〮野
  야〯ㅣ오文문이質질〮을勝스ᇰ〯ᄒᆞ면〮史ᄉᆞ〯ㅣ니〮文
  문과〮質질〮이彬빈彬빈ᄒᆞᆫ後후〯에〮君군子ᄌᆞ〮ㅣ
  니라〮}
\原{○子ᄌᆞ〮ㅣ曰왈〮人ᅀᅵᆫ之지生ᄉᆡᇰ也야〯ㅣ直딕〮ᄒᆞ니罔
  마ᇰ〮之지生ᄉᆡᇰ也야〯ᄂᆞᆫ幸ᄒᆡᇰ〯而ᅀᅵ免면〯이니라}
\諺{子ᄌᆞ〮ㅣᄀᆞᆯᄋᆞ샤〮ᄃᆡ〮사〯ᄅᆞᆷ〮의〮生ᄉᆡᇰ이〮直딕〮ᄒᆞᆫ거시〮
  니罔마ᇰ〮의生ᄉᆡᇰ홈〯은〮히ᇰ〯혀〮免면〯ᄒᆞ〮얀〮ᄂᆞ〮니라〮}
\原{○子ᄌᆞ〮ㅣ曰왈〮知디之지者쟈〮ㅣ不블〮如ᅀᅧ好호〯
  之지者쟈〮ㅣ오好호〯之지者쟈〮ㅣ不블〮如ᅀᅧ樂락〮之
  지者쟈〮ㅣ니라}
\諺{子ᄌᆞ〮ㅣᄀᆞᆯᄋᆞ샤〮ᄃᆡ〮아〯ᄂᆞᆫ〮이됴〯히너기〮ᄂᆞᆫ이〮만ᄀᆞᆮ〮
  디몯〯ᄒᆞ고〮됴〯히너기〮ᄂᆞᆫ이〮즐〮겨ᄒᆞ〮ᄂᆞᆫ이〮만곤〮디
  몯〯ᄒᆞ니〮라}
\原{○子ᄌᆞ〮ㅣ曰왈〮中듀ᇰ人ᅀᅵᆫ以이〯上샤ᇰ〯ᄋᆞᆫ可가〯以이〯
  語어〯上샤ᇰ〯也야〯ㅣ어니와中듀ᇰ人ᅀᅵᆫ以이〯下하〯ᄂᆞᆫ不블〮
  可가〯以이〯語어〯上샤ᇰ〯也야〯ㅣ니라}
\諺{子ᄌᆞ〮ㅣᄀᆞᆯᄋᆞ샤〮ᄃᆡ〮中듀ᇰ人ᅀᅵᆫ으〮로ᄡᅥ〮우흔〮可가〯
  히〮ᄡᅥ우흘〮니ᄅᆞ려〮니와〮中듀ᇰ人ᅀᅵᆫ으〮로ᄡᅥ〮아래〯
  ᄂᆞᆫ〮可가〯히〮ᄡᅥ〮우흘〮니ᄅᆞ디〮몯〯ᄒᆞᆯ꺼시〮니라〮}
\原{○樊번遲디ㅣ問문〯知디〮ᄒᆞᆫ대子ᄌᆞ〮ㅣ曰왈〮務무〯民
  민之지義의〯오敬겨ᇰ〯鬼귀〯神신而ᅀᅵ遠원〮之지면
  可가〯謂위〮知디〮矣의〯니라問문〯仁ᅀᅵᆫᄒᆞᆫ대曰왈〮仁ᅀᅵᆫ者
  쟈〮ㅣ先션難난而ᅀᅵ後후〯獲획〮이면可가〯謂위〮仁ᅀᅵᆫ
  矣의〯니라}
\諺{樊번遲디ㅣ知디〮를〮묻〯ᄌᆞ온대〮子ᄌᆞ〮ㅣᄀᆞᆯᄋᆞ샤〮
  ᄃᆡ〮民민의義의〯를〮힘〮ᄡᅳ〮고鬼귀〯神신을〮고ᇰ겨ᇰ〯코
  멀리〮ᄒᆞ면〮可가〯히〮知디〮라닐ᄋᆞᆯ띠〮니라〮仁ᅀᅵᆫ을〮
  묻〯ᄌᆞ온대〮ᄀᆞᆯᄋᆞ샤〮ᄃᆡ〮仁ᅀᅵᆫᄒᆞᆫ者쟈〮ㅣ難난을〮몬
  져ᄒᆞ고〮獲획〮홈〯을〮後후〯ᄒᆞ면〮可가〯히〮仁ᅀᅵᆫ이라〮
  니ᄅᆞᆯ띠〮니라〮}
\原{○子ᄌᆞ〮ㅣ曰왈〮知디〮者쟈〮ᄂᆞᆫ樂요〯水슈〮ᄒᆞ고仁ᅀᅵᆫ者
  쟈〮ᄂᆞᆫ樂요〯山산이니知디〮者쟈〮ᄂᆞᆫ動도ᇰ〯ᄒᆞ고仁ᅀᅵᆫ者쟈〮
  ᄂᆞᆫ靜져ᇰ〯ᄒᆞ며知디〮者쟈〯ᄂᆞᆫ樂락〮ᄒᆞ고仁ᅀᅵᆫ者쟈〮ᄂᆞᆫ壽슈〮
  ㅣ니라}
\諺{子ᄌᆞ〮ㅣᄀᆞᆯᄋᆞ샤〮ᄃᆡ〮知디〮ᄒᆞᆫ者쟈〮ᄂᆞᆫ〮水슈〮를〮됴〯히〮
  너기〮고仁ᅀᅵᆫᄒᆞᆫ者쟈〮ᄂᆞᆫ〮山산을〮됴〯히너기〮ᄂᆞ니〮
  知디〮ᄒᆞᆫ者자〮ᄂᆞᆫ〮動도ᇰ〯ᄒᆞ고仁ᅀᅵᆫᄒᆞᆫ者쟈〮ᄂᆞᆫ〮靜져ᇰ〯
  ᄒᆞ며〮知디〮ᄒᆞᆫ者쟈〮ᄂᆞᆫ〮樂락〮ᄒᆞ고〮仁ᅀᅵᆫᄒᆞᆫ者쟈〮ᄂᆞᆫ〮
  壽슈〮ᄒᆞ〮ᄂᆞ〮니라〮}
\原{○子ᄌᆞ〮ㅣ曰왈〮齊졔一일〮變변〯이면至지〮於어魯로〮
  ᄒᆞ고魯로〮一일〮變변〯이면至지〮於어道도〯ㅣ니라}
\諺{子ᄌᆞ〮ㅣᄀᆞᆯᄋᆞ샤〮ᄃᆡ〮齊졔ㅣᄒᆞᆫ번變변〯ᄒᆞ면〮魯로〮
  애니르고〮魯로〮ㅣᄒᆞᆫ번變변〯ᄒᆞ면〮道도〯애〮니를〮
  띠〮니라〮}
\原{○子ᄌᆞ〮ㅣ曰왈〮觚고ㅣ不블〮觚고ㅣ면觚고哉ᄌᆡ觚
  고哉[?|ᄌퟅ]아}
\諺{子ᄌᆞ〮ㅣᄀᆞᆯᄋᆞ샤〮ᄃᆡ〮觚고ㅣ觚고티〮아니〮면〮觚고
  ㅣ랴〮觚고ㅣ랴〮}
\原{○宰ᄌᆡ〯我아〯ㅣ問문〮曰왈〮仁ᅀᅵᆫ者쟈〮ᄂᆞᆫ雖슈告고〯
  之지曰왈〮井져ᇰ〯有유〮仁ᅀᅵᆫ焉언이라도其기從죠ᇰ之
  지也야〯ㅣ로소이다子ᄌᆞㅣ曰왈〮何하爲위其기然ᅀᅧᆫ
  也야〯ㅣ리오君군子ᄌᆞ〮ᄂᆞᆫ可가〯逝셔〯也야〯ㅣ언뎌ᇰ不블〮
  可가〯陷함〯也야〯ㅣ며可가〯欺긔也야〯ㅣ언뎌ᇰ不블〮可가〯
  罔마ᇰ〮也야〯ㅣ니라}
\諺{宰ᄌᆡ〯我아〯ㅣ묻〯ᄌᆞ와ᄀᆞᆯ오〮ᄃᆡ仁ᅀᅵᆫ者쟈〮ᄂᆞᆫ〮비륵〮
  告고〯ᄒᆞ〮야ᄀᆞᆯ오〮ᄃᆡ〮井져ᇰ〮에사〮ᄅᆞᆷ〮이〮잇다〮ᄒᆞ〮야도
  그從죠ᇰᄒᆞ리〮로소ᇰ이다〮子ᄌᆞ〮ㅣᄀᆞᆯᄋᆞ샤〮ᄃᆡ〮엇〯디
  그그러〮ᄒᆞ리〮오〮君군子ᄌᆞ〮ᄂᆞᆫ〮可가〯히가게〮ᄒᆞᆯ띠〮
  언뎌ᇰ〮可가〯히ᄲᅡ〮디게〮몯〯ᄒᆞ며〮可가〯히欺긔ᄒᆞᆯ띠〮
  언뎌ᇰ〮可가〯히罔마ᇰ〮티몯〯ᄒᆞᆯ꺼시〮니라〮}
\原{○子ᄌᆞ〯ㅣ曰왈〮君군子ᄌᆞ〮ㅣ博박〮學ᄒᆞᆨ〮於어文문
  이오約약〮之지以이〯禮례〮면亦역〮可가〯以이〯弗블〮畔
  ■〯矣의〯夫부ᅟᅵᆫ뎌}
\諺{子ᄌᆞ〮ㅣᄀᆞᆯᄋᆞ샤〮ᄃᆡ〮君군子ᄌᆞ〮ㅣ文문에〮너비〮學
  ᄒᆞᆨ〮ᄒᆞ고〮約약〮호〯ᄃᆡ〮禮례〮로ᄡᅥ〮ᄒᆞ면〮ᄯᅩ〮可가〯히ᄡᅥ〮
  畔반〯티〮아니〮ᄒᆞ린〮뎌}
\原{○子ᄌᆞ〮ㅣ見견〯南남子ᄌᆞ〮ᄒᆞ신대子ᄌᆞ〮路로〮ㅣ不블〮
  說열〮이어ᄂᆞᆯ夫부子ᄌᆞ〮ㅣ矢시〯之지曰왈〮予여所소〯
  否부〯者쟈〮딘ᅟᅢᆫ天텬厭염〯之지天텬厭염〯之지시리라}
\諺{子ᄌᆞ〮ㅣ南남子ᄌᆞ〮를〮보〮신대〮子ᄌᆞ〮路로〮ㅣ깃거〮
  티〮아니〮ᄒᆞ〮거ᄂᆞᆯ〮夫부子ᄌᆞ〮ㅣ矢시〯ᄒᆞ〮야〮ᄀᆞᆯᄋᆞ샤〮
  ᄃᆡ〮내〮否부〯ᄒᆞᆫ밴〮댄〮하ᄂᆞᆯ〮히厭염〯ᄒᆞ〮시〮리라〮하ᄂᆞᆯ〮
  히厭염〯ᄒᆞ〮시〮리라〮}
\原{○子ᄌᆞ〮ㅣ曰왈〮中듀ᇰ庸요ᇰ之지爲위德덕〮也야〯ㅣ
  其기至지〮矣의〯乎호ᅟᅵᆫ뎌民민鮮션〯이久구〯矣의〯니라}
\諺{子ᄌᆞ〮ㅣᄀᆞᆯᄋᆞ샤〮ᄃᆡ〮中듀ᇰ庸요ᇰ의德덕〮이로옴이〮
  그至지〮ᄒᆞᆫ뎌〮民민이〮鮮션〯컨디〮오라〮니라〮}
\原{○子ᄌᆞ〮貢고ᇰ〯이曰왈〮如ᅀᅧ有유〯博박〮施시〯於어民
  민而ᅀᅵ能느ᇰ濟졔〯衆쥬ᇰ〯혼ᄃᆡᆫ何하如ᅀᅧᄒᆞ니ᇰ잇고可가〯謂
  위〮仁ᅀᅵᆫ乎호ㅣ잇가子ᄌᆞ〮ㅣ曰왈〮何하事ᄉᆞ〯於어仁
  ᅀᅵᆫ이리오必필〮也야〯聖셔ᇰ〯乎호ᅟᅵᆫ뎌堯요舜슌〯도其기
  猶유病벼ᇰ〯諸져ㅣ시니라}
\諺{子ᄌᆞ〮貢고ᇰ〯이ᄀᆞᆯ오〮ᄃᆡ〮만〯일〮에〮民민의〮게〮施시〯홈〮
  을〮너비〮ᄒᆞ고〮能느ᇰ히〮濟졔〯홈〯이〮衆쥬ᇰ〯혼〯ᄃᆡᆫ〮엇〯더
  ᄒᆞ니ᇰ잇고〮可가〯히〮仁ᅀᅵᆫ이〮라〮니ᄅᆞ리ᇰ〮잇가〮子ᄌᆞ〮
  ㅣᄀᆞᆯᄋᆞ샤〮ᄃᆡ〮엇〯디仁ᅀᅵᆫ에〮事ᄉᆞ〯ᄒᆞ리〮오반〮ᄃᆞ〮시
  聖셔ᇰ〯인뎌〮堯요舜슌〯도그오〮히〮려病벼ᇰ〯또이〮너
  기〮시〮니라〮}
\原{夫부仁ᅀᅵᆫ者쟈〮ᄂᆞᆫ已긔〮欲욕〮立립〮而ᅀᅵ立립〮人ᅀᅵᆫ
  ᄒᆞ며已긔〮欲욕〮達달〮而ᅀᅵ達달〮人ᅀᅵᆫ이니라}
\諺{仁ᅀᅵᆫᄒᆞᆫ者쟈〮ᄂᆞᆫ〮몸〮이立립〮고〮져홈〯애사〯ᄅᆞᆷ〮을〮立
  립〮게〮ᄒᆞ며〮몸〮이達달〮코져〮홈〯애사〯ᄅᆞᆷ〮을〮達달〮케〮
  ᄒᆞ〮ᄂᆞ〮니라〮}
\原{能느ᇰ近근〯取ᄎᆔ〯譬비〯면可가〯謂위〮仁ᅀᅵᆫ之지方바ᇰ
  也야〯巳이〯니라}
\諺{能느ᇰ히〮갓〮가온〮ᄃᆡ〮取ᄎᆔ〯ᄒᆞ〮야〮譬비〯ᄒᆞ면〮可가〯히〮
  仁ᅀᅵᆫ의〮方바ᇰ이〮라니ᄅᆞᆯ띠〮니라〮}
\篇{述슐〯而ᅀᅵ第뎨〯七칠〮}
\原{子ᄌᆞㅣ曰왈〮述슐〮而ᅀᅵ不블〮作작〮ᄒᆞ며信신〯而ᅀᅵ好
  호〯古고〯를竊졀〮比비〯於어我아〯老로〯彭ᄑᆡᇰᄒᆞ노라}
\諺{子ᄌᆞ〮ㅣᄀᆞᆯᄋᆞ샤〮ᄃᆡ〮述슐〮ᄒᆞ고〮作작〮디아니〮ᄒᆞ며〮
  信신〯ᄒᆞ고〮녜〯를〮됴〯히너김〯을〮그윽이우〮리老로〯
  彭ᄑᆡᇰ의〮게比비〯ᄒᆞ노라〮}
\原{○子ᄌᆞ〮ㅣ曰왈〮默믁〮而ᅀᅵ識지〮之지ᄒᆞ며學ᄒᆞᆨ〮而ᅀᅵ
  不블〮厭염〯ᄒᆞ며誨회〯人ᅀᅵᆫ不블〮倦권〯이何하有유〯於
  어我아〯哉ᄌᆡ오}
\諺{子ᄌᆞ〮ㅣᄀᆞᆯᄋᆞ샤〮ᄃᆡ默믁〮ᄒᆞ〮야〮識지〮ᄒᆞ며〮學ᄒᆞᆨ〮ᄒᆞ
  야〮厭염〯티아니〮ᄒᆞ며〮사〯ᄅᆞᆷ〮ᄀᆞᄅᆞ침〮을〮게을이〮아
  니〮홈〯이〮므서〮시내게인ᄂᆞ뇨〮}
\原{○子ᄌᆞ〮ㅣ曰왈〮德덕〮之지不블〮修슈와學ᄒᆞᆨ〮之지
  不블〮講가ᇰ〯과聞문義의〯不블〮能느ᇰ徙ᄉᆞ〯ᄒᆞ며不블〮善
  션〯不블〮能느ᇰ改ᄀᆡ〯ㅣ是시〯吾오憂우也야〯ㅣ니라}
\諺{子ᄌᆞ〮ㅣᄀᆞᆯᄋᆞ샤〮ᄃᆡ〮德덕〮의〮修슈티〮몯〯홈〯과〮學ᄒᆞᆨ〮
  의〮講가ᇰ〯티〮몯〯홈〯과〮義의〯를〮듣고〮能느ᇰ히〮徙ᄉᆞ〯티〮
  몯〯ᄒᆞ며〮善션〯티〮몯〯ᄒᆞᆫ거슬〮能느ᇰ히〮고티〮디몯〯홈〯
  이이〮내의〮시름〯이〮니라〮}
\原{○子ᄌᆞ〮之지燕연〯居거애申신申신如ᅀᅧ也야〯ᄒᆞ시
  며夭요夭요如ᅀᅧ也야〯ㅣ러시다}
\諺{子ᄌᆞ〮의燕연〯居거ᄒᆞ〮심애〮申신申신ᄐᆞᆺ〮ᄒᆞ〮시며〮
  夭요夭요ᄐᆞᆺ〮ᄒᆞ〮더〮시다〮}
\原{○子ᄌᆞ〮ㅣ曰왈〮甚심〯矣의〯라吾오衰쇠也야〯ㅣ여久
  구〯矣의〯라吾오不블〮復부〯夢모ᇰ〯見견〯周쥬公고ᇰ이로
  다}
\諺{子ᄌᆞ〮ㅣᄀᆞᆯᄋᆞ샤〮ᄃᆡ〮甚심〯ᄒᆞ다〮내〮衰쇠홈〯이〮여오
  라〮다내〮다시〮ᄭᅮᆷ〮에周쥬公고ᇰ을〮보디〮몯〯ᄒᆞ리로
  다〮}
\原{○子ᄌᆞ〮ㅣ曰왈〮志지〮於어道도〯ᄒᆞ며}
\諺{子ᄌᆞ〮ㅣᄀᆞᆯᄋᆞ샤〮ᄃᆡ〮道도〯애〮志지〮ᄒᆞ며〮}
\原{據거〯於어德덕〮ᄒᆞ며}
\諺{德덕〮에據거〯ᄒᆞ며〮}
\原{依의於어仁ᅀᅵᆫᄒᆞ며}
\諺{仁ᅀᅵᆫ에〮依의ᄒᆞ며}
\原{游유於어藝예〯니라}
\諺{藝예〯에〮游유홀〯띠〮니라〮}
\原{○子ᄌᆞ〮ㅣ曰왈〮自ᄌᆞ〮行ᄒᆡᇰ束속〮修슈以이〯上샤ᇰ〯은
  吾오未미〯嘗샤ᇰ無무誨회〯焉언이로라}
\諺{子ᄌᆞ〮ㅣᄀᆞᆯᄋᆞ샤〮ᄃᆡ〮束속〮修슈行ᄒᆡᇰᄒᆞᆫ이〮로브터〮
  ᄡᅥ〮우흔〮내〮일쯕〮ᄀᆞᄅᆞ침〮이업〯디아니〮호〯라}
\原{○子ᄌᆞ〮ㅣ曰왈〮不블憤분〯이어든不블〮啓계〯ᄒᆞ며不블〮
  悱비〯어든不블〮發발〮호ᄃᆡ擧거〮一일〮隅우애不블〮以이〯
  三삼隅우反반〯이어든則즉〮不블〮復부〯也야〯ㅣ니라}
\諺{子ᄌᆞ〮ㅣᄀᆞᆯᄋᆞ샤〮ᄃᆡ〮憤분〯티〮아니〮커든〮啓계〯티〮아
  니〮ᄒᆞ며〮悱비〯티〮아니〮커든〮發발〮티〮아니〮호〯ᄃᆡ〮一
  일〮隅우를〮擧거〯홈〯애〮三삼隅우로〮ᄡᅥ〮反반〯티〮
  몯〯ᄒᆞ〮거든〮곧〮다시〮아니〮ᄒᆞ〮ᄂᆞ니라〮}
\原{○子ᄌᆞ〮ㅣ食식〮於어有유〯喪사ᇰ者쟈〮之지側측〮애
  未미〯嘗샤ᇰ飽포〯也야〯ㅣ러시다}
\諺{子ᄌᆞ〮ㅣ喪사ᇰ인ᄂᆞᆫ者쟈〮의〮겨틔〮셔〮食[?|식〮ᄒᆞ심]애〮
  일쯕〮飽포〯티〮아니〮터시〮다〮}
\原{子ᄌᆞ〮ㅣ於어是시〯日ᅀᅵᆯ〮에哭곡〮則즉〮不블〮歌가ㅣ러
  시다}
\諺{子ᄌᆞ〮ㅣ아〮날애〮哭곡ᄒᆞ시면〮歌가티〮아니〮ᄒᆞ〮더
  시다〮}
\原{○子ᄌᆞ〮ㅣ謂위〮顏안〮淵연曰왈〮用요ᇰ〯之지則즉〮行
  ᄒᆡᇰᄒᆞ고舍샤〯之지則즉〮藏자ᇰ을惟유我아〯與여〯爾ᅀᅵ〯
  ㅣ有유〯是시〯夫부ᅟᅵᆫ뎌}
\諺{子ᄌᆞ〮ㅣ顏ᅌᅡᆫ淵연ᄃᆞ려〮닐러〮ᄀᆞᆯᄋᆞ샤〮ᄃᆡ〮用요ᇰ〯ᄒᆞ
  면〮行ᄒᆡᇰᄒᆞ고〮舍샤〯ᄒᆞ면〮藏자ᇰ홈〯을〮오직〮나와〮다
  ᄆᆞᆺ〮네〯이〮ᄅᆞᆯ〮둔〮ᄂᆞᆫ〮뎌}
\原{子ᄌᆞ〮路로〯ㅣ曰왈〮子ᄌᆞ〮ㅣ行ᄒᆡᇰ三삼軍군則즉〮誰
  슈與여〯ㅣ시리잇고}
\諺{子ᄌᆞ〮路로〯ㅣᄀᆞᆯ오〮ᄃᆡ〯子ᄌᆞ〮ㅣ三삼軍군을〮行ᄒᆡᇰ
  ᄒᆞ〮시〮면〮누〯를〮더브〮러〮ᄒᆞ〮시리잇고〮}
\原{子ᄌᆞ〮ㅣ曰왈〮暴포〯虎호〯馮비ᇰ河하ᄒᆞ야死ᄉᆞ〯而ᅀᅵ無
  무悔회〯者쟈〮를吾오不블〮與여〯也야〯ㅣ니必필〮也야〯
  臨림事ᄉᆞ〯而ᅀᅵ懼구〯ᄒᆞ며好호〮謀모而ᅀᅵ成셔ᇰ者쟈〮
  也야ㅣ니라}
\諺{子ᄌᆞ〮ㅣᄀᆞᆯᄋᆞ샤〮ᄃᆡ〮虎호〯를〮暴포〯ᄒᆞ며〮河하를〮馮
  비ᇰᄒᆞ〮야〮죽어〮도〮뉘〯웃츰〮이〮업〯ᄂᆞᆫ〮者쟈〮를〮내〮더브〮
  러아니〮호〯리니〮반〮ᄃᆞ〮시〮일〯에〮臨림ᄒᆞ〮야〮저허〮ᄒᆞ
  며〮謀모홈〯을〮됴〯히〮너기〮고〮成셔ᇰᄒᆞ〮ᄂᆞᆫ〮者쟈〮ㅣ니
  라〮}
\原{○子ᄌᆞ〮ㅣ曰왈〮富부〯而ᅀᅵ可가〯求구也야〯딘ᅟᅢᆫ雖슈
  執집〮鞭편之지士ᄉᆞ〯ㅣ라도吾오亦역〮爲위之지어니
  와如ᅀᅧ不블〮可가〯求구딘ᅟᅢᆫ從죠ᇰ吾오所소〯好호〮호리
  라}
\諺{子ᄌᆞ〮ㅣᄀᆞᆯᄋᆞ샤〮ᄃᆡ〮富부〯를〮可가〯히求구ᄒᆞᆯ꺼신〮
  댄〮비록〮채〮를〮잠ᄂᆞᆫ士ᄉᆞ〯ㅣ라도〮내〮ᄯᅩ〮ᄒᆞᆫᄒᆞ려〮니
  와〮만〯일〮에〮可가〯히求구티〮몯〯ᄒᆞᆯ꺼신〮댄〮내의〮됴〯
  히〮너기〮ᄂᆞᆫ〮바〮ᄅᆞᆯ〮조초〮리라〮}
\原{○子ᄌᆞ〮之지所소〯愼신〯은齊ᄌᆡ戰젼〯疾질〮이러시다}
\諺{子ᄌᆞ〮의〮삼가〮시ᄂᆞᆫ바〮ᄂᆞᆫ〮齊ᄌᆡ와〮戰젼〯과疾질〮이
  러시다〮}
\原{○子ᄌᆞ〮ㅣ在ᄌᆡ〯齊졔聞문韶쇼ᄒᆞ시고三삼月월을
  不블〮知디肉육〮味미〯ᄒᆞ샤曰왈〮不블〮圖도爲위樂악〮
  之지至지〮於어斯ᄉᆞ也야〯호라}
\諺{子ᄌᆞ〮ㅣ齊졔예〮겨〯샤〮韶쇼를〮드르시〮고〮學ᄒᆞᆨ〮ᄒᆞ〮
  신석〯ᄃᆞᆯ〮을〮肉육〮味미〯를〮아〯디몯〯ᄒᆞ〮샤ᄀᆞᆯᄋᆞ샤〮ᄃᆡ〮
  樂악〮을〮홈〯이〮이〮예〮니를〮줄〮을〮圖도티〮아니〮호〯라〮}
\原{○冉ᅀᅧᆷ〯有유〯ㅣ曰왈〮夫부子ᄌᆞ〮ㅣ爲위〮衛위〮君군
  乎호아子ᄌᆞ〮貢고ᇰ〯이曰왈〮諾락〮다吾오將쟈ᇰ問문〯
  之지호리라}
\諺{冉ᅀᅧᆷ〯有유〯ㅣᄀᆞᆯ오〮ᄃᆡ〮夫부子ᄌᆞ〮ㅣ衛위〮君군을〮
  爲위〮ᄒᆞ〮시랴〮子ᄌᆞ〮貢고ᇰ〯이〮ᄀᆞᆯ오〮ᄃᆡ〮諾락다〮내〮쟈ᇰ
  ᄎᆞᆺ〮묻〯ᄌᆞ오〮리라〮}
\原{入ᅀᅵᆸ〮曰왈〮伯ᄇᆡᆨ〮夷이叔슉〮齊졔ᄂᆞᆫ何하人ᅀᅵᆫ也야
  ㅣ잇고曰왈〮古고〯之지賢현人ᅀᅵᆫ也야〯ㅣ니라曰왈〮怨
  원〯乎호ㅣ잇가曰왈〮求구仁ᅀᅵᆫ而ᅀᅵ得득〮仁ᅀᅵᆫ이어니
  又우〯何하怨원〯이리오出츌〮曰왈〮夫부子ᄌᆞ〮ㅣ不블〮
  爲위〮也야〯ㅣ시리러라}
\諺{들〮어가〮ᄀᆞᆯ오〮ᄃᆡ〮伯ᄇᆡᆨ〮夷이와〮叔슉〮齊졔ᄂᆞᆫ〮엇〯던〮
  사〯ᄅᆞᆷ이〮니잇고〮ᄀᆞᆯᄋᆞ샤〮ᄃᆡ〮녯〯賢현人ᅀᅵᆫ이니라〮
  ᄀᆞᆯ오〮ᄃᆡ〮怨원〯ᄒᆞ〮더니〮잇가〮ᄀᆞᆯᄋᆞ샤〮ᄃᆡ〮仁ᅀᅵᆫ을〮求
  구ᄒᆞ〮야仁ᅀᅵᆫ을〮得득〮ᄒᆞ〮야〮니〮ᄯᅩ〮엇〯디〮怨원〯ᄒᆞ리〮
  오〮나〮와〮ᄀᆞᆯ오〮ᄃᆡ〮夫부子ᄌᆞ〮ㅣ爲위〮티아니〮ᄒᆞ〮시〮
  리〮러라〮}
\原{○子ᄌᆞ〮ㅣ曰왈〮飯반〮疏소食ᄉᆞ〯飮음〯水슈〮ᄒᆞ고曲곡〮
  肱그ᇰ而ᅀᅵ枕침〯之지라도樂락〮亦역〮在ᄌᆡ〯其기中듀ᇰ
  矣의〯니不블〮義의〯而ᅀᅵ富부〯且챠〯貴귀〯ᄂᆞᆫ於어我
  아〯애如ᅀᅧ浮부雲운이니라}
\諺{子ᄌᆞ〮ㅣᄀᆞᆯᄋᆞ샤〮ᄃᆡ〮疏소食ᄉᆞ〯를〮飯반〮ᄒᆞ며〮水슈〮
  를飮음〯ᄒᆞ고〮肱그ᇰ을〮曲곡〮ᄒᆞ〮야〮枕침〯ᄒᆞ〮야도〮樂
  락〮이〮ᄯᅩ〮ᄒᆞᆫ그가온〮대〮인ᄂᆞ니〮義의〯아니〮오〮富부〯
  코〮ᄯᅩ〮貴귀〯홈〯은〮내게浮부雲운ᄀᆞ〮ᄐᆞ〮니라〮}
\原{○子ᄌᆞ〮ㅣ曰왈〮加가〯我아〯數수〯年년ᄒᆞ아卒졸〮以
  이〯學ᄒᆞᆨ〮易역〮이면可가〯以이〯無무大대〯過과〯矣의〯리라}
\諺{子ᄌᆞ〮ㅣᄀᆞᆯᄋᆞ샤〮ᄃᆡ〮나〯ᄅᆞᆯ〮두〯어〮ᄒᆡ〮를〮假가〯ᄒᆞ〮야〮ᄆᆞ
  ᄎᆞᆷ〮내〮ᄡᅥ〮易역〮을〮學ᄒᆞᆨ〮ᄒᆞ면〮可가〯히〮ᄡᅥ〮큰〮허믈〮이〮
  업〯스〮리라〮}
\原{○子ᄌᆞ〮所소〯雅아〯言언ᄋᆞᆫ詩시〮書셔執집〮禮례〮ㅣ
  皆ᄀᆡ雅아〯言언也야〯ㅣ러시다}
\諺{子ᄌᆞ〮의〮샤ᇰ해〮言언ᄒᆞ〮시ᄂᆞᆫ〮바〮ᄂᆞᆫ〮詩시와〮書셔와〮
  자받〮ᄂᆞᆫ〮禮례〮ㅣ다〯샤ᇰ해〮言언이〮러시다〮}
\原{○葉셥〮公고ᇰ이問문〯孔고ᇰ〮子ᄌᆞ〮於어子ᄌᆞ〮路로〯ㅣ어
  ᄂᆞᆯ子ᄌᆞ〮路로〯ㅣ不블〮對ᄃᆡ〯ᄒᆞᆫ대}
\諺{葉셥〮公고ᇰ이孔고ᇰ〮子ᄌᆞ〮를〮子ᄌᆞ〮路로〯의〮게무러〮
  ᄂᆞᆯ〮子ᄌᆞ〮路로〯ㅣ對ᄃᆡ〯티〮아니〮ᄒᆞᆫ대〮}
\原{子ᄌᆞ〮ㅣ曰왈〮女ᅀᅧ〯ㅣ奚ᄒᆡ不블〮曰왈〮其기爲위人
  ᅀᅵᆫ也야〯ㅣ發발〮憤분〯忘마ᇰ食식〮ᄒᆞ며樂락〮以이〯忘마ᇰ
  憂우ᄒᆞ야不블〮知디老로〯之지將쟈ᇰ至지〮云운爾ᅀᅵ〯
  오}
\諺{子ᄌᆞ〮ㅣᄀᆞᆯᄋᆞ샤〮ᄃᆡ〮네〯엇〯디〮ᄀᆞᆯ오〮ᄃᆡ〮그사〯ᄅᆞᆷ〮이〮로
  옴〮이〮憤분〯을〮發발〮ᄒᆞ〮야〮食식〮을〮니즈〮며〮樂락〮ᄒᆞ〮
  야〮ᄡᅥ〮시름〮을〮니저〮늘금〮의〮쟈ᇰᄎᆞᆺ〮니르〮롬을〮아〯디〮
  몯〯ᄒᆞᆫ〮다〮아니〮ᄒᆞ뇨〮}
\原{○子ᄌᆞ〮ㅣ曰왈〮我아〯非비生ᄉᆡᇰ而ᅀᅵ知디之지者
  쟈〮ㅣ라好호〯古고〯敏민〮以이〯求구之지者쟈〮也야〯ㅣ로
  라}
\諺{子ᄌᆞ〮ㅣᄀᆞᆯᄋᆞ샤〮ᄃᆡ〮내〮生ᄉᆡᇰᄒᆞ〮야〮아〯ᄂᆞᆫ〮者쟈〮ㅣ아
  니〮라〮녜〯를〮됴〯히〮너겨〮敏민〮히〮ᄡᅥ〮求구ᄒᆞ〮ᄂᆞᆫ〮者쟈〮
  ㅣ로라〮}
\原{○子ᄌᆞ〮ㅣ不블語어〯怪괴〯力력〮亂란〯神신이러시다}
\諺{子ᄌᆞ〮ㅣ怪괴〯와〮力력〮과〮亂란〯과〮神신을〮니ᄅᆞ디〮
  아니〮ᄒᆞ〮더시다〮}
\原{○子ᄌᆞ〮ㅣ曰왈〮三삼人ᅀᅵᆫ行ᄒᆡᇰ애必필〮有유〯我아〯
  師ᄉᆞ焉언이니擇ᄐᆡᆨ〮其기善션〯者쟈〮而ᅀᅵ從죠ᇰ之지
  오其기不블〮善션〯者쟈〮而ᅀᅵ改ᄀᆡ〯之지니라}
\諺{子ᄌᆞ〮ㅣᄀᆞᆯᄋᆞ샤〮ᄃᆡ〮세〯사〯ᄅᆞᆷ〮이〮行ᄒᆡᇰ홈〯애반〮ᄃᆞ〮시〮
  내스스ᇰ이〮인ᄂᆞ니〮그〮어〮딘〮者쟈〮ᄅᆞᆯ〮ᄀᆞᆯᄒᆡ〮여〮좃고〮
  그어〮디디〮아닌〮者쟈〮ᄅᆞᆯ〮고틸〮띠〮니라〮}
\原{○子ᄌᆞ〮ㅣ曰왈〮天텬生ᄉᆡᇰ德덕〮於어予여ㅣ시니桓
  환魋퇴ㅣ其기如ᅀᅧ予여何하ㅣ리오}
\諺{子ᄌᆞ〮ㅣᄀᆞᆯ〮ᄋᆞ〮샤〮ᄃᆡ〮하ᄂᆞᆯ〮히〮德덕〮을〮내게〮生ᄉᆡᇰᄒᆞ〮
  시니〮桓환魋퇴ㅣ그내게〮엇〯디〮리오〮}
\原{○子ᄌᆞ〮ㅣ曰왈〮二ᅀᅵ〯三삼子ᄌᆞ〮ᄂᆞᆫ以이〯我아〯爲위
  隱은〯乎호아吾오無무隱은〯乎호爾ᅀᅵ〯로라吾오無
  무行ᄒᆡᇰ而ᅀᅵ不블〮與여〯二ᅀᅵ〯三삼子ᄌᆞ〮者쟈〮ㅣ是
  시〯丘구也야〯ㅣ니라}
\諺{子ᄌᆞ〮ㅣᄀᆞᆯᄋᆞ샤〮ᄃᆡ〮二ᅀᅵ〯三삼子ᄌᆞ〮ᄂᆞᆫ〮날〯로〮ᄡᅥ〮隱
  은〮ᄒᆞᆫ〮다〮ᄒᆞ〮ᄂᆞ〮냐〮내〮네게隱은〮홈〯이〮업〯소〮라〮내〮行
  ᄒᆡᇰᄒᆞ고〮二ᅀᅵ〯三삼子ᄌᆞ〮에〮與여〮티〮아니〮홈〯이〮업〯
  ᄉᆞᆫ〮者쟈〮ㅣ이〮丘구ㅣ니라〮}
\原{○子ᄌᆞ〮ㅣ以이〯四ᄉᆞ〯敎교〯ᄒᆞ시니文문行ᄒᆡᇰ〯忠튜ᇰ信
  신〯이니라}
\諺{子ᄌᆞ〮ㅣ네〯흐〮로ᄡᅥ〮ᄀᆞᄅᆞ치〮시니〮文문과〮行ᄒᆡᇰ〯과〮
  忠튜ᇰ과〮信신〯이〮니라〮}
\原{○子ᄌᆞ〮ㅣ曰왈〮聖셔ᇰ〯人ᅀᅵᆫ을吾오不블〮得득〮而ᅀᅵ
  見견〯之지矣의〯어든得득〮見견〯君군子ᄌᆞ〮者쟈〮ㅣ면斯
  ᄉᆞ可가〯矣의〯니라}
\諺{子ᄌᆞ〮ㅣᄀᆞᆯᄋᆞ샤〮ᄃᆡ〮聖셔ᇰ〯人ᅀᅵᆫ을〮내〮어〯더〮보디〮몯〮
  ᄒᆞ〮거든〮君군子ᄌᆞ〮를〮어〯더〮보면〮이〮可가〯ᄒᆞ니〮라〮}
\原{子ᄌᆞ〮ㅣ曰왈〮善션〯人ᅀᅵᆫ을吾오不블〮得득〮而ᅀᅵ見
  견〯之지矣의〯어든得득〮見견〯有유〯恒ᄒᆞᇰ者쟈〮ㅣ면斯ᄉᆞ
  可가〯矣의〯니라}
\諺{子ᄌᆞ〮ㅣᄀᆞᆯᄋᆞ샤〮ᄃᆡ〮善션〯人ᅀᅵᆫ을〮내〮어〯더〮보디〮몯〯
  ᄒᆞ〮거든〮恒ᄒᆞᇰ인ᄂᆞᆫ者쟈〮를〮어〯더〮보면〮이〮可가〯ᄒᆞ
  니라}
\原{亡무而ᅀᅵ爲위有유〯ᄒᆞ며虛허而ᅀᅵ爲우盈여ᇰᄒᆞ며約
  약〮而ᅀᅵ爲위泰태〯면難난乎호有유〯恒ᄒᆞᇰ矣의〯니라}
\諺{亡무호〯ᄃᆡ〮有유〯호〯라〮ᄒᆞ며〮虛허호〯ᄃᆡ〮盈여ᇰ호〯라〮
  ᄒᆞ며〮約약〮호〯ᄃᆡ〮泰태〯호〯라〮ᄒᆞ면〮恒ᄒᆞᇰ이심이〮어
  려〮우니라〮}
\原{○子ᄌᆞᄂᆞᆫ釣됴〮而ᅀᅵ不블〮綱가ᇰᄒᆞ시며戈익〮不블〮射
  셕〮宿슉〮이러시다}
\諺{子ᄌᆞ〮ᄂᆞᆫ〮釣됴〮ᄒᆞ〮시고〮綱가ᇰ티〮아니〮ᄒᆞ〮시며〮戈익
  ᄒᆞ〮샤〮ᄃᆡ〮宿슉〮을〮射셕〮디〮아니〮ᄒᆞ〮더〮시다〮}
\原{○子ᄌᆞ〮ㅣ曰왈〮蓋개〯有유〯不블〮知디而ᅀᅵ作작〮之
  지者쟈〮ᅀᅡ我아〯無무是시〯也야〯ㅣ로라多다聞문ᄒᆞ아
  擇ᄐᆡᆨ〮其기善션〯者쟈〮而ᅀᅵ從죠ᇰ之지ᄒᆞ며多다見견〯
  而ᅀᅵ識지〮之지ㅣ知디之지次ᄎᆞ〮也야〯ㅣ니라}
\諺{子ᄌᆞ〮ㅣᄀᆞᆯᄋᆞ샤〮ᄃᆡ〮아〯디〮몯〯ᄒᆞ고〮作작〮ᄒᆞᆯ이〮인ᄂᆞ
  냐〮내〮이〮업〯소라〮해〯들어〮그善션〯을〮擇ᄐᆡᆨ〮ᄒᆞ〮야〮졷
  ᄎᆞ〮며〮해〯보〮와〮識지〮홈〯이〮知디의〮次ᄎᆞ〮ㅣ니라}
\原{○互호〯鄕햐ᇰ은難난與여〯言언이러니童도ᇰ子ᄌᆞ〮ㅣ
  見현〮커ᄂᆞᆯ門문人ᅀᅵᆫ이惑혹〮ᄒᆞᆫ대}
\諺{互호〯鄕햐ᇰ은〮더브〮러〮말〯홈이어렵〮더니〮童도ᇰ子
  ᄌᆞ〮ㅣ뵈〯ᄋᆞ〮와늘〮門문人ᅀᅵᆫ이〮惑혹〮ᄒᆞᆫ대〮}
\原{子ᄌᆞ〮ㅣ曰왈〮與여〯其기進진〯也야〯ㅣ오不블〮與여〯其
  기退퇴〯也야〯ㅣ니唯유何하甚심〯이리오人ᅀᅵᆫ이潔결〮
  已긔〮以이〯進진〯이어든與여〯其기潔결〮也야〯ㅣ오不블〮
  保보〯其기往와ᇰ〯也야〯ㅣ며}
\諺{子ᄌᆞ〮ㅣᄀᆞᆯᄋᆞ샤〮ᄃᆡ〮사〯ᄅᆞᆷ〮이〮已긔〮를潔결〮ᄒᆞ〮야〮ᄡᅥ〮
  進진〯ᄒᆞ〮거든〮그潔결〮을〮與여〯ᄒᆞ고〮그往와ᇰ〯을〮保
  보〯티〮몯〯ᄒᆞ며〮그進진〯홈〯을〮與여〯ᄒᆞ고〮그退퇴〯를
  與여〯홈〯이〮아니〮니엇〯디〮甚심〯히〮ᄒᆞ리〮오〮}
\原{○子ᄌᆞ〮ㅣ曰왈〮仁ᅀᅵᆫ遠원〯乎호哉ᄌᆡ아我아〯欲욕〮
  仁ᅀᅵᆫ이면斯ᄉᆞ仁ᅀᅵᆫ이至지〮矣의〮니라}
\諺{子ᄌᆞㅣᄀᆞᆯᄋᆞ샤〮ᄃᆡ〮仁ᅀᅵᆫ이〮머〯냐〮내〮仁ᅀᅵᆫ코〮쟈〮ᄒᆞ
  면〮이〮예〮仁ᅀᅵᆫ이〮니르〮ᄂᆞ니라〮}
\原{○陳딘司ᄉᆞ敗패〯ㅣ問문〯昭쇼公고ᇰ이知디禮례〮
  乎호가ㅣ잇孔고ᇰ〮子ᄌᆞ〮ㅣ曰왈〮知디禮례〮시니라}
\諺{陳딘人司ᄉᆞ敗패〯ㅣ묻〯ᄌᆞ〮오ᄃᆡ〮昭쇼公고ᇰ이〮禮
  례〮를〮아〯ᄅᆞ〮시더니〮잇가〮孔고ᇰ〮子ᄌᆞ〮ㅣᄀᆞᆯᄋᆞ샤〮ᄃᆡ〮
  禮례〮를〮아〯ᄅᆞ〮시더니〮라〮}
\原{孔고ᇰ〮子ᄌᆞ〮ㅣ退퇴〯커시ᄂᆞᆯ揖읍〮巫무馬마〯期긔而ᅀᅵ
  進진〯之지曰왈〮吾오聞문君군子ᄌᆞ〮ᄂᆞᆫ不블〮黨■
  이라호니君군子ᄌᆞ〮도亦역〮黨다ᇰ〮乎호아君군이取ᄎᆔ〯
  於어吳오ᄒᆞ니爲위同도ᇰ姓셔ᇰ〯이라謂위〮之지吳오孟
  ᄆᆡᇰ〯子ᄌᆞ〮ㅣ라ᄒᆞ니君군而ᅀᅵ知디禮례〮면孰슉〮不블〮知
  디禮례〯리오}
\諺{孔고ᇰ〮子ᄌᆞ〮ㅣ退퇴〯ᄒᆞ〮야〮시늘〮巫무馬마〯期긔를
  揖읍〮ᄒᆞ〮야〮나오〮와〮ᄀᆞᆯ오〮ᄃᆡ〮나〮ᄂᆞᆫ〮들오〮니君군子
  ᄌᆞ〮ᄂᆞᆫ〮黨다ᇰ〮티〮아니〮ᄒᆞᆫ〮다〮호〯니〮君군子ᄌᆞ〮도〮ᄯᅩᄒᆞᆫ
  黨다ᇰ〮ᄒᆞ〮ᄂᆞ〮냐〮君군이〮吳오애〮取ᄎᆔ〯ᄒᆞ니〮同도ᇰ姓
  셔ᇰ〯인〮디〮라〮닐오ᄃᆡ〮吳오孟ᄆᆡᇰ〯子ᄌᆞ〮ㅣ라〮ᄒᆞ니〮君
  군이〮오〮禮례〮를〮알〯면〮뉘〮禮례〮를〮아〯디〮몯〯ᄒᆞ리오〮}
\原{巫무馬마〯期긔ㅣ以이〯告고〯ᄒᆞᆫ대子ᄌᆞ〮ㅣ曰왈〮丘구
  也야〯ㅣ幸ᄒᆡᇰ〯이로다苟구〯有유〯過과〮ㅣ어든人ᅀᅵᆫ必필〮
  知디之지온여}
\諺{巫무馬마〯期긔ㅣᄡᅥ〮告고〯ᄒᆞᆫ〯대〮子ᄌᆞ〮ㅣᄀᆞᆯᄋᆞ샤〮
  ᄃᆡ〮立구ㅣ幸ᄒᆡᇰ〯이〮로다〮진실〮로〮허믈〮이〮잇거든〮
  사〯ᄅᆞᆷ〮이〮반〮ᄃᆞ〮시〮알〯고〮녀〮}
\原{○子ᄌᆞ〮ㅣ與여〯人ᅀᅵᆫ歌가而ᅀᅵ善션〯이어든必필〮使
  ᄉᆞ〯反반〯之지ᄒᆞ시고而ᅀᅵ後후〯和화〯之지러시다}
\諺{子ᄌᆞ〮ㅣ사〯ᄅᆞᆷ〮으〮로〮더브〮러〮歌가ᄒᆞ〮심〮애〮善션〯ᄒᆞ〮
  거든〮반〮ᄃᆞ시〮ᄒᆞ여〮곰〮反반〯ᄒᆞ라〮ᄒᆞ〮시고〮後후〯에〮
  和화〯ᄒᆞ〮더시다〮}
\原{○子ᄌᆞ〮ㅣ曰왈〮文문莫막〮吾오猶유人ᅀᅵᆫ也야〯아
  躬구ᇰ〮行ᄒᆡᇰ君군子ᄌᆞ〮ᄂᆞᆫ則즉〮吾오ㅣ未미〯之지有
  유〯得득〮호라}
\諺{子ᄌᆞ〮ㅣᄀᆞᆯᄋᆞ샤〮ᄃᆡ〮文문ᄋᆞᆫ〮아니내〮사〯ᄅᆞᆷ〮ᄀᆞ〮ᄐᆞ냐〮
  君군子ᄌᆞ〮를〮몸〮소行ᄒᆡᇰ홈〯은〮곧〮내〮得득〮홈〯이〮잇
  디〮몯〯호〯라}
\原{○子ᄌᆞ〮ㅣ曰왈〮若ᅀᅣᆨ〮聖셔ᇰ〯與여〯仁ᅀᅵᆫ은則즉〮吾오
  豈긔〮敢감〯이리오抑억〮爲위之지不블〮厭염〯ᄒᆞ며誨회〯
  人ᅀᅵᆫ不블〮倦권〯은則즉〮可가〯謂위〮云운爾ᅀᅵ〯已이〯
  矣의〯니라公고ᇰ西셔華화ㅣ曰왈〮正져ᇰ〯唯유弟뎨〯子
  ᄌᆞ〮ㅣ不블〮能느ᇰ學ᄒᆞᆨ〮也야〯ㅣ로소이다}
\諺{子ᄌᆞ〮ㅣᄀᆞᆯᄋᆞ샤〮ᄃᆡ〮만〯일〮聖셔ᇰ〯과〮다ᄆᆞᆺ仁ᅀᅵᆫ은〮내〮
  엇〯디〮敢감〯ᄒᆞ리오爲위홈〯을〮厭염〯티〮아니〮ᄒᆞ며〮
  사〯ᄅᆞᆷ〮ᄀᆞᄅᆞ침〮을〮게을리〮아니〮홈〯은〮곧〮可가〯히〮니
  ᄅᆞᆯᄯᆞᄅᆞᆷ이〮니라〮公고ᇰ西셔華화ㅣᄀᆞᆯ오〮ᄃᆡ〮正져ᇰ〯
  히〮弟뎨〯子ᄌᆞㅣ能느ᇰ히學ᄒᆞᆨ〮디몯〯홈〯이〮로소이〮
  다〮}
\原{○子ᄌᆞ〮ㅣ疾질〮病벼ᇰ〯이어시ᄂᆞᆯ子ᄌᆞ〮路로〯ㅣ請쳐ᇰ〮禱도〯
  ᄒᆞᆫ대子ᄌᆞ〮ㅣ曰왈〮有유〯諸져아子ᄌᆞ〮路로〯ㅣ對ᄃᆡ〯曰
  왈〮有유〯之지ᄒᆞ니誄뢰〯예曰왈〮禱도〯爾ᅀᅵ〯于우上샤ᇰ〯
  下하〯神신祇기라ᄒᆞ도소이다子ᄌᆞ〮ㅣ曰왈〮丘구之지禱
  도〯ㅣ久구〯矣의〯니라}
\諺{子ᄌᆞ〮ㅣ疾질〮이〮病벼ᇰ〯ᄒᆞ〮거시ᄂᆞᆯ〮子ᄌᆞ路로〯ㅣ禱
  도〯홈〯을〮請쳐ᇰ〮ᄒᆞᆫ대〮子ᄌᆞ〮ㅣᄀᆞᆯᄋᆞ샤〮ᄃᆡ〮인ᄂᆞ냐〮子
  ᄌᆞ〮路로〯ㅣ對ᄃᆡ〯ᄒᆞ〮야〮ᄀᆞᆯ오〮ᄃᆡ〮인ᄂᆞ니〮誄뢰〯예〮ᄀᆞᆯ
  오〮ᄃᆡ〮너ᄅᆞᆯ〮上샤ᇰ〯下하〯人神신祇기예〮비〯다〮ᄒᆞ〮도〮
  소이〮다〮子ᄌᆞ〮ㅣᄀᆞᆯᄋᆞ샤〮ᄃᆡ〮丘구의〮禱도〯홈〯이〮오
  라〮니라〮}
\原{○子ᄌᆞ〮ㅣ曰왈〮奢샤則즉〮不블〮孫손〯ᄒᆞ고儉검〯則즉〮
  固고〯ㅣ니與여〯其기不블〮孫손〮也야〯론寧녀ᇰ固고〯ㅣ니
  라}
\諺{子ᄌᆞ〮ㅣᄀᆞᆯᄋᆞ샤〮ᄃᆡ〮奢샤ᄒᆞ면〮孫손〯티〮아니〮ᄒᆞ고〮
  儉검〯ᄒᆞ면〮固고〯ᄒᆞ〮ᄂᆞ〮니그孫손〯티〮아니〮홈〯오로
  더브〮러〮론ᄎᆞᆯ하리〮固고〯홀〯띠〮니라〮}
\原{○子ᄌᆞ〮ㅣ曰왈〮君군子ᄌᆞ〮ᄂᆞᆫ坦탄〯蕩타ᇰ〯蕩타ᇰ〯이오小
  쇼〯人ᅀᅵᆫ은長댜ᇰ戚쳑〮戚쳑〮이니라}
\諺{子ᄌᆞ〮ㅣᄀᆞᆯᄋᆞ샤〮ᄃᆡ〮君군子ᄌᆞ〮ᄂᆞᆫ〮坦탄〯히蕩타ᇰ〯蕩
  타ᇰ〯ᄒᆞ고〮小쇼〯人ᅀᅵᆫᄋᆞᆫ〮기리〮戚쳑〮戚쳑〮ᄒᆞ〮ᄂᆞ〮니라〮}
\原{○子ᄌᆞ〮ᄂᆞᆫ溫온而ᅀᅵ厲려〯ᄒᆞ시며威위而ᅀᅵ不블〮猛
  ᄆᆡᇰ〯ᄒᆞ시며恭고ᇰ而ᅀᅵ安안이러시다}
\諺{子ᄌᆞ〮ᄂᆞᆫ〮溫온호〯ᄃᆡ〮厲려〯ᄒᆞ〮시며〮威위호〯ᄃᆡ〮猛ᄆᆡᇰ〯
  티〮아니〮ᄒᆞ〮시며〮恭고ᇰ호〯ᄃᆡ〮安ᅙᅡᆫᄒᆞ〮더시다〮}

\篇{泰태〮伯ᄇᆡᆨ〮第뎨〯八팔〮}
\原{子ᄌᆞ〮ㅣ曰왈〮泰태〯伯ᄇᆡᆨ〮은其기可가〯謂위〮至지〮德
  덕〮也야〯已이〯矣의〯로다三삼以이〯天텬下하〯讓ᅀᅣᇰ〯호ᄃᆡ
  民민無무得득〮而ᅀᅵ稱치ᇰ焉언이온여}
\諺{子ᄌᆞ〮ㅣᄀᆞᆯᄋᆞ샤〮ᄃᆡ〮泰태〮伯ᄇᆡᆨ〮은〮그可가〯히지〮극〮
  ᄒᆞᆫ德덕〮이〮라〮니ᄅᆞᆯᄯᆞᄅᆞᆷ이〮로다세〯번〮天텬下하〯
  로ᄡᅥ〮讓ᅀᅣᇰ〯호〯ᄃᆡ〮民민이〮시러〮곰稱치ᇰ홈〯이업〮고녀}
\原{○子ᄎᆞ〮ㅣ曰왈〮恭고ᇰ而ᅀᅵ無무禮례〮則즉〮勞로ᄒᆞ고
  愼신〯而ᅀᅵ無무禮례〮則즉〮葸싀〯ᄒᆞ고勇요ᇰ〯而ᅀᅵ無무
  禮례〮則즉〮亂란〯ᄒᆞ고直딕〮而ᅀᅵ無무禮례〮則즉〮絞교〯
  ㅣ니라}
\諺{子ᄌᆞ〮ㅣᄀᆞᆯᄋᆞ샤〮ᄃᆡ〮恭고ᇰᄒᆞ고〮禮례〮ㅣ업〯스〮면〮勞
  로ᄒᆞ고〮愼신〯ᄒᆞ고〮禮례〮ㅣ업〯스〮면〮葸싀〯ᄒᆞ고〮勇
  요ᇰ〯ᄒᆞ고〮禮례〮ㅣ업〯스〮면〮亂란〯ᄒᆞ고〮直딕〮ᄒᆞ고〮禮
  례〮ㅣ업〯스〮면〮絞교〯ᄒᆞ〮ᄂᆞ〮니라〮}
\原{君군子ᄌᆞ〮ㅣ篤독〮於어親친則즉〮民민興흐ᇰ於어
  仁ᅀᅵᆫᄒᆞ고故고〯舊구〯를不블〮遺유則즉〮民민不블〮偸
  투ㅣ니라}
\諺{君군子ᄌᆞ〮ㅣ親친애〮篤독〮ᄒᆞ면〮民민이〮仁ᅀᅵᆫ〮애〮
  興흐ᇰᄒᆞ고〮故고〯舊구〯를〮遺유티〮아니〮ᄒᆞ면〮民민
  이〮偸투티〮아니〮ᄒᆞ〮ᄂᆞ〮니라〮}
\原{○曾즈ᇰ子ᄌᆞ〮ㅣ有유〯疾질〮ᄒᆞ샤召쇼〯門문弟뎨〯子ᄌᆞ〮
  曰왈〮啓계〯予여足죡〮ᄒᆞ며啓계〯予여手슈〮ᄒᆞ라詩시云
  운戰젼〯戰젼〯兢그ᇰ兢그ᇰᄒᆞ야如ᅀᅧ臨림深심淵연ᄒᆞ며
  如ᅀᅧ履리〯薄박〮氷비ᇰ이라ᄒᆞ니而ᅀᅵ今금而ᅀᅵ後후〯에ᅀᅡ
  吾오知디免면〯夫부ㅣ와라小쇼〯子ᄌᆞ〮아}
\諺{曾즈ᇰ子ᄌᆞ〮ㅣ疾질〮이〮겨〯샤〮門문弟뎨〯子ᄌᆞ〮를〮블
  러〮ᄀᆞᆯᄋᆞ샤〮ᄃᆡ〮내발〮을〮啓계〯ᄒᆞ며〮내손〮을〮啓계〯ᄒᆞ
  라〮詩시예〯닐오〮ᄃᆡ〮戰젼〯戰젼〯ᄒᆞ며〮兢그ᇰ兢그ᇰᄒᆞ〮
  야〮기픈〮모〮슬〮디ᄂᆞᄃᆞᆺ〮ᄒᆞ며〮여론〮어름〮을〮ᄇᆞᆲ〯ᄃᆞᆺ〮ᄒᆞ
  다〮ᄒᆞ니〮이〮젠〮後후에〮ᅀᅡ〮내〮免면〯홈〯을〮알〯와〮라小
  쇼〯子ᄌᆞ〮아}
\原{○曾즈ᇰ子ᄌᆞ〮ㅣ有유〯疾질〮이어시ᄂᆞᆯ孟ᄆᆡᇰ〯敬겨ᇰ〯子ᄌᆞㅣ
  問문〯之지러니}
\諺{曾즈ᇰ子ᄌᆞ〮ㅣ疾질〮이〮잇거시ᄂᆞᆯ〮孟ᄆᆡᇰ〯敬겨ᇰ〯子ᄌᆞ
  ㅣ묻〯ᄌᆞᆸ〮더니〮}
\原{曾즈ᇰ子ᄌᆞ〮ㅣ言언曰왈〮鳥됴〮之지將쟈ᇰ死ᄉᆞ〯애其
  기鳴며ᇰ也야〯ㅣ哀ᄋᆡᄒᆞ고人ᅀᅵᆫ之지將쟈ᇰ死ᄉᆞ〯애其
  기言언也야〯ㅣ善션〯이니라}
\諺{曾즈ᇰ子ᄌᆞ〮ㅣ닐러〮ᄀᆞᆯᄋᆞ샤〮ᄃᆡ새〯쟈ᇰᄎᆞᆺ〮죽음〮애〮그
  우롬〮이〮슬프고〮사〯ᄅᆞᆷ〮이〮쟈ᇰᄎᆞᆺ〮죽음〮애〮그마〯리어〮
  디〮ᄂᆞ니라〮}
\原{君군子ᄌᆞ〮ㅣ所소〯貴귀〯乎호道도〯者쟈〮ㅣ三삼이니
  動도ᇰ〯容요ᇰ貌모〮애斯ᄉᆞ遠완〯暴포〯慢만〮矣의〯며正
  져ᇰ〯顏안色ᄉᆡᆨ〮애斯ᄉᆞ近근〯信신〯矣의〯며出츌〮辭ᄉᆞ
  氣긔〮애斯ᄉᆞ遠완〯鄙비〯倍패〯矣의〯니籩변豆두〯之
  지事ᄉᆞ〯則즉〮有유〯司ᄉᆞㅣ存존이니라}
\諺{君군子ᄌᆞ〮ㅣ道도〯애〮貴귀〯히너기〮ᄂᆞᆫ〮배〮세〯히〮니〮
  容요ᇰ貌모〮를〮動도ᇰ〯ᄒᆞ〮욤애〮이〮예〮暴포〯慢만〮을〮멀
  리〮ᄒᆞ며〮ᄂᆞᆺ비〮츨〮正져ᇰ〯ᄒᆞ〮욤애〮이〮예〮信신〯에〮갓가
  오〮며〮辭ᄉᆞ氣긔〮를〮내욤〮애〮이〮예〮鄙비〯倍패〯를〮멀
  리〮홀〯띠〮니〮籩변豆두〯ㅅ일〯은〮有유〯司ᄉᆞㅣ인ᄂᆞ
  니〮라}
\原{○曾즈ᇰ子ᄌᆞ〮ㅣ曰왈〮以이〯能느ᇰ으로問문〯於어不블〮
  能느ᇰᄒᆞ며以이〯多다로問문〯於어寡과〯ᄒᆞ며有유〯若ᅀᅣᆨ〮
  無무ᄒᆞ며實실〮若ᅀᅣᆨ〮虛허ᄒᆞ며犯범〯而ᅀᅵ不블〮校교〯를
  昔셕〮者쟈〮吾오友우〯ㅣ嘗샤ᇰ從죠ᇰ事ᄉᆞ〯於어斯ᄉᆞ
  矣의〯러니라}
\諺{曾즈ᇰ子ᄌᆞ〮ㅣᄀᆞᆯᄋᆞ샤〮ᄃᆡ〮能느ᇰ오〮로ᄡᅥ〮不블〮能느ᇰ
  애〮무르〮며〮多다로〮ᄡᅥ〮寡과〯애〮무르〮며〮이슈〮ᄃᆡ〮업〯
  슨〮ᄃᆞᆺ〮ᄒᆞ며〮實실〮호〯ᄃᆡ〮虛허ᄒᆞᆫᄃᆞᆺ〮ᄒᆞ며〮犯범〯ᄒᆞ〮야〮
  도〮校교〯티〮아니〮홈〯을〮녜〯내버〯디〮일쯕〮이〮예〮從죠ᇰ
  事ᄉᆞ〯ᄒᆞ〮더니라〮}
\原{○曾즈ᇰ子ᄌᆞ〮ㅣ曰왈〮可가〯以이〯託탁〮六륙〮尺쳑〮之
  지孤고ᄒᆞ며可가〯以이〯寄긔〮百ᄇᆡᆨ〮里리〯之지命며ᇰ〯이오
  臨림大대〯節졀〮而ᅀᅵ不블〮可가〮奪탈〮也야〯ㅣ면君군
  子ᄌᆞ〮人ᅀᅵᆫ與여아君군子ᄌᆞ〮人ᅀᅵᆫ也야〯ㅣ니라}
\諺{曾즈ᇰ子ᄌᆞ〮ㅣᄀᆞᆯᄋᆞ샤〮ᄃᆡ〮可가〯히〮ᄡᅥ〮六륙〮尺쳑〮人
  孤고를〮託탁〮ᄒᆞ〮얌〮즉〮ᄒᆞ며〮可가〯히〮ᄡᅥ〮百ᄇᆡᆨ〮里리〯
  人命며ᇰ〯을〮寄긔〮ᄒᆞ〮얌〮즉〮ᄒᆞ고〮大대〯節졀〮애〮臨림
  ᄒᆞ〮야〮可가〯히〮奪탈〮티〮몯〯ᄒᆞ리〮면〮君군子ᄌᆞ〮앳〮사〯
  ᄅᆞᆷ〮가〮君군子ᄌᆞ〮앳〮사〯ᄅᆞᆷ〮이〮니라〮}
\原{○曾즈ᇰ子ᄌᆞ〮ㅣ曰왈〮士ᄉᆞ〯ㅣ不블〮可가〯以이〯不블〮
  弘호ᇰ毅의〯니任ᅀᅵᆷ〯重듀ᇰ〯而ᅀᅵ道도〯遠원〯이니라}
\諺{曾즈ᇰ子ᄌᆞ〮ㅣᄀᆞᆯᄋᆞ샤〮ᄃᆡ〮士ᄉᆞ〯ㅣ可가〯히〮ᄡᅥ〮弘호ᇰ
  ᄒᆞ며〮毅의〯티〮아니〮티〮몯〯ᄒᆞᆯ꺼시〮니〮任ᅀᅵᆷ〯이〮重쥬ᇰ〯
  ᄒᆞ고〮道도〯ㅣ遠원〯ᄒᆞ니라〮}
\原{仁ᅀᅵᆫ以이〯爲위已긔〮任ᅀᅵᆷ〯이니不블〮亦역〮重듀ᇰ〯乎호
  아死ᄉᆞ〯而ᅀᅵ後후〯已이〯니不블〮亦역〮遠원〯乎호아}
\諺{仁ᅀᅵᆫ으〮로〮ᄡᅥ〮몸의〮任ᅀᅵᆷ〯을〮삼〯ᄂᆞ〮니〮ᄯᅩ〮ᄒᆞᆫ重듀ᇰ〯티〮
  아니〮ᄒᆞ냐〮죽은〮後후〯에〮마〯ᄂᆞ니〮ᄯᅩ〮ᄒᆞᆫ遠원〯티〮아
  니〮ᄒᆞ냐〮}
\原{○子ᄌᆞ〮ㅣ曰왈〮與호ᇰ於어詩시ᄒᆞ며}
\諺{子ᄌᆞ〮ㅣᄀᆞᆯᄋᆞ샤〮ᄃᆡ〮詩시예興호ᇰᄒᆞ며〮}
\原{立립〮於어禮례〮ᄒᆞ며}
\諺{禮례〮예立립〮ᄒᆞ며〮}
\原{成셔ᇰ於어樂악〮어니라}
\諺{樂악〮애〮成셔ᇰᄒᆞ〮ᄂᆞ〮니라〮}
\原{○子ᄌᆞ〮ㅣ曰왈〮民민은可가〯使ᄉᆞ〯由유之지오不
  블〮可가〯使ᄉᆞ〯知디〯之지니라}
\諺{子ᄌᆞ〮ㅣᄀᆞᆯᄋᆞ샤〮ᄃᆡ〮民민은〮可가〯히〮ᄒᆞ여〯곰〮由유
  케ᄒᆞ고〮可가〯히〮ᄒᆞ여곰〮알〯게〮몯〯ᄒᆞ〮ᄂᆞ니라}
\原{○子ᄌᆞ〮ㅣ曰왈〮好호〯勇요ᇰ〯疾질〮貧빈이亂란〯也야〯
  ㅣ오人ᅀᅵᆫ而ᅀᅵ不블〮仁ᅀᅵᆫ을疾잘〮之지已이〯甚심〯이
  亂란〯也야〯ㅣ니라}
\諺{子ᄌᆞ〮ㅣᄀᆞᆯᄋᆞ샤〮ᄃᆡ〮勇요ᇰ〮을〮됴〯히〮너기〮고〮貧빈을〮
  疾질〮홈〯이〮亂란〯홈〯이〮오〮사〯ᄅᆞᆷ〮이오〮仁ᅀᅵᆫ티〮아니〮
  ᄒᆞ니〮ᄅᆞᆯ〮疾질〮홈〯을〮너모甚심〯히〮홈〯이〮亂란〯홈〯이〮
  니라〮}
\原{○子ᄌᆞ〮ㅣ曰왈〮如ᅀᅧ有유〯周쥬公고ᇰ之지才ᄌᆡ之
  지美미〯오도使ᄉᆞ〯驕교且챠〯吝린〮이면其기餘여ᄂᆞᆫ不
  블〮足죡〮觀관也야〯已이〯니라}
\諺{子ᄌᆞ〮ㅣᄀᆞᆯᄋᆞ샤〮ᄃᆡ〮만〯일〮에〮周쥬公고ᇰ의〮才ᄌᆡ의〮
  美미〯홈〯을〮두고〮도〮ᄒᆞ여〯곰〮驕교ᄒᆞ고〮ᄯᅩ〮吝린〮ᄒᆞ
  면〮그나믄〮거슨〮足죡〮히〮보디〮몯〯홀〮꺼시〮니라〮}
\原{○子ᄌᆞ〮ㅣ曰왈〮三삼年년學ᄒᆞᆨ〮애不블〮至지〮於어
  穀곡〮을不블〮易이〯得득〮也야〯ㅣ니라}
\諺{子ᄌᆞ〮ㅣᄀᆞᆯᄋᆞ샤〮ᄃᆡ〮三삼年년을〮學ᄒᆞᆨ〮홈〯애〮穀곡〮
  애〮ᄠᅳᆫ〮ᄒᆞ디〮아니〮ᄒᆞ〮ᄂᆞ니〮ᄅᆞᆯ〮수〯이얻〯디〮몯〯ᄒᆞ리〮니
  라〮}
\原{○子ᄌᆞ〮ㅣ曰왈〮篤독〮信신〯好호〯學ᄒᆞᆨ〮ᄒᆞ며守슈〮死ᄉᆞ〯
  善션〯道도〯ㅣ니라}
\諺{子ᄌᆞ〮ㅣᄀᆞᆯᄋᆞ샤〮ᄃᆡ〮篤독〮히〮信신〯ᄒᆞ고〮도〮學ᄒᆞᆨ〮을〮
  好호〯ᄒᆞ며〮死ᄉᆞ〯를〮守슈〮ᄒᆞ고〮도〮道도〯를〮善션〯히〮
  홀〯띠〮니라〮}
\原{危위邦바ᇰ不블〮入십〮ᄒᆞ고亂란〯邦바ᇰ不블〮居거ᄒᆞ며天
  텬下하〯ㅣ有유〯道도〯則즉〮見현〯ᄒᆞ고無무道도〯則즉〮
  隱은〮이니라}
\諺{危위ᄒᆞᆫ邦바ᇰ애〮入ᅀᅵᆸ〮디〮아니〮ᄒᆞ고〮亂란〯ᄒᆞᆫ邦바ᇰ
  애〮居거티〮아니〮ᄒᆞ며〮天텬下하〯ㅣ道도〯ㅣ이시
  면〮見현〯ᄒᆞ고〮道도〯ㅣ업〯ᄉᆞ면〮隱은〮홀〯띠〮니라〮}
\原{邦바ᇰ有유〯道도〯애貧빈且챠〯賤쳔〯焉언이恥티〯也
  야〯ㅣ며邦바ᇰ無무道도〯애富부〯且챠〯貴귀〯焉언이恥
  티〯也야〯ㅣ니라}
\諺{邦바ᇰ이〮道도〯ㅣ이숌〯애〮貧빈ᄒᆞ고〮ᄯᅩ〮賤쳔〯홈〯이〮
  붓그〮러우며〮邦바ᇰ이〮道도〯ㅣ업〯슴〮애〮富부〯ᄒᆞ고〮
  ᄯᅩ〮貴귀〯홈〯이〮붓그〮러우니〮라〮}
\原{○子ᄌᆞ〮ㅣ曰왈〮不블〮在ᄌᆡ〯其기位위〮ᄒᆞ얀不블〮謀모
  其기政져ᇰ〮이니라}
\諺{子ᄌᆞ〮ㅣᄀᆞᆯᄋᆞ샤〮ᄃᆡ〮그位위〮예〮잇디〮아니〮ᄒᆞ〮얀〮그
  政져ᇰ〮을〮謀모〮티〮아니〮홀〯띠〯니라〮}
\原{○子ᄌᆞ〮ㅣ曰왈〮師ᄉᆞ摯지〮之지始시〯예〮關관雎져
  之지亂란〯이洋야ᇰ洋야ᇰ乎호盈여ᇰ耳ᅀᅵ〯哉ᄌᆡ라}
\諺{子ᄌᆞ〮ㅣᄀᆞᆯᄋᆞ샤〮ᄃᆡ〮師ᄉᆞ摯지〮의〮始시〯예〮關관雎
  져人亂란〯이〮洋야ᇰ洋야ᇰ히〮귀예〮盈여ᇰᄒᆞ다〮}
\原{○子ᄌᆞ〮ㅣ曰왈〮狂과ᇰ而ᅀᅵ不블〮直딕〮ᄒᆞ며侗토ᇰ而ᅀᅵ
  不블〮愿원〯ᄒᆞ며悾고ᇰ悾고ᇰ而ᅀᅵ不블〮信신〯을吾오不
  블〮知디之지矣의〯로라}
\諺{子ᄌᆞ〮ㅣᄀᆞᆯᄋᆞ샤〮ᄃᆡ〮狂과ᇰ호ᄃᆡ〮直딕〮디〮아니〮ᄒᆞ며〮
  侗토ᇰ호〯ᄃᆡ〮愿원〯티〮아니〮ᄒᆞ며〮悾고ᇰ悾고ᇰ호〯ᄃᆡ〮信
  신〯티〮아닌〮이ᄅᆞᆯ〮내〮아〯디〮몯〯ᄒᆞ〮노〮라〮}
\原{○子ᄌᆞ〮ㅣ曰왈〮學ᄒᆞᆨ〮如ᅀᅧ不블〮及급〮이오猶유恐고ᇰ
  失실〯之지니라}
\諺{子ᄌᆞ〮ㅣᄀᆞᆯᄋᆞ샤〮ᄃᆡ〮學ᄒᆞᆨ〮홈〯을〮밋디〮몯〯ᄒᆞᆯ듯〮ᄒᆞ고〮
  오〮히〮려〯일흘〮가〮저허〮홀〯띠〮니라〮}
\原{○子ᄌᆞ〮ㅣ曰왈〮巍외巍외乎호舜슌〯禹우〯之지右
  유〯天텬下하〯也야〯而ᅀᅵ不블〮與여〯焉언이여}
\諺{子ᄌᆞ〮ㅣᄀᆞᆯᄋᆞ샤〮ᄃᆡ〮巍외巍외ᄒᆞ다〮舜슌〯과禹우
  의〮天텬下하〯ᄅᆞᆯ〮두〮시〮되〮與여〯티〮아니〮ᄒᆞ〮심이[?]}
\原{○子ᄌᆞ〮ㅣ曰왈〮大대〯哉ᄌᆡ라堯요之지爲위君군
  也야〯ㅣ여巍외巍외乎호唯유天텬이爲위大대〯어시
  늘唯유堯요ㅣ則측〮之지ᄒᆞ시니蕩타ᇰ〯蕩타ᇰ〯乎호民
  민無무能느ᇰ名며ᇰ焉언이로다}
\諺{子ᄌᆞ〮ㅣᄀᆞᆯᄋᆞ샤〮ᄃᆡ크〮다堯요의〮님〯금되샴〮이여〮
  巍외巍외ᄒᆞ다〮오직〮하ᄂᆞᆯ〮히〮크〮거시늘〮오직堯
  요ㅣ則측〮ᄒᆞ〮시니〮蕩타ᇰ〯蕩타ᇰ〯ᄒᆞ다〮民민어〮能느ᇰ
  히〮일홈〯홈〯이업〯도다}
\原{巍외巍외乎호其기有유〯成셔ᇰ功고ᇰ也야〯ㅣ여煥환〯
  乎호其기有유〯文문章쟈ᇰ이여}
\諺{巍외巍외ᄒᆞ다〮그成셔ᇰ功고ᇰ이〮이숌〯이〮여〮煥환〯
  ᄒᆞ다〮그文문章쟈ᇰ이〮이숌〯이여}
\原{○舜슌〯이有유〯臣신五오〯人ᅀᅵᆫ而ᅀᅵ天텬下하〯ㅣ
  治티〯ᄒᆞ니라}
\諺{舜슌〯이〮신하〮다ᄉᆞᆺ〮사〯ᄅᆞᆷ〮을〮두〮심애〮天텬下하〯ㅣ
  다〮스니라〮}
\原{武무〯王와ᇰ이曰왈〮予여有유〯亂란〯臣신十십〮人ᅀᅵᆫ
  호라}
\諺{武무〯王와ᇰ이〮ᄀᆞᆯᄋᆞ샤〮ᄃᆡ〮내〮다ᄉᆞ〮리ᄂᆞᆫ〮신하〯얼〮싸〯
  ᄅᆞᆷ〮을〮둗〮노라〮}
\原{孔고ᇰ〮子ᄌᆞ〮ㅣ曰왈〮才ᄌᆡ難난이不블〮其기然ᅀᅧᆫ乎
  호아唐다ᇰ虞우之지際졔〯ㅣ於어斯ᄉᆞ爲위盛셔ᇰ〯
  ᄒᆞ나有유〯婦부〮人ᅀᅵᆫ焉언이라九구〮人ᅀᅵᆫ而ᅀᅵ巳이〯니라}
\諺{孔고ᇰ〮子ᄌᆞ〮ㅣᄀᆞᆯᄋᆞ샤〮ᄃᆡ〮才ᄌᆡ어렵〮다〮홈〯이〮그그
  러〮티〮아니〮ᄒᆞ냐〮唐다ᇰ虞우人際졔〯ㅣ이〮에〮셔〮盛
  셔ᇰ〯ᄒᆞ나〮婦부〮人ᅀᅵᆫ이인ᄂᆞᆫ디〮라〮아흡〮사〯ᄅᆞᆷ〮일〮ᄯᆞ
  ᄅᆞᆷ이니라〮}
\原{三삼分분天텬下하〯애有유〯其기二ᅀᅵ〯ᄒᆞ샤以이〯服
  복〮事ᄉᆞ〯殷은ᄒᆞ시니周쥬之지德덕〮은其기可가〯謂
  위〮至지〮德덕〮也야〯巳이〯矣의〯로다}
\諺{天텬下하〯를〮三삼分분홈〯애〮그둘〯흘〮두〮샤〮ᄡᅥ〮殷
  은을〮服복〮事ᄉᆞ〯ᄒᆞ〮시니〮周쥬의〮德덕〮은〮그可가〯
  히〮지〮극〮ᄒᆞᆫ德덕〮이〮라니ᄅᆞᆯᄯᆞᄅᆞᆷ이〮로다〮}
\原{○子ᄌᆞ〮ㅣ曰왈〮禹우〯ᄂᆞᆫ吾오無무間간〯然ᅀᅧᆫ矣의〮
  로다菲비〯飮음〯食식〮而ᅀᅵ致티〯孝효〯乎호鬼귀〯神[?]
  ᄒᆞ시며惡악〮衣의服복〮而ᅀᅵ致티〯美미〯乎호黻블〮[?]
  면〯ᄒᆞ시며卑비宮구ᇰ室실〮而ᅀᅵ盡진〯力력乎호溝구}
\諺{洫혁〮ᄒᆞ시니禹우〯ᄂᆞᆫ吾오無무間간〯然ᅀᅧᆫ矣의〯로다
  子ᄌᆞ〮ㅣᄀᆞᆯᄋᆞ샤〮ᄃᆡ〮禹우〯ᄂᆞᆫ내〮間간〯然ᅀᅧᆫ홈〯이업〯
  도다〮飮음〯食식〮을〮菲비〯히〮ᄒᆞ〮시고〮孝효〯를〮鬼귀〯
  神신애〮닐위〮시며〮衣의服복〮을〮惡악〮히ᄒᆞ〮시고
  美미〯를〮黻블〮冕면〯애〮닐위〮시며〮宮구ᇰ室실〮을〮ᄂᆞᆺ
  게〮ᄒᆞ〮시고〮힘〮을〮溝구洫혁〮애〮다〯ᄒᆞ〮시니〮禹우〯ᄂᆞᆫ〮
  내〮間간〯然ᅀᅧᆫ홈〯이〮업〯도다〮}

\篇{子ᄌᆞ〮罕한〯第뎨〯九구〮}
\原{子ᄌᆞ〮ᄂᆞᆫ罕한〯言언利리〯與여〯命며ᇰ〯與여〯仁ᅀᅵᆫ이러시다}
\諺{子ᄌᆞ〮ᄂᆞᆫ〮利리〯와〮다ᄆᆞᆺ〮命며ᇰ〯과〮다ᄆᆞᆺ〮仁ᅀᅵᆫ을〮져〯기〮
  니ᄅᆞ더〮시다〮}
\原{○達달〮卷하ᇰ〯黨다ᇰ〮人ᅀᅵᆫ〮이曰왈〮大대〯載ᄎᆡ라孔고ᇰ〮
  子ᄌᆞ〮ㅣ여博박〮學ᄒᆞᆨ〮而ᅀᅵ無무所소〯成셔ᇰ名며ᇰ이로다}
\諺{達달〮卷하ᇰ〯黨다ᇰ〮人ᅀᅵᆫ이〮ᄀᆞᆯ오〮ᄃᆡ〮크〮다孔고ᇰ〮子ᄌᆞ〮
  ㅣ여〮넙이〮學ᄒᆞᆨ〮호〯ᄃᆡ〮名며ᇰ을〮成셔ᇰᄒᆞᆫ배〮업〯도〮다}
\原{子ᄌᆞ〮ㅣ聞문之지ᄒᆞ시고謂위〮門문弟뎨〯子ᄌᆞ〮曰왈〮
  吾오何하執집〮고執집〮御어〯乎호아執집〮射샤〯乎
  호아吾오ㅣ執집〮御어〯矣의〯로리라}
\諺{子ᄌᆞ〮ㅣ드르시〮고門문弟뎨〯子ᄌᆞ〮ᄃᆞ려〮닐어〮ᄀᆞᆯ
  ᄋᆞ샤〮ᄃᆡ〮내〮므서〮슬〮執집〮ᄒᆞ료〮御어〯를〮執집〮ᄒᆞ랴〮
  射샤〯를〮執집〮ᄒᆞ랴〮내〮御어〯를〮執집호〯리〮라}
\原{○子ᄌᆞ〮ㅣ曰왈〮麻마冕면〯이禮례〮也야〯ㅣ어늘今금
  也야〯純슌ᄒᆞ니儉검〯이라吾오從죠ᇰ〮衆쥬ᇰ〯호리라}
\諺{子ᄌᆞ〮ㅣᄀᆞᆯᄋᆞ샤〮ᄃᆡ〮麻마로〮冕면〯이禮례〮어늘〮이〮
  제純슌오〮로〮ᄒᆞ니〮儉검〯ᄒᆞᆫ디〮라내〮衆쥬ᇰ〯을〮從죠ᇰ
  호〯리〮라}
\原{拜ᄇᆡ〯下하〯ㅣ禮례〮也야〯ㅣ어ᄂᆞᆯ今금拜ᄇᆡ〯乎호上샤ᇰ〯
  ᄒᆞ니泰태〮也야〯ㅣ라雖슈違위衆쥬ᇰ〯이나吾오從죠ᇰ下하〯
  호리라}
\諺{下하〯에〮셔拜ᄇᆡ〯홈〯이〮禮례〮어ᄂᆞᆯ〮이〮제上샤ᇰ〯에〮셔
  拜ᄇᆡ〯ᄒᆞ니〮泰태〮ᄒᆞᆫ디〮라비록〮衆쥬ᇰ〯을〮違위ᄒᆞ나〮
  내〮下하〯를〮從죠ᇰ호〯리〮라}
\原{○子ᄌᆞ〮ㅣ絶졀〮四ᄉᆞ〯ㅣ러시니毋무意의〯毋무必필〮毋
  무固고〯毋무我아〯ㅣ러시다}
\諺{子ᄌᆞ〮ㅣ四ᄉᆞ〯ㅣ絶졀〮터〮시니〮意의〯ㅣ업〯스〮며必
  필〮이업〯스〮며固고〯ㅣ업〯스〮며我아〯ㅣ업〯더〮시다〮}
\原{○子ᄌᆞ〮ㅣ畏외〯於어匡과ᇰ이러시니}
\諺{子ᄌᆞ〮ㅣ匡과ᇰ애〮畏외〯ᄒᆞ더시니}
\原{曰왈〮文문王와ᇰ이旣긔〮沒몰〮ᄒᆞ시니文문不블〮在ᄌᆡ〯
  玆ᄌᆞ乎호아}
\諺{ᄀᆞᆯᄋᆞ샤〮ᄃᆡ〮文문王와ᇰ이〮이믜〮沒몰ᄒᆞ〮시〮니文문
  이〮이〮예〮잇디〮아니〮ᄒᆞ〮냐}
\原{天텬之지將쟈ᇰ喪사ᇰ〯斯ᄉᆞ文문也야〯ㅣ신댄後후〯死
  ᄉᆞ〯者쟈〮ㅣ不블〮得득〮與여〯於어斯ᄉᆞ文문也야〯ㅣ어
  니와天텬〮之지未미〯喪사ᇰ〯斯ᄉᆞ文문也야〯ㅣ시니匡과ᇰ
  人ᅀᅵᆫ이其기如ᅀᅧ予여에何하ㅣ리오}
\諺{하ᄂᆞᆯ〮히쟈ᇰᄎᆞᆺ〮이〮文문을〮喪사ᇰ〯ᄒᆞ〮실띤〮댄〮後후〯에〮
  死ᄉᆞ〯ᄒᆞᆯ者쟈〮ㅣ시러〮곰〮이〮文문에〮與여〯티〮몯〯ᄒᆞ
  려〮니와하ᄂᆞᆯ〮히〮이〮文문을〮喪사ᇰ〯티〮아녀〮겨〯시〮니
  匡과ᇰ人人ᅀᅵᆫ이〮그내게엇〯디〮ᄒᆞ리〮오〮}
\原{○大태〮宰ᄌᆡ〯ㅣ問문〯於어子ᄌᆞ〮貢고ᇰ〯曰왈〮夫부子
  ᄌᆞ〮ᄂᆞᆫ聖셔ᇰ〯者쟈〮與여아何하〮其기多다能느ᇰ也야〯
  오}
\諺{大태〮宰ᄌᆡ〯ㅣ子ᄌᆞ〮貢고ᇰ〯의〮게무러〮ᄀᆞᆯ오〮ᄃᆡ夫부
  子ᄌᆞ〮ᄂᆞᆫ〮聖셔ᇰ〯이〮신者쟈〮가엇〯디〮그能느ᇰ이〮하〮시
  뇨〮}
\原{子ᄌᆞ〮貢고ᇰ〯이曰왈〮固고〯天텬縱죠ᇰ〯之지將쟈ᇰ聖셔ᇰ〯
  이시고又우〯多다能느ᇰ也야〯ㅣ시니라}
\諺{子ᄌᆞ〮貢고ᇰ〯이〮ᄀᆞᆯ오〮ᄃᆡ〮진실〮로天텬이〮縱죠ᇰ〯ᄒᆞ〮신
  쟈ᇰᄎᆞᆺ〮聖셔ᇰ〯이〮시고〮ᄯᅩ〮能느ᇰ이〮하〮시〮니라〮}
\原{子ᄌᆞ〮ㅣ聞문之지曰왈〮大태〮宰ᄌᆡ〯ㅣ知디我아〯乎
  호ᅟᅵᆫ뎌吾오ㅣ少쇼〯也야〯애賤쳔〯故고〮로多다能느ᇰ
  鄙비〯事ᄉᆞ〯호니君군子ᄌᆞ〮ᄂᆞᆫ多다乎호哉ᄌᆡ아不블〮
  多다也야〯ㅣ니라}
\諺{子ᄌᆞ〮ㅣ드ᄅᆞ시〮고〮ᄀᆞᆯᄋᆞ샤〮ᄃᆡ〮大태〮宰ᄌᆡ〯나〯ᄅᆞᆯ〮아〯
  ᄂᆞᆫ〮뎌〮내〮졈은〮제賤쳔〯ᄒᆞᆫ〮故고〮로鄙비〯ᄒᆞᆫ일〮ᄋᆞᆯ〮해〯
  能느ᇰ히〮호〯니〮君군子ᄌᆞ〮ᄂᆞᆫ多다ᄒᆞᆯ것가〮多다티〮
  아닐〮꺼시〮니라〮}
\原{牢로ㅣ曰왈〮子ᄌᆞ〮ㅣ云운吾오ㅣ不블〮試시〯故고〮
  로藝예〯라ᄒᆞ시니라}
\諺{牢로ㅣᄀᆞᆯ오〮ᄃᆡ〮子ᄌᆞ〮ㅣ닐ᄋᆞ샤〮ᄃᆡ〮내〮試시〯티〮몯〯
  ᄒᆞᆫ故고〮로藝예〯호〯라〮ᄒᆞ〮시〮니라〮}
\原{○子ᄌᆞ〮ㅣ曰왈〮吾오ㅣ有유〯知디乎호哉ᄌᆡ아無
  무知디也야〯ㅣ로라有유〯鄙비〯夫부ㅣ問문於어我
  아〯호ᄃᆡ空고ᇰ空고ᇰ如ᅀᅧ也야〯ㅣ라도我아〯ㅣ叩구〯其기
  兩랴ᇰ〯端단而ᅀᅵ竭갈〮焉언ᄒᆞ노라}
\諺{子ᄌᆞ〮ㅣᄀᆞᆯᄋᆞ샤〮ᄃᆡ〮내〮알옴〮이인ᄂᆞ냐〮알옴〮이업〯
  소〮라鄙비〯夫부ㅣ이셔〮내게무로〮ᄃᆡ〮空고ᇰ空고ᇰ
  ᄒᆞ〮야〮도내〮그두〯큰〮틀〮叩구〯ᄒᆞ〮야〮竭갈〮ᄒᆞ〮노라〮}
\原{○子ᄌᆞ〮ㅣ曰왈〮鳳보ᇰ〯鳥됴〮ㅣ不블〮至지〮ᄒᆞ며河하不
  를〮出츌〮圖도ᄒᆞ니吾오巳이矣의〯夫부ᅟᅵᆫ뎌}
\諺{子ᄌᆞ〮ㅣᄀᆞᆯᄋᆞ샤〮ᄃᆡ〮鳳보ᇰ〯鳥됴〮ㅣ니르〮디〮아니〮ᄒᆞ
  며〮河하애〮圖도ㅣ나디〮아니〮ᄒᆞ니〮내〮마를〮띤〮뎌}
\原{○子ᄌᆞ〮ㅣ見견〯齊ᄌᆡ衰최者쟈〮와冕면〯衣의裳샤ᇰ
  者쟈〮와與여〯瞽고〮者쟈〮ᄒᆞ시고見견〯之지예雖슈少
  쇼〯ㅣ나必필〮作작〮ᄒᆞ시며過과〯之지必필〮趨추ㅣ러시다}
\諺{子ᄌᆞ〮ㅣ齊ᄌᆡ衰최ᄒᆞᆫ〮者쟈〮와冕면〯ᄒᆞ고〮衣의裳
  샤ᇰᄒᆞᆫ者쟈〮와다ᄆᆞᆺ〮瞽고〮者쟈〮ᄅᆞᆯ보〮시고〮보〮심애〮
  비록〮少쇼〯ᄒᆞ나〮반〮ᄃᆞ〮시作작〮ᄒᆞ〮시며〮디〯나〮심애〮
  반〮ᄃᆞ시〮趨추ᄒᆞ〮더〮시다〮}
\原{○顏안淵연이喟위〮然ᅀᅧᆫ歎탄〯曰왈〮仰아ᇰ〯之지彌
  미高고ᄒᆞ며鑽찬〯之지彌미堅견ᄒᆞ며瞻쳠之지在ᄌᆡ〯
  前젼이러니忽홀〮焉언在ᄌᆡ〯後후〯ㅣ로다}
\諺{顏안淵연이〮喟위〮然ᅀᅧᆫ히〮歎탄〯ᄒᆞ야〮ᄀᆞᆯ오〮ᄃᆡ〮仰
  아ᇰ〯홈〯애〮더욱〮놉프〮며鑽찬〯홈〯애〮더욱〮구ᄃᆞ〮며瞻
  쳠홈〯애〮앏ᄑᆡ〮잇더〮니믄득〮뒤〯헤〮잇도〮다}
\原{夫부子ᄌᆞ〮ㅣ循슌循슌然ᅀᅧᆫ善션〯誘유〯人ᅀᅵᆫᄒᆞ샤博
  박〮我아〯以이〯文문ᄒᆞ시고約약〮我아〯以이〯禮례〮ᄒᆞ시니라}
\諺{夫부子ᄌᆞ〮ㅣ循슌循슌히〮사〯ᄅᆞᆷ〮을〮善션〯히〮誘유〯
  ᄒᆞ샤나〯ᄅᆞᆯ〮博박〮ᄒᆞ〮샤ᄃᆡ〮文문으〮로ᄡᅥ〮ᄒᆞ〯시고〮나〮
  ᄅᆞᆯ〮約약〮ᄒᆞ〮샤ᄃᆡ〮禮례〮로ᄡᅥ〮ᄒᆞ〮시니라〮}
\原{欲욕〮罷파〯不블〮能느ᇰᄒᆞ야旣긔〮竭갈〮吾오才ᄌᆡ호니如
  ᅀᅧ有유〯所소〯立립〮이卓탁〮爾ᅀᅵ〯라雖슈欲욕〮從죠ᇰ
  之지나末말〮由유也야〯巳이〯로다}
\諺{罷파〯코〮쟈ᄒᆞ나〮能느ᇰ티〮몯〯ᄒᆞ〮야〮임의〯내才ᄌᆡ를〮
  竭갈〮호〯니〮立립〮ᄒᆞᆫ배〮卓탁〮홈〯이〮인ᄂᆞᆫᄃᆞᆺ〮ᄒᆞᆫ디〮라
  비록〮좃고〮져〮ᄒᆞ나〮말ᄆᆡ〮암옴〮이업〯도〮다}
\原{○子ᄌᆞ〮ㅣ疾질〮病벼ᇰ〯이이시늘子ᄌᆞ〮路로〯ㅣ使ᄉᆞ〯門문
  人ᅀᅵᆫ으로爲위臣신이러니}
\諺{子ᄌᆞ〮ㅣ疾질〮이病벼ᇰ〯커〮시늘〮子ᄌᆞ〮路로〯ㅣ門문
  人ᅀᅵᆫ으〮로〮ᄒᆞ여〮곰〮臣신을〮사맛〮더니〮}
\原{病벼ᇰ〯間간曰왈〮久구〯矣의〯哉ᄌᆡ라由유之지行ᄒᆡᇰ
  詐사〯也야〯ㅣ여無무臣신而ᅀᅵ爲위有유〯臣신ᄒᆞ니吾
  오誰슈欺긔오欺긔天텬乎호ᅟᅵᆫ뎌}
\諺{病벼ᇰ〯이〮間간ᄒᆞ〮심애〮ᄀᆞᆯᄋᆞ샤〮ᄃᆡ〮올〮아〯다〮由유의〮
  詐사〯ᄅᆞᆯ〮行ᄒᆡᇰ홈〯이〮여〮臣신업〯슬〮꺼시〮臣신두〯믈〮
  ᄒᆞ니〮내〮누〯를〮소기〮료하ᄂᆞᆯ〮ᄒᆞᆯ〮소긴〮뎌}
\原{且챠〯予여ㅣ與여〯其기死ᄉᆞ〯於어臣신之지手슈〮
  也야〯론無무寧려ᇰ死ᄉᆞ〯於어二ᅀᅵ〯三삼子ᄌᆞ〮之지
  手슈〮乎호아且챠〯予여ㅣ縱죠ᇰ〯不블〮得득〮大대〯葬
  자ᇰ〯이나予여ㅣ死ᄉᆞ〯於어道도〯路로〯乎호아}
\諺{ᄯᅩ〮내〮그臣신의〮手슈〮에〮死ᄉᆞ〯홈〯으〮로더브〮러론〮
  二ᅀᅵ〯三삼子ᄌᆞ〮의〮手슈〮애〮死ᄉᆞ〯홈〯이〮ᄎᆞᆯ티〮아니
  ᄒᆞ냐〮ᄯᅩ〮내〮비록〮시러〯곰〮大대〯葬자ᇰ〯티〮몯〯ᄒᆞ나〮내〮
  道도〯路로〯애〮死ᄉᆞ〯ᄒᆞ랴〮}
\原{○子ᄌᆞ〮貢고ᇰ〯이曰왈〮有유〯美미〯玉옥〮於어斯ᄉᆞᄒᆞ니
  韞온〯匵독〮而ᅀᅵ藏자ᇰ諸져잇가求구善션〯賈가〯而ᅀᅵ
  沽고諸져잇가子ᄌᆞ〮ㅣ曰왈〮沽고之지哉ᄌᆡ沽고之
  지哉ᄌᆡ나我아〯ᄂᆞᆫ待ᄃᆡ〯賈가〯者쟈〮也야〯ㅣ로라}
\諺{子ᄌᆞ〮貢고ᇰ〯이〮ᄀᆞᆯ오〮ᄃᆡ〮美미〯ᄒᆞᆫ玉옥〮이〮이〮에〮이시
  니匵독〮애〮韞온〯ᄒᆞ〮야〮藏자ᇰᄒᆞ리ᇰ〮잇가〮善션〯ᄒᆞᆫ賈
  가〯ᄅᆞᆯ〮求구ᄒᆞ〮야〮沽고ᄒᆞ리ᇰ잇가〮子ᄌᆞ〮ㅣᄀᆞᆯᄋᆞ샤〮
  ᄃᆡ〮沽고ᄒᆞᆯ띠〮나〮沽고ᄒᆞᆯ띠〮나〮나〮ᄂᆞᆫ〮賈가〯ᄅᆞᆯ〮기ᄃᆞ〮
  리〮ᄂᆞᆫ〮者쟈〮ㅣ로라〮}
\原{○子ᄌᆞ〮ㅣ欲욕〮居거九구〮夷이러시니}
\諺{子ᄌᆞ〮ㅣ九구〮夷이예〮居거코〮져ᄒᆞ〮더〮시니〮}
\原{或혹〮曰왈〮陋루〯커니如ᅀᅧ之지何하잇고子ᄌᆞ〮ㅣ曰왈〮
  君군子ᄌᆞ〮ㅣ居거之지면何하陋루〯之지有유〯ㅣ[?|라]
  오}
\諺{或혹〮이ᄀᆞᆯ오〮ᄃᆡ〮陋루〯ᄒᆞ〮거니〮엇〯디〮ᄒᆞ리ᇰ잇고〮子
  ᄌᆞ〮ㅣᄀᆞᆯᄋᆞ샤〮ᄃᆡ〮君군子ᄌᆞ〮ㅣ居거ᄒᆞ면〮므슴〮陋
  루〯홈〯이〮이시리〮오〮}
\原{○子ᄌᆞ〮ㅣ曰왈〮吾오ㅣ自ᄌᆞ〮衛위〮反반〯魯로〮然ᅀᅧᆫ
  後후〯에樂악〮正져ᇰ〯ᄒᆞ야雅아〯頌쇼ᇰ〯이各각〮得득〮其기
  所소〯ᄒᆞ니라}
\諺{子ᄌᆞ〮ㅣᄀᆞᆯᄋᆞ샤〮ᄃᆡ〮내〮ㅣ衛위〮로브터〮魯로〮애〮도
  라〮온然ᅀᅧᆫ後후〯에〮樂악〮이正져ᇰ〯ᄒᆞ〮야〮雅아〯와〮頌
  쇼ᇰ〯이〮각〮각〮그所소〯를〮得득〮ᄒᆞ니라〮}
\原{○子ᄌᆞ〮ㅣ曰왈〮出츌〮則즉〮事ᄉᆞ〯公고ᇰ鄕겨ᇰᄒᆞ고入ᅀᅵᆸ〮
  則즉〮事ᄉᆞ〯父부〮兄혀ᇰᄒᆞ며喪사ᇰ事ᄉᆞ〯를不블〮敢간〯不
  블〮勉면〯ᄒᆞ며不블〮爲위酒슈〮困곤〯이何하有유〯於어
  我아〯哉ᄌᆡ오}
\諺{子ᄌᆞ〮ㅣᄀᆞᆯᄋᆞ샤〮ᄃᆡ〮나〮ᄂᆞᆫ〮公고ᇰ卿겨ᇰ을〮셤기〮고〮드〮
  러ᄂᆞᆫ〮父부〮兄혀ᇰ을〮셤기〮며〮喪사ᇰ事ᄉᆞ〯를〮敢감〯히〮
  힘〮ᄡᅳ〮디〮아니〮티〮아니〮ᄒᆞ며〮술의〮困곤〯홈〯이〮되디〮
  아니〮홈〯이〮므스〮거시〮내게인ᄂᆞ뇨〮}
\原{○子ᄌᆞ〮ㅣ在ᄌᆡ〯川쳔上샤ᇰ〯曰왈逝셔〯者쟈〮ㅣ如ᅀᅧ
  斯ᄉᆞ夫부ᅟᅵᆫ뎌不블〮舍샤〯晝듀〮夜야〯ㅣ로다}
\諺{子ᄌᆞ〮ㅣ川쳔上샤ᇰ〯의〮겨〯셔〮ᄀᆞᆯᄋᆞ샤〮ᄃᆡ〮逝셔〯ᄒᆞ〮ᄂᆞᆫ〮
  者쟈〮ㅣ이〮ᄀᆞ〮ᄐᆞᆫ뎌〮晝듀〮夜야〯의〮舍샤〯티〮아니〮ᄒᆞ〮
  놋〮다}
\原{○子ᄌᆞ〮ㅣ曰왈〮吾오未미〯見견〯好호德덕〮이如ᅀᅧ
  好호〮色ᄉᆡᆨ〮者쟈〮也야〯케라}
\諺{子ᄌᆞ〮ㅣᄀᆞᆯᄋᆞ샤〮ᄃᆡ〮내〮德덕〮을〮됴〯히〮너김〮이〮色ᄉᆡᆨ〮
  됴〯히〮너김〮ᄀᆞ〮티〮ᄒᆞ〮ᄂᆞᆫ〮이ᄅᆞᆯ〮보디〮몯〯게〮라}
\原{○子ᄌᆞ〮ㅣ曰왈〮譬비〯如ᅀᅧ爲위山산애未미〯成셔ᇰ
  一일〮簣궤〯ᄒᆞ야止지〮도吾오止지〮也야〯ㅣ며譬비〯如ᅀᅧ
  平펴ᇰ地디〮예雖슈覆복〮一일〮簣궤〯나進진〯도吾오
  往와ᇰ〯也야〯ㅣ니라}
\諺{子ᄌᆞ〮ㅣᄀᆞᆯᄋᆞ샤〮ᄃᆡ〮譬비〯컨〮댄〮뫼〯흘〮ᄆᆡᇰᄀᆞ롬애〮ᄒᆞᆫ
  簣궤〯를〮일오〮디〮몯〯ᄒᆞ〮야〮셔그침〯도〮내의〮그침〯이〮
  ᄀᆞ〮ᄐᆞ〮며譬비〯컨〮댄〮平펴ᇰ地디〮예〮비록〮ᄒᆞᆫ簣궤〯를〮
  覆복〮ᄒᆞ나〮나아〮감〯도〮내의〮감〯ᄀᆞ〮ᄐᆞ〮니라〮}
\原{○子ᄌᆞ〮ㅣ曰왈〮語어〯之지而ᅀᅵ不블〮惰타〯者쟈〮ᄂᆞᆫ
  其기回회也야〯與여ᅟᅵᆫ뎌}
\諺{子ᄌᆞ〮ㅣᄀᆞᆯᄋᆞ샤〮ᄃᆡ〮語어〯홈〯애〮惰타〯티〮아니〮ᄒᆞ〮ᄂᆞᆫ〮
  이〮ᄂᆞᆫ〮그回회ᅟᅵᆫ뎌}
\原{○子ᄌᆞ〮ㅣ謂위〮顏안淵연曰왈〮惜셕〮乎호ㅣ라吾오
  見견〯其기進진〯也야〯ㅣ오未미〯見견〯其기止지〮也야〯
  호리}
\諺{子ᄌᆞ〮ㅣ顏안淵연을〮닐어〮ᄀᆞᆯᄋᆞ샤〮ᄃᆡ〮惜셕〮홉〯다
  내〮그나아〮감〯을〮보고〮그그침〯을〮보디〮몯〯호〯라〮}
\原{○子ᄌᆞ〮ㅣ曰왈〮苗묘而ᅀᅵ不블〮秀슈〮者쟈〮ㅣ有유〯
  矣의〯夫부ㅣ며秀슈〮而ᅀᅵ不블〮實실〮者쟈〮ㅣ有유〯矣
  의〯夫부ᅟᅵᆫ뎌}
\諺{子ᄌᆞ〮ㅣᄀᆞᆯᄋᆞ샤〮ᄃᆡ〮苗묘ᄒᆞ고〮秀슈〮티〮몯〯ᄒᆞ리〮이
  시며〮秀슈〮ᄒᆞ고〮實실〮티〮몯〯ᄒᆞ리〮인〮ᄂᆞᆫ뎌〮}
\原{○子ᄌᆞ〮ㅣ曰왈〮後후〯生ᄉᆡᇰ이可가〯畏외〯니焉언知
  니來ᄅᆡ者쟈〮之지不블〮如ᅀᅧ今금也야〯ㅣ리오四ᄉᆞ〯
  十십〮五오〯十십〮而ᅀᅵ無무聞문焉언이면斯ᄉᆞ亦역
  不블〮足죡〮畏외〯也야〯巳이〯니라}
\諺{子ᄌᆞ〮ㅣᄀᆞᆯᄋᆞ샤〮ᄃᆡ〮後후〯生ᄉᆡᇰ이〮可가〯히〮두려〮오
  니〮엇〯디〮來ᄅᆡ者쟈〮의〮이〮제ᄀᆞᆮ〮디몯〯ᄒᆞᆯ줄〮을알〯리〮
  오四ᄉᆞ〯十십〮五오〯十십〮이〮오드름〯이업스〮면〮이〮
  ᄯᅩ〮ᄒᆞᆫ足죡〮히〮두〮렵〮디〮아니〮ᄒᆞ니〮라}
\原{○子ᄌᆞ〮ㅣ曰왈〮法법〮語어〯之지言언ᄋᆞᆫ能느ᇰ無무
  從죠ᇰ乎호아改ᄀᆡ〯之지爲위貴귀〯니라巽손〯與여〯之
  지言언ᄋᆞᆫ能느ᇰ無무說열〮乎호아繹역〮之지爲위
  貴귀〯니라說열〮而ᅀᅵ不블〮繹역〮ᄒᆞ며從죠ᇰ而ᅀᅵ不블〮改
  ᄀᆡ〯면吾오末말〮如ᅀᅧ之지〮何하也야〯已이〯矣의〯니라}
\諺{子ᄌᆞ〮ㅣᄀᆞᆯᄋᆞ샤〮ᄃᆡ〮法법〮으〮로語어〯ᄒᆞ〮ᄂᆞᆫ말〯ᄋᆞᆫ〮能
  느ᇰ히〮從죠ᇰ홈〯이〮업〯ᄉᆞ〮랴〮改ᄀᆡ〯홈〯이〮貴귀〯ᄒᆞ니〮라〮
  巽손〯히〮與여〯ᄒᆞ〮ᄂᆞᆫ말〯ᄋᆞᆫ〮能느ᇰ히〮說열〮홈〯이〮업〯ᄉᆞ〮
  랴〮繹역〮홈〯이〮貴귀〯ᄒᆞ니〮라〮說열〮호〯ᄃᆡ〮繹역〮디〮아
  니〮ᄒᆞ며〮從죠ᇰ호〯ᄃᆡ〮改ᄀᆡ〯티〮아니〮ᄒᆞ면〮내〮엇〯디〮려
  뇨〮홈〯이〮업〯ᄉᆞ〮니라〮}
\原{○子ᄌᆞ〮ㅣ曰왈〮主쥬〮忠튜ᇰ信신〯ᄒᆞ며毋무友우〯不블〮
  如ᅀᅧ巳긔〮者쟈〮ㅣ오過과〮則즉〮勿믈〮憚탄〯改ᄀᆡ〯니라\\
  ○子ᄌᆞ〮ㅣ曰왈〮三삼軍군ᄋᆞᆫ可가〯奪탈〮師슈〮也야〯
  ㅣ어니와匹필〮夫부ᄂᆞᆫ不블〮可가〯奪탈〮志지〮也야〯ㅣ니라}
\諺{子ᄌᆞ〮ㅣᄀᆞᆯᄋᆞ샤〮ᄃᆡ〮三삼軍군ᄋᆞᆫ〮可가〯히〮師슈〮를〮
  奪탈〮ᄒᆞ려〮니와〮匹필〮夫부ᄂᆞᆫ〮可가〯히〮志지〮를〮奪
  탈〮티〮몯〯ᄒᆞ〮ᄂᆞ니라〮}
\原{○子ᄌᆞ〮ㅣ曰왈〮衣의〯弊폐〯縕온〯袍포ᄒᆞ야與여〯衣의〯
  狐호貊락〮者쟈〮로立립〮而ᅀᅵ不블〮恥티〯者쟈〮ᄂᆞᆫ其
  기由유也야〯與여ᅟᅵᆫ뎌}
\諺{子ᄌᆞ〮ㅣᄀᆞᆯᄋᆞ샤〮ᄃᆡ〮ᄒᆞ〮여〮딘縕온〯袍포를〮닙어〮狐
  호貊락〮닙은〮이〮로〮더〮브〮러立립〮호〯ᄃᆡ〮붓그〮려아
  니〮ᄒᆞ〮ᄂᆞ니ᄂᆞᆫ〮그由유ᅟᅵᆫ뎌〮}
\原{不블〮忮기〯不블〮求구ㅣ면何하用요ᇰ〯不블〮臧자ᇰ이리오}
\諺{忮기〯티〮아니〮ᄒᆞ며〮求구티〮아니〮ᄒᆞ면〮엇〯디〮ᄡᅥ〮臧
  자ᇰ티〮아니〮ᄒᆞ리〮오〮}
\原{子ᄌᆞ〮路로〯ㅣ終죠ᇰ身신誦쇼ᇰ〯之지ᄒᆞᆫ대子ᄌᆞ〮ㅣ曰왈〮
  是시〯道도〯也야〯ㅣ何하足죡〮以이〯臧자ᇰ이리오}
\諺{子ᄌᆞ〮路로〯ㅣ몸〮이ᄆᆞᆺ도〮록〮외오〮려ᄒᆞᆫ대〮子ᄌᆞ〮ㅣ
  ᄀᆞᆯᄋᆞ샤〮ᄃᆡ〮이〮道도〯ㅣ엇〯디〮足죡〮히〮ᄡᅥ〮藏자ᇰᄒᆞ리〮
  오}
\原{○子ᄌᆞ〮ㅣ曰왈〮歲셰〯寒한然ᅀᅧᆫ後후〯에知디松쇼ᇰ
  柏ᄇᆡᆨ〮之지後후〯彫됴也야〯ㅣ니라}
\諺{子ᄌᆞ〮ㅣᄀᆞᆯᄋᆞ샤〮ᄃᆡ〮歲셰〯ㅣ寒한ᄒᆞᆫ然ᅀᅧᆫ後후〯에〮
  松쇼ᇰ柏ᄇᆡᆨ〮의〮後후〯에〮彫됴ᄒᆞ〮ᄂᆞᆫ줄〮을〮아〯ᄂᆞ〮니라〮}
\原{○子ᄌᆞ〮ㅣ曰왈〮知디〮者쟈〮ᄂᆞᆫ不블〮惑혹〮ᄒᆞ고仁ᅀᅵᆫ者
  자〮ᄂᆞᆫ不블〮憂우ᄒᆞ고勇요ᇰ〯者쟈〮ᄂᆞᆫ不블〮懼구〯ㅣ니라}
\諺{子ᄌᆞ〮ㅣᄀᆞᆯᄋᆞ샤〮ᄃᆡ〮知디〮ᄒᆞᆫ者쟈〮ᄂᆞᆫ惑혹〮디아니〮
  ᄒᆞ고〮仁ᅀᅵᆫᄒᆞᆫ者쟈ᄂᆞᆫ〮憂우티〮아니ᄒᆞ고勇요ᇰ〯ᄒᆞᆫ
  者쟈〮ᄂᆞᆫ〮懼구〯티〮아니〮ᄒᆞ〮ᄂᆞ〮니라〮}
\原{○子ᄌᆞ〮ㅣ曰왈〮可가〯與여〯共고ᇰ〯學ᄒᆞᆨ〮이오도未미〯可
  가〯與여〯適뎍〮道도〯ㅣ며可가〯與여〯適뎍〮道도〯ㅣ오도未
  미〯可가〯與여〯立립〮이며可가〯與여〯立립〮이오도未미〯可
  가〯與여〯權권이니라}
\諺{子ᄌᆞ〮ㅣᄀᆞᆯᄋᆞ샤〮ᄃᆡ〮可가〯히〮더브〮러〮ᄒᆞᆫ가지〮로學
  ᄒᆞᆨ〮ᄒᆞ고〮도〮可가〯히〮더브〮러〮道도〯애〮가디〮몯〯ᄒᆞ며〮
  可가〯히〮더브〮러〮道도〯애〮가고〮도〮可가〯히〮더브〮러〮
  立립〮디〮몯〯ᄒᆞ며〮可가〯히〮더브〮러〮立립ᄒᆞ고〮도〮可
  가〯히〮더브〮러〮權권티〮몯〯ᄒᆞ〮ᄂᆞ〮니라〮}
\原{○唐다ᇰ棣톄〮之지華화ㅣ여偏편其기反번而ᅀᅵ로다
  豈긔〮不블〮爾ᅀᅵ〯思ᄉᆞㅣ리오마ᄂᆞᆫ室실〮是시〯遠원〯而ᅀᅵ
  니라}
\諺{唐다ᇰ棣톄〮人고지〮여〮偏편히〮그反번〮ᄒᆞᄂᆞᄯᅩ다〮
  엇〯디〮너를〮思ᄉᆞ티〮아니ᄒᆞ리〮오마ᄂᆞᆫ〮塵실이이
  멀음〯이니라〮}
\原{子ᄌᆞ〮ㅣ曰왈〮未미〯之지思ᄉᆞ也야〯ㅣ언뎌ᇰ夫부何하
  遠원〯之지有유〯ㅣ리오}
\諺{子ᄌᆞ〮ㅣᄀᆞᆯᄋᆞ샤〮ᄃᆡ〮思ᄉᆞ티〮아니〮ᄒᆞ〮건뎌ᇰ〮엇〯디〮머
  름〯이〮이시리〮오〮}

\篇{鄕햐ᇰ黨다ᇰ〮第뎨〯十십〮}
\原{孔고ᇰ〮子ᄌᆞ〮ㅣ於어鄕햐ᇰ黨다ᇰ〮애恂슌恂슌如ᅀᅧ也
  야〯ᄒᆞ샤似ᄉᆞ〯不블〮能느ᇰ言언者쟈〮ㅣ러시다}
\諺{孔고ᇰ〮子ᄌᆞ〮ㅣ鄕햐ᇰ黨다ᇰ〮애〮恂슌恂슌ᄐᆞᆺ〮ᄒᆞ〮샤〮能
  느ᇰ히〮言언티〮몯〯ᄒᆞ〮ᄂᆞᆫ〮者쟈〮ᄀᆞᆮ〮더시다〮}
\原{其기在ᄌᆡ〯宗조ᇰ廟묘〯朝됴延뎌ᇰᄒᆞ샤ᄂᆞᆫ便변便변言
  언ᄒᆞ샤ᄃᆡ唯유謹근〯爾ᅀᅵ〯러시다}
\諺{그宗조ᇰ廟묘〯와〮朝됴廷뎌ᇰ에〮겨〯샤〮ᄂᆞᆫ〮便변便변
  히〮言언ᄒᆞ〮샤ᄃᆡ〮오직〮삼가〮더시다〮}
\原{○朝됴애與여〯下하〯大대〯夫부言언애侃간〯侃간〯
  如ᅀᅧ也야〯ᄒᆞ시며與여〯上샤ᇰ〯大대〯夫부言언애誾은
  誾은如ᅀᅧ也야〯ㅣ러시다}
\諺{朝됴애〮下하〯태〮우로〮더브〮러言언ᄒᆞ〮심애〮侃간〯
  侃간〯ᄐᆞᆺ〮ᄒᆞ〮시며〯上샤ᇰ〯태〮우로〮더〮브러〮言언ᄒᆞ〮심
  애〮誾은誾은ᄐᆞᆺ〮ᄒᆞ〮더시다〮}
\原{君군在ᄌᆡ〯어시든踧츅〮踖쳑〮如ᅀᅧ也야〯ᄒᆞ시며與이與
  여如ᅀᅧ也야〯ㅣ러시다}
\諺{君군이〮겨〯시거〮시든〮踧츅〮踖쳑〮ᄐᆞᆺ〮ᄒᆞ〮시며〮與어
  與여ᄐᆞᆺ〮ᄒᆞ〮더시다〮}
\原{○君군ㅣ召쇼使ᄉᆞ〯擯빈〮ㅣ어시든色ᄉᆡᆨ〮勃ᄇᆞᆯ〮如ᅀᅧ也
  야〯ᄒᆞ시며足죡〮躩확〮如ᅀᅧ也야〯ㅣ러시다}
\諺{君군이〮블러〮ᄒᆞ여〯곰〮擯빈〮ᄒᆞ라〮ᄒᆞ〮거시든〮色ᄉᆡᆨ〮
  이〮勃ᄇᆞᆯ〮ᄐᆞᆺ〮ᄒᆞ〮시며〮足죡〮이〮躩확〮ᄃᆞᆺ〮ᄒᆞ〮더시다〮}
\原{揖읍〮所소〯與여〯立립〮ᄒᆞ샤ᄃᆡ左자〯右우〯手슈〮ㅣ러시니衣
  의前젼後후〯ㅣ襜쳠如ᅀᅧ也야〯ㅣ러시다}
\諺{더브〮러〮立립〮ᄒᆞ〮신바〮애〮揖읍〮ᄒᆞ샤〮ᄃᆡ〮손〮을〮左자〯
  로〮ᄒᆞ며〮右우〯로〮ᄒᆞ〮더시니〮옷〮앏뒤〯히〮襜쳑ᄐᆞᆺ〮ᄒᆞ〮
  더시다〮}
\原{趨추進진〯에翼익〮如ᅀᅧ也야〯ㅣ러시다}
\諺{趨추ᄒᆞ〮야〮進진〯ᄒᆞ〮심애〮翼익〮ᄃᆞᆺ〮ᄒᆞ〮더시다〮}
\原{賓빈退퇴〯어든必필〮復복〮命며ᇰ〯曰왈〮賓빈不블〮顧고〯
  矣의〯라ᄒᆞ더시다}
\諺{賓빈이〮退퇴〯커든〮반〮ᄃᆞ시〮命며ᇰ〯을〮復복〮ᄒᆞ〮야〮ᄀᆞᆯ
  ᄋᆞ샤〮ᄃᆡ〮賓빈이顧고〯ᄒᆞ디〮아니〮타〮ᄒᆞ〮더시다〮}
\原{○入입〮公고ᇰ門문ᄒᆞ실시鞠국〮躬구ᇰ如ᅀᅧ也야〯ᄒᆞ샤如
  ᅀᅧ不블〮容요ᇰ이러시다}
\諺{公고ᇰ門문에〮드〮르실ᄉᆡ〮躬구ᇰ을〮鞠국〮ᄃᆞᆺ〮ᄒᆞ〮샤容
  요ᇰ티〮몯〯ᄒᆞᇎᄃᆞᆺ〮ᄒᆞ〮더시다〮}
\原{立립〮不블〮中듀ᇰ門문ᄒᆞ시며行ᄒᆡᇰ不블〮履리〯閾억〮이러
  시다}
\諺{立립〮ᄒᆞ심애〮門문에〮中듀ᇰ티〮아니〮ᄒᆞ〮시며〮行ᄒᆡᇰ
  ᄒᆞ〮심애〮閾역〮을〮ᄇᆞᆲ〯디〮아니〮ᄒᆞ〮더시다〮}
\原{過과〯位위〮ᄒᆞ실ᄉᆡ色ᄉᆡᆨ〮勃ᄇᆞᆯ〮如ᅀᅧ也야〯ᄒᆞ시〮며足죡〮躩
  확〮如ᅀᅧ也야〯ᄒᆞ시며其기言언어似ᄉᆞ〯不블〮足죡〮者
  쟈〮ㅣ러시다}
\諺{位위〮예〮디〯나실ᄉᆡ〮色ᄉᆡᆨ〮이〮勃ᄇᆞᆯ〮ᄐᆞᆺ〮ᄒᆞ〮시며〮足죡
  이〮躩확〮ᄃᆞᆺ〮ᄒᆞ〮시며〮그말〯ᄉᆞᆷ〮이〮足죡〮디〮몯〯ᄒᆞᆫ者쟈〮
  ᄀᆞᆮ〮더시다〮}
\原{攝셥〮齊ᄌᆡ升스ᇰ堂다ᇰᄒᆞ실ᄉᆡ鞠국〮躬구ᇰ如ᅀᅧ也야〯ᄒᆞ시
  며屛벼ᇰ〯氣긔〮ᄒᆞ샤似ᄉᆞ〯不블〮息식〮者쟈〮ㅣ러시다}
\諺{齊ᄌᆡ를〮攝셥〮ᄒᆞ〮야〮堂다ᇰ의〮오ᄅᆞ실〮ᄉᆡ〮躬구ᇰ을〮鞠
  국〮ᄃᆞᆺ〮ᄒᆞ〮시며〮氣긔〮를〮屛벼ᇰ〯ᄒᆞ〮샤〮息식디〮몯〯ᄒᆞ〮ᄂᆞᆫ〮
  者쟈〮ᄀᆞᆮ〮더시다〮}
\原{出츌〮降가ᇰ〯一일〮等드ᇰ〯ᄒᆞᄂᆞᆫ샤逞려ᇰ〯顏안色ᄉᆡᆨ〮ᄒᆞ샤怡이
  怡이如ᅀᅧ也야〯ᄒᆞ시며沒몰〮階계ᄒᆞ샤ᄂᆞᆫ趨추進진〯翼
  의〮如ᅀᅧ也야〯ᄒᆞ시며復복〮其기位위〮ᄒᆞ샤ᄂᆞᆫ踧츅踖쳑〮
  如ᅀᅧ也야〯ㅣ러시다}
\諺{出츌〮ᄒᆞ〮샤〮一일〮等드ᇰ〯에〮ᄂᆞ리〮샤〮ᄂᆞᆫ〮ᄂᆞᆺ빗〮츨〮逞려ᇰ〮
  ᄒᆞ〮샤〮怡이怡이ᄐᆞᆺᄒᆞ〮시며〮階계를〮沒몰〮ᄒᆞ〮샤〮ᄂᆞᆫ〮
  趨추ᄒᆞ심애〮翼익〮ᄃᆞᆺ〮ᄒᆞ〮시며〮그位위〮예〮復복〮ᄒᆞ〮
  샤〮ᄂᆞᆫ〮踧츅〮踖쳑〮ᄃᆞᆺ〮ᄒᆞ〮더시다〮}
\原{○執집〮圭규ᄒᆞ샤ᄃᆡ鞠국〮躬구ᇰ如ᅀᅧ也야〯ᄒᆞ샤如ᅀᅧ不
  블〮勝스ᇰ〯ᄒᆞ시며上샤ᇰ〯如ᅀᅧ揖읍〮ᄒᆞ시고下하〯如ᅀᅧ授슈〮
  ᄒᆞ시며勃ᄇᆞᆯ〮如ᅀᅧ戰젼〯色ᄉᆡᆨ〮ᄒᆞ시며足죡〮蹜츅〮蹜츅〮如
  ᅀᅧ有유〯循슌이러시다}
\諺{圭규를〮잡ᄋᆞ샤〮ᄃᆡ〮躬구ᇰ을〮鞠국〮ᄃᆞᆺ〮ᄒᆞ〮샤〮이긔〮디〮
  몯〯ᄒᆞᆯᄃᆞᆺ〮ᄒᆞ〮시며〮上샤ᇰ〯으〮로〮揖읍〮ᄃᆞᆺ〮ᄒᆞ〮시고〮下하〯
  로〮授슈〮ᄐᆞᆺ〮ᄒᆞ〮시며〮勃ᄇᆞᆯ〮히〮戰젼〯ᄒᆞ〮ᄂᆞᆫ色ᄉᆡᆨ〮ᄀᆞ〮ᄐᆞ〮
  시며〮足죡〮이〮蹜츅〮蹜츅〮ᄒᆞ〮야〮徇슌홈〯이〮인ᄂᆞᆫᄃᆞᆺ〮
  ᄒᆞ〮더시다}
\原{享햐ᇰ〯禮례〮예有유〯容요ᇰ色ᄉᆡᆨ〮ᄒᆞ시며}
\諺{享햐ᇰ〯ᄒᆞ〮ᄂᆞᆫ〮禮례〮예容요ᇰ色ᄉᆡᆨ〮이겨〯시며〮}
\原{私ᄉᆞ覿뎍〮에愉유愉유如ᅀᅧ也야〯ㅣ러시다}
\諺{私ᄉᆞ로〮覿뎍〮홈〯애愉유愉유ᄐᆞᆺ〮ᄒᆞ〮더시다〮}
\原{○君군子ᄌᆞ〮ᄂᆞᆫ不블〮以이〯紺감〯緅츄로飾식〮ᄒᆞ시며}
\諺{君군子ᄌᆞ〮ᄂᆞᆫ〮紺감〯과〮緅츄로〮ᄡᅥ〮飾식〮디〮아니〮ᄒᆞ〮
  시며〮}
\原{紅호ᇰ紫ᄌᆞ〮로不블〮以이〯爲위䙝셜〮服복〮이러시다}
\諺{紅호ᇰ과〮紫ᄌᆞ〮로〮ᄡᅥ〮䙝셜〮服복〮도〮ᄒᆞ디〮아니〮ᄒᆞ〮더
  시다〮}
\原{當다ᇰ暑셔〯ᄒᆞ샤袗딘〯絺티〯綌격〮을必필〮表표〮而ᅀᅵ出
  츌〮之지러시다}
\諺{暑셔〯를〮當다ᇰᄒᆞ〮샤〮홋絺티〯와〮綌격〮을〮반〮ᄃᆞ〮시表
  표〮ᄒᆞ〮야〮내〯더시다〮}
\原{緇츼衣의옌羔고裘구ㅣ오素소〯衣의옌魔예裘구
  ㅣ오黃화ᇰ衣의옌狐호裘구ㅣ러시다}
\諺{검은〮오샌〮羔고裘구ㅣ오〮흰〮오샌〮魔예裘구ㅣ
  오〮누른〮오샌〮狐호裘구ㅣ러시다〮}
\原{䙝셜〮裘구ᄂᆞᆫ長댜ᇰ호ᄃᆡ短단〯右우〯袂몌〯러시다}
\諺{䙝셜〮裘구ᄂᆞᆫ〮길〯게〮호〯ᄃᆡ〮을〮ᄒᆞᆫᄉᆞ〮매를댜ᄅᆞ게〮ᄒᆞ〮
  더시다〮}
\原{必필〮有유〯寢침〯衣의ᄒᆞ시니長댜ᇰ이一일〮身신有유〯
  半반〯이러라}
\諺{반〮ᄃᆞ〮시寢침〯衣의를〮두〮시니〮기릐〮一일〮身신이〮
  오〮ᄯᅩ〮半반〯이〮러라〮}
\原{狐호貊락〮之지厚후〯로以이〯居거ㅣ러시다}
\諺{狐호貊락〮의〮두터온〮거스〮로〮ᄡᅥ〮居거ᄒᆞ〮더시다〮}
\原{去거〯喪사ᇰᄒᆞ샤ᄂᆞᆫ無무所소〯不블〮佩패〯러시다}
\諺{喪사ᇰ을〮去거〯ᄒᆞ〮샤ᄂᆞᆫ〮ᄎᆞ〮디〮아니〮ᄒᆞᇎ배〮업〯더시다〮}
\原{非비帷유裳샤ᇰ이어든必필〮殺쇄〯之지러시다}
\諺{帷유裳샤ᇰ이〮아니〮어든〮반〮ᄃᆞ〮시殺쇄〯ᄒᆞ〮더시다}
\原{羔고裘구玄현冠관으로不블〮以이〯弔됴〮ㅣ러시다}
\諺{羔고裘구와〮玄현冠관으〮로ᄡᅥ〮弔됴〮티〮아니〮ᄒᆞ〮
  더시다〮}
\原{吉길〮月월〮에必필〮朝됴服복〮而ᅀᅵ朝됴ㅣ러시다}
\諺{吉길〮月월〮에〮반〮ᄃᆞ〮시朝됴服복〮ᄒᆞ고〮朝됴〮ᄒᆞ디
  시다〮}
\原{○齊ᄌᆡ必필〮有유〯明며ᇰ衣의러시니布포〯ㅣ러라}
\諺{齊ᄌᆡᄒᆞ〮실제반〮ᄃᆞ〮시明며ᇰ衣의〮를〮듯〮더시니〮布
  포〯ㅣ러라〮}
\原{濟ᄌᆡ必필〮變변〯食식〮ᄒᆞ시며居거必필遷쳔坐좌〯ㅣ러
  시다}
\諺{齊ᄌᆡᄒᆞ〮실〮제〮반〮ᄃᆞ〮시食식〮을〮變변〯ᄒᆞ〮시며〮居거
  홈〯을〮반〮ᄃᆞ〮시坐좌〯를〮遷쳔ᄒᆞ더시다〮}
\原{○食ᄉᆞ〯不블〮厭염〯精져ᇰᄒᆞ시며膾회〮不블〮厭임〯細세〯
  러시다}
\諺{食ᄉᆞ〯ᄂᆞᆫ〮精져ᇰ홈〯을〮厭염〯티〮아니〮ᄒᆞ〮시며〮膾회〯ᄂᆞᆫ〮
  細셰〯홈〯을〮厭염〯티〮아니〮ᄒᆞ〮더시다〮}
\原{食ᄉᆞ〯饐에〯而ᅀᅵ餲애〯와魚어餒뢰〯而ᅀᅵ肉ᅀᅲᆨ〮敗패〯
  를不블〮食식〮ᄒᆞ시며色ᄉᆡᆨ〮惡악〮不블〮食식〮ᄒᆞ시며臭ᄎᆔ〯
  惡악〮不블〮食식〮ᄒᆞ시며失실〮飪ᅀᅵᆷ〯不블〮食식〮ᄒᆞ시며不
  블時시不블食식〮이러시다}
\諺{食ᄉᆞ〯ㅣ饐에〯ᄒᆞ〮야〮餲애〯ᄒᆞ니〮와魚어ㅣ餒뢰〯ᄒᆞ
  며〮肉〮ᅀᅲᆨ〮이〮敗패〯ᄒᆞ니〮를〮食식〮디〮아니〮ᄒᆞ〮시며〮色
  ᄉᆡᆨ〮이〮惡악〮ᄒᆞ니〮를〮食식〮디〮아니〮ᄒᆞ〮시며〮臭ᄎᆔ〯ㅣ
  惡악〮ᄒᆞ니〮를〮食식〮디〮아니〮ᄒᆞ〮시며〮飪ᅀᅵᆷ〯을失실〮
  ᄒᆞ〮엿〮거든〮食식〮디〮아니〮ᄒᆞ〮시며〮時시ㅣ아니〮어
  든〮食식〮디〮아니〮터시다〮}
\原{割할〮不블〮正져ᇰ〯이어든不블〮食식〮ᄒᆞ시며不블〮得득〮其
  기醬자ᇰ〯이어든不블〮食식〮이러시다}
\諺{割할〮ᄒᆞᆫ거시〮正〮져ᇰ〯티〮아니〮커든〮食〮식〮디〮아니〮ᄒᆞ〮
  시며그醬쟈ᇰ〯을〮得득디〮몯〯ᄒᆞ〮야〮든〮食식〮디〮아니〮
  터시다〮}
\原{肉ᅀᅲᆨ〮雖슈多다ㅣ나不블〮使ᄉᆞ〯勝스ᇰ〯〮食ᄉᆞ〯氣긔〮ᄒᆞ시며
  唯유酒쥬〮無무量랴ᇰ〮ᄒᆞ샤ᄃᆡ不블〮及급〮亂란〯이러시다}
\諺{肉ᅀᅲᆨ〮이〮비록〮하나〮ᄒᆞ여〮곰〮食ᄉᆞ〯氣긔〮를〮勝스ᇰ〮케〮
  아니〮ᄒᆞ〮시며〮오직〮酒쥬〮ᄂᆞᆫ〮量랴ᇰ업〯시〮ᄒᆞ〮샤ᄃᆡ〮亂
  란〯에〮밋게〮아니〮터시다〮}
\原{沽고酒쥬〮市시〯脯포를不블〮食식〮ᄒᆞ시며}
\諺{沽고ᄒᆞᆫ酒쥬〮와〮市시〯ᄒᆞᆫ脯포를〮食식〮디〮아니〮ᄒᆞ〮
  시며〮}
\原{不블〮撤텰〮薑가ᇰ食식〮ᄒᆞ시며}
\諺{薑가ᇰ食식〮홈〯을〮撤텰〮티〮아니〮ᄒᆞ〮시며〮}
\原{不블〮多다食식〮이러시다}
\諺{해〯食식〮디〮아니〮터시다〮}
\原{祭졔〯於어公고ᇰ애不블〮宿슉〮肉ᅀᅲᆨ〮ᄒᆞ시며祭졔〯肉ᅀᅲᆨ〮
  은不블〮出츌〮三삼日일〮ᄒᆞ더시니出츌〮三삼日일〮이면不
  블〮食식〮之지矣의〯니라}
\諺{公고ᇰ애〮祭졔ᄒᆞ〮심애〮肉ᅀᅲᆨ〮을〮宿슉〮디〮아니〮ᄒᆞ〮시
  며〮祭졔〯肉ᅀᅲᆨ〮은〮三삼日일〮에〮出츌〮티〮아니〮ᄒᆞ〮더
  시니〮三삼日일〮에〮出츌〮ᄒᆞ면〮食식〮디〮몯〯ᄒᆞᆯ꺼시
  니라}
\原{食식〮不블〮語어〯ᄒᆞ시며寢침〯不블〮言언이라시다}
\諺{食식〮ᄒᆞ〮심애〮語어〯티〮아니〮ᄒᆞ〮시며〮寢침〯ᄒᆞ〮심애〮
  言언티〮아니〮터시다〮}
\原{雖슈疏소食ᄉᆞ〯菜ᄎᆡ〯羹ᄀᆡᇰ이라도瓜필〮祭졔〯ᄒᆞ샤ᄃᆡ必
  필〮齊졔如ᅀᅧ也야〯ㅣ러시다}
\諺{비록〮疏소食ᄉᆞ〯와〮菜ᄎᆡ〯羹ᄀᆡᇰ이라도〮반〮ᄃᆞ〮시祭
  졔〯ᄒᆞ〮샤ᄃᆡ〮반ᄃᆞ〮시齊졔ᄐᆞᆺ〮ᄒᆞ〮더〮시다〮}
\原{○席셕〮不블〮正져ᇰ〯이어든不블〮坐좌〯ㅣ러시다}
\諺{席셕〮이正져ᇰ〯티〮아니〮커든〮坐좌〯티〮아니〮터시다〮}
\原{○鄕햐ᇰ人ᅀᅵᆫ飮음〯酒쥬〮애杖댜ᇰ〯者쟈〮ㅣ出츌〮이어든
  斯ᄉᆞ出츌〮矣의〯러시다}
\諺{鄕햐ᇰ人ᅀᅵᆫ이〮酒쥬〮를〮飮음〯홈〯애〮杖댜ᇰ〯ᄒᆞᆫ者쟈〮ㅣ
  出츌〮ᄒᆞ거든〮이〮예〮出츌〮ᄒᆞ〮더시다〮}
\原{鄕햐ᇰ人ᅀᅵᆫ儺나애朝됴服복〮而ᅀᅵ立립〮於어阼조〯
  階계러시다}
\諺{鄕햐ᇰ人ᅀᅵᆫ이〮儺나홈〯애〮朝됴服복〮ᄒᆞ〮시고〮阼조〯
  階계예〮立립〮ᄒᆞ〮더시다〮}
\原{○問문〯人ᅀᅵᆫ於어他타邦바ᇰᄒᆞ실ᄉᆡ再ᄌᆡ〯拜ᄇᆡ〯而ᅀᅵ
  送소ᇰ〯之지러시다}
\諺{사〯ᄅᆞᆷ〮을〮다ᄅᆞᆫ나라〮ᄒᆡ〮무ᄅᆞ〮실ᄉᆡ〮再ᄌᆡ〯拜ᄇᆡ〯ᄒᆞ〮야〮
  보내〮더시다〮}
\原{康가ᇰ子ᄌᆞ〮ㅣ饋궤〯藥약〮이어늘拜ᄇᆡ〯而ᅀᅵ受슈〮之지
  曰왈〮丘구ㅣ未미〯達달〮이라不블〮敢감〯嘗샤ᇰ이라ᄒᆞ시다}
\諺{康가ᇰ子ᄌᆞ〮ㅣ藥약〮을〮饋궤〯ᄒᆞ〮야〮늘〮拜ᄇᆡ〯ᄒᆞ고〮受
  슈〮ᄒᆞ〮샤〮ᄀᆞᆯᄋᆞ샤〮ᄃᆡ〮丘구ㅣ達달〮티〮몯〯ᄒᆞᆫ디〮라敢
  감〯히〮嘗샤ᇰ티〮몯〯ᄒᆞ〮노라〮ᄒᆞ〮시다〮}
\原{○廐구〯焚분이어늘子ᄌᆞ〮ㅣ退퇴〯朝됴曰왈傷샤ᇰ人
  ᅀᅵᆫ乎호아ᄒᆞ시고不블〮問문〯馬미〯ᄒᆞ시다}
\諺{廐구〯ㅣ焚분커늘〮子ᄌᆞ〮ㅣ朝됴로〮退퇴〯ᄒᆞ〮샤〮ᄀᆞᆯ
  ᄋᆞ샤〮ᄃᆡ〮人ᅀᅵᆫ이〮傷샤ᇰᄒᆞ냐〮ᄒᆞ〮시고〮馬마〯를〮묻〯디〮
  아니〮ᄒᆞ〮시다〮}
\原{○君군이賜ᄉᆞ〯食식〮이어시든必필〮正져ᇰ〯席셕〮先션嘗
  샤ᇰ之지ᄒᆞ고시〮君군이賜ᄉᆞ〯腥셔ᇰ이어시든必필〮熟슉〮而
  ᅀᅵ薦쳔〯之지ᄒᆞ시고君군이賜ᄉᆞ〯生ᄉᆡᇰ이어시든必필〮畜
  휵〮之지러시다}
\諺{君군이〮食식〮을〮賜ᄉᆞ〯ᄒᆞ〮야〮시든〮반〮ᄃᆞ〮시席셕〮을〮
  正져ᇰ〯히〮ᄒᆞ고〮몬〮져嘗샤ᇰᄒᆞ〮시고〮君군이〮腥셔ᇰ을〮
  賜ᄉᆞ〯ᄒᆞ〮야〮시든〮반〮ᄃᆞ〮시熟슉ᄒᆞ〮야〮薦쳔〯ᄒᆞ〮시고〮
  君군이〮生ᄉᆡᇰ을〮賜ᄉᆞ〯ᄒᆞ〮야〮시든〮반〮ᄃᆞ시畜휵〮ᄒᆞ〮
  더시다〮}
\原{侍시〯食식〮於어君군애君군祭졔〯어시든先션飯반〮
  이러시다}
\諺{君군ᄭᅴ〮뫼〯셔〮食식ᄒᆞ〮실〮제〮君군이〮祭졔〯ᄒᆞ〮시거
  든〮몬져飯반〮ᄒᆞ〮더시다〮}
\原{○疾질〮에君군이視시〯之지어시든東도ᇰ首슈〮ᄒᆞ시고
  加가朝됴服복〮拖타〯紳신이러시다}
\諺{疾질〮에〮君군이〮視시〯ᄒᆞ〮거시든〮東도ᇰ으〮로首슈〮
  ᄒᆞ〮시고〮朝됴服복〮을〮加가ᄒᆞ〮시고〮紳신을〮施타〯
  ᄒᆞ〮더시다〮}
\原{君군이命며ᇰ〯召쇼〯ㅣ어시든不블〮俟ᄉᆞ〯駕가〮行ᄒᆡᇰ矣의〯
  러시다}
\諺{君군이〮命며ᇰ〯ᄒᆞ〮야〮召쇼〯ᄒᆞ〮거시든〮駕가〯를〮俟ᄉᆞ〯
  티〮아〮니〮ᄒᆞ〮시고〮行ᄒᆡᇰᄒᆞ〮더시다〮}
\原{○入ᅀᅵᆸ〮太태〮廟묘〯ᄒᆞ샤每ᄆᆡ〯事ᄉᆞ〯를問문〯이러시다
  ○朋브ᇰ友우〯ㅣ死ᄉᆞ〯ᄒᆞ야無무所소〯歸귀어든曰왈〮於
  어我아〯殯빈〮이라ᄒᆞ더시다}
\諺{朋브ᇰ友우〯ㅣ死ᄉᆞ〯ᄒᆞ〮야〮歸귀홀〯빼〮업〯거든〮ᄀᆞᆯᄋᆞ
  샤〮ᄃᆡ〮내게殯빈〮ᄒᆞ라〮ᄒᆞ〮더시다〮}
\原{朋브ᇰ友우〯之지饋궤〯ᄂᆞᆫ雖슈車거馬마〯ㅣ라도非비
  祭졔〯肉ᅀᅲᆨ〮이어든不블〮拜ᄇᆡ〯러시다}
\諺{朋브ᇰ友우〯의〮饋궤〯ᄂᆞᆫ〮비록〮車거〮馬마〯ㅣ라도〮祭
  졔〯肉ᅀᅲᆨ〮이아니〮어든〮拜ᄇᆡ〯티아니〮ᄒᆞ〮더시다}
\原{○寢침〯不블〮尸시ᄒᆞ시며居거不블〮容요ᇰ이러시다}
\諺{寢침〯홈〯애〮尸시티〮아니〮ᄒᆞ〮시며〮居거홈〯애〮容요ᇰ
  티〮아니〮터시다〮}
\原{見견〯齊ᄌᆡ衰최者쟈〮ᄒᆞ시고雖슈狎압〮이나必필〮變변〯
  ᄒᆞ시며見견〯冕면〯者쟈〮與여〯瞽고〮者쟈〮ᄒᆞ시고雖슈䙝
  셜〮이나必필〮以이〯貌모〯ㅣ러시다}
\諺{齊ᄌᆡ衰최ᄒᆞᆫ者쟈〮를〮보〮시고〮비록〮狎압〮ᄒᆞ나〮반〮
  ᄃᆞ〮시變변〯ᄒᆞ〮시며〮冕면〯ᄒᆞᆫ者쟈〮와〮다ᄆᆞᆺ〮瞽고〮ᄒᆞᆫ
  者쟈〮를〮보〮시고〮비록〮䙝셜〮ᄒᆞ나〮반〮ᄃᆞ〮시ᄡᅥ〮貌모〯
  ᄒᆞ〮더시다〮}
\原{凶휴ᇰ服복〮者쟈〮를式식〮之지ᄒᆞ시며式식〮負부〯版판〮
  者쟈〮ㅣ러시다}
\諺{凶휴ᇰ服복〮ᄒᆞᆫ者쟈〮를式식〮ᄒᆞ〮시며〮版판〮負부〯ᄒᆞᆫ
  者쟈〮를〮式식〮ᄒᆞ〮더시다〮}
\原{有유〯盛셔ᇰ〯饌찬〯이어든必필〮變변〯色ᄉᆡᆨ〮而ᅀᅵ作작〮이러
  시다}
\諺{盛셔ᇰ〯ᄒᆞᆫ饌찬〯이〮잇거든〮반〮ᄃᆞ〮시色ᄉᆡᆨ〮을〮變변〯ᄒᆞ〮
  시고〮作작〮ᄒᆞ〮더시다〮}
\原{迅신〯雷뢰風푸ᇰ烈렬〮에必필〮變변〯이러시다}
\諺{迅신〯ᄒᆞᆫ雷뢰〮와〮風푸ᇰ이〮烈렬〮홈〯애〮반〮ᄃᆞ〮시變변〯
  ᄒᆞ〮더시다}
\原{○升스ᇰ車거ᄒᆞ샤必필〮正져ᇰ〯立립〮執집〮綏유ㅣ러시다}
\諺{車거에〮升스ᇰᄒᆞ〮샤〮반〮ᄃᆞ시正져ᇰ〯히〮立립〮ᄒᆞ〮샤〮綏
  유를〮執집〮ᄒᆞ〮더시다〮}
\原{車거中듀ᇰ애不블〮內ᄂᆡ〯顧고〮ᄒᆞ시며不블〮疾질〮言언
  ᄒᆞ시며不블〮親친指지〮러시다}
\諺{車거中듀ᇰ애〮內ᄂᆡ〮顧고〯티〮아니〮ᄒᆞ〮시며〮疾질〮히〮
  言언티〮아니〮ᄒᆞ〮시며〮親친히〮指지〮티〮아니〮터〮시
  다〮}
\原{○色ᄉᆡᆨ〮斯ᄉᆞ擧거〯矣의〯ᄒᆞ야翔샤ᇰ而ᅀᅵ後후〯集집〮이니
  라}
\諺{色ᄉᆡᆨ〮ᄒᆞ고〮이〮에〮擧거〯ᄒᆞ〮야〮翔샤ᇰᄒᆞᆫ後후〯에〮集집〮
  ᄒᆞ〮ᄂᆞ〮니라〮}
\原{曰왈〮山산粱랴ᇰ雌ᄌᆞ雉티〯ㅣ時시哉ᄌᆡ時시哉ᄌᆡ
  ㄴ뎌子ᄌᆞ〮路로〯ㅣ共고ᇰ〯之지ᄒᆞᆫ대三삼嗅후〯而ᅀᅵ作작〯
  ᄒᆞ시다}
\諺{ᄀᆞᆯᄋᆞ샤〮ᄃᆡ〮山산粱랴ᇰ엣〮雌ᄌᆞ雉티〯ㅣ時시ㄴ뎌〮
  時시ㄴ뎌〮子ᄌᆞ〮路로〯ㅣ共고ᇰ〯ᄒᆞᆫ대〮세〯번嗅후〯ᄒᆞ〮
  시고〮作작ᄒᆞ〮시다〮}
\文[parsing.assistIdeograph(1)]{[nextColumn|1]論론語어〯諺언〯解ᄒᆡ〯卷권〯之지二ᅀᅵ〯}
\end{document}

%%% Local Variables:
%%% mode: latex 
%%% TeX-engine: luatex
%%% TeX-master: t 
%%% End:
